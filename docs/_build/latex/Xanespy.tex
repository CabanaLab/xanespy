%% Generated by Sphinx.
\def\sphinxdocclass{report}
\documentclass[letterpaper,10pt,english]{sphinxmanual}
\ifdefined\pdfpxdimen
   \let\sphinxpxdimen\pdfpxdimen\else\newdimen\sphinxpxdimen
\fi \sphinxpxdimen=.75bp\relax

\usepackage[utf8]{inputenc}
\ifdefined\DeclareUnicodeCharacter
 \ifdefined\DeclareUnicodeCharacterAsOptional\else
  \DeclareUnicodeCharacter{00A0}{\nobreakspace}
\fi\fi
\usepackage{cmap}
\usepackage[T1]{fontenc}
\usepackage{amsmath,amssymb,amstext}
\usepackage{babel}
\usepackage{times}
\usepackage[Bjarne]{fncychap}
\usepackage[dontkeepoldnames]{sphinx}

\usepackage{geometry}

% Include hyperref last.
\usepackage{hyperref}
% Fix anchor placement for figures with captions.
\usepackage{hypcap}% it must be loaded after hyperref.
% Set up styles of URL: it should be placed after hyperref.
\urlstyle{same}
\addto\captionsenglish{\renewcommand{\contentsname}{Contents:}}

\addto\captionsenglish{\renewcommand{\figurename}{Fig.}}
\addto\captionsenglish{\renewcommand{\tablename}{Table}}
\addto\captionsenglish{\renewcommand{\literalblockname}{Listing}}

\addto\extrasenglish{\def\pageautorefname{page}}

\setcounter{tocdepth}{1}



\title{Xanespy Documentation}
\date{Jul 13, 2017}
\release{0.1.2}
\author{Mark Wolf}
\newcommand{\sphinxlogo}{\vbox{}}
\renewcommand{\releasename}{Release}
\makeindex

\begin{document}

\maketitle
\sphinxtableofcontents
\phantomsection\label{\detokenize{index::doc}}


Python library for analyzing \sphinxstylestrong{X-Ray absorbance spectroscopy} data.

\begin{sphinxadmonition}{warning}{Warning:}
This documentation is still under development. While it is usable, the code
could change at any time.
\end{sphinxadmonition}


\chapter{Introduction}
\label{\detokenize{intro::doc}}\label{\detokenize{intro:introduction}}\label{\detokenize{intro:xanespy-documentation}}
Xanespy is a toolkit for interacting with X-ray microscopy data, most
likely collected at a synchrotron beamline. By collecting a set of
frames at multiple X-ray energies, spectral maps are reconstructed to
provide chemical insight. Multiple framesets can be collected
sequentially as part of an \sphinxstyleemphasis{operando} experiment and analyzed
simultaneously in python. Slow operations take advantage of multiple
cores when available.

This project has the following design goals:
\begin{itemize}
\item {} 
Provide a python toolkit for analysis of X-ray absorbance frames.

\item {} 
Store data in an open format for easy distribution.

\end{itemize}

GUI tools (eg. TXM-Wizard) exist for performing this type of
analysis. While convenient, the downside to this approach is the
potential inability to exactly reproduce a given set of steps. Xanespy
does provide an interactive GUI for visualizing the data, but this GUI
does not alter the data or export results. This way, the analysis
steps are caputed either in an IPython notebook or conventional python
script. These steps, together with the original data, should then be
sufficient to reproduce the results exactly.


\section{Installation}
\label{\detokenize{intro:installation}}
Xanespy can be installed from the \sphinxstylestrong{python package index (PyPI) using pip}

\begin{sphinxVerbatim}[commandchars=\\\{\}]
\PYGZdl{} pip install xanespy
\end{sphinxVerbatim}

or from anaconda.org using \sphinxstylestrong{conda}

\begin{sphinxVerbatim}[commandchars=\\\{\}]
\PYGZdl{} conda install \PYGZhy{}c canismarko xanespy
\end{sphinxVerbatim}


\section{X-Ray Absorbance Basics}
\label{\detokenize{intro:x-ray-absorbance-basics}}
Coming soon…


\section{Example Workflow}
\label{\detokenize{intro:example-workflow}}
A typical prcocedure for interacting with microscope frame-sets involves the following parts:
\begin{itemize}
\item {} 
Import the raw data

\item {} 
Apply corrections and align the images

\item {} 
Calculate some metric and create maps of it

\item {} 
Visualize the maps, staticly or interactively.

\end{itemize}

Example for a single frameset across an X-ray absorbance edge:

\begin{sphinxVerbatim}[commandchars=\\\{\}]
\PYG{k+kn}{import} \PYG{n+nn}{xanespy}

\PYG{c+c1}{\PYGZsh{} Example for importing from SSRL beamline 6\PYGZhy{}2c}
\PYG{n}{xanespy}\PYG{o}{.}\PYG{n}{import\PYGZus{}ssrl\PYGZus{}frameset}\PYG{p}{(}\PYG{l+s+s1}{\PYGZsq{}}\PYG{l+s+s1}{\PYGZlt{}data\PYGZus{}dir\PYGZgt{}}\PYG{l+s+s1}{\PYGZsq{}}\PYG{p}{,} \PYG{n}{hdf\PYGZus{}filename}\PYG{o}{=}\PYG{l+s+s1}{\PYGZsq{}}\PYG{l+s+s1}{imported\PYGZus{}data.h5}\PYG{l+s+s1}{\PYGZsq{}}\PYG{p}{)}

\PYG{c+c1}{\PYGZsh{} Load a pre\PYGZhy{}defined XAS edge or create your own subclass xanespy.Edge}
\PYG{n}{edge} \PYG{o}{=} \PYG{n}{xanespy}\PYG{o}{.}\PYG{n}{k\PYGZus{}edges}\PYG{p}{[}\PYG{l+s+s1}{\PYGZsq{}}\PYG{l+s+s1}{Ni\PYGZus{}NCA}\PYG{l+s+s1}{\PYGZsq{}}\PYG{p}{]}
\PYG{c+c1}{\PYGZsh{} Now load the newly created HDF5 file and the X\PYGZhy{}ray absorbance edge}
\PYG{n}{fs} \PYG{o}{=} \PYG{n}{xanespy}\PYG{o}{.}\PYG{n}{XanesFrameset}\PYG{p}{(}\PYG{n}{filename}\PYG{o}{=}\PYG{l+s+s1}{\PYGZsq{}}\PYG{l+s+s1}{imported\PYGZus{}data.h5}\PYG{l+s+s1}{\PYGZsq{}}\PYG{p}{,} \PYG{n}{edge}\PYG{o}{=}\PYG{n}{edge}\PYG{p}{)}

\PYG{c+c1}{\PYGZsh{} Perform automatic frame alignment}
\PYG{n}{fs}\PYG{o}{.}\PYG{n}{align\PYGZus{}frames}\PYG{p}{(}\PYG{n}{passes}\PYG{o}{=}\PYG{l+m+mi}{5}\PYG{p}{)}
\PYG{c+c1}{\PYGZsh{} Fit the absorbance spectra and extract the edge position (SLOW!)}
\PYG{n}{fs}\PYG{o}{.}\PYG{n}{fit\PYGZus{}spectra}\PYG{p}{(}\PYG{p}{)}

\PYG{c+c1}{\PYGZsh{} Inspect the result with the built\PYGZhy{}in Qt5 GUI}
\PYG{n}{fs}\PYG{o}{.}\PYG{n}{qt\PYGZus{}viewer}\PYG{p}{(}\PYG{p}{)}
\end{sphinxVerbatim}


\chapter{Importing Data into Xanespy}
\label{\detokenize{importing::doc}}\label{\detokenize{importing:importing-data-into-xanespy}}
The first step in any Xanespy workflow will be to \sphinxstylestrong{import the raw
data into a common format}. These importer functions are written as
needed: if your preferred beamline is not here, \sphinxhref{https://github.com/m3wolf/xanespy/issues}{submit an issue}.


\section{APS Beamline 8-BM-B - Energy Stack (TXRM)}
\label{\detokenize{importing:aps-beamline-8-bm-b-energy-stack-txrm}}
\begin{sphinxadmonition}{note}{Note:}
Currently this function can only import one XANES
stack. Time-resolved measurement is not implemented. If you would
find this feature valuable, please \sphinxhref{https://github.com/m3wolf/xanespy/issues}{submit an issue}.
\end{sphinxadmonition}

The Xradia microscope can save an entire stack in one \sphinxcode{.txrm}
file. This file can be imported using the
{\hyperref[\detokenize{xanespy:xanespy.importers.import_aps_8BM_xanes_file}]{\sphinxcrossref{\sphinxcode{import\_aps\_8BM\_xanes\_file()}}}} function. The
list of energies is automatically extracted from the file. The
reference frames will then reside in a different \sphinxcode{.txrm} file.

Example usage:

\begin{sphinxVerbatim}[commandchars=\\\{\}]
\PYG{k+kn}{import} \PYG{n+nn}{xanespy} \PYG{k}{as} \PYG{n+nn}{xp}

\PYG{n}{xp}\PYG{o}{.}\PYG{n}{import\PYGZus{}aps\PYGZus{}8BM\PYGZus{}xanes\PYGZus{}file}\PYG{p}{(}\PYG{l+s+s1}{\PYGZsq{}}\PYG{l+s+s1}{exp1\PYGZhy{}sample\PYGZhy{}stack.txrm}\PYG{l+s+s1}{\PYGZsq{}}\PYG{p}{,}
                             \PYG{n}{ref\PYGZus{}filename}\PYG{o}{=}\PYG{l+s+s1}{\PYGZsq{}}\PYG{l+s+s1}{exp1\PYGZhy{}reference\PYGZus{}stack.txrm}\PYG{l+s+s1}{\PYGZsq{}}\PYG{p}{,}
                             \PYG{n}{hdf\PYGZus{}filename}\PYG{o}{=}\PYG{l+s+s1}{\PYGZsq{}}\PYG{l+s+s1}{txm\PYGZhy{}data.h5}\PYG{l+s+s1}{\PYGZsq{}}\PYG{p}{,}
                             \PYG{n}{groupname}\PYG{o}{=}\PYG{l+s+s1}{\PYGZsq{}}\PYG{l+s+s1}{experiment1}\PYG{l+s+s1}{\PYGZsq{}}\PYG{p}{)}
\end{sphinxVerbatim}


\section{APS Beamline 8-BM-B - Directory of XRM Files}
\label{\detokenize{importing:aps-beamline-8-bm-b-directory-of-xrm-files}}
In-house XANES scan scripts often save a directory full of \sphinxcode{.xrm}
files with the metadata coding in the filenames. From the Xradia TXM
at sector 8-BM-B, this XANES scan script can be generated with
{\hyperref[\detokenize{xanespy:xanespy.beamlines.sector8_xanes_script}]{\sphinxcrossref{\sphinxcode{sector8\_xanes\_script()}}}}, and the results
can then be imported with
{\hyperref[\detokenize{xanespy:xanespy.importers.import_aps_8BM_xanes_dir}]{\sphinxcrossref{\sphinxcode{import\_aps\_8BM\_xanes\_dir()}}}}. The list of
energies is automatically extracted from the filenames. The reference
frames will also be identified in the directory.

Example usage:

\begin{sphinxVerbatim}[commandchars=\\\{\}]
\PYG{k+kn}{import} \PYG{n+nn}{xanespy} \PYG{k}{as} \PYG{n+nn}{xp}

\PYG{c+c1}{\PYGZsh{} First a script should be created with sector\PYGZus{}8\PYGZus{}xanes\PYGZus{}script()}
\PYG{c+c1}{\PYGZsh{} Once the script is done, import the data with this function}
\PYG{n}{xp}\PYG{o}{.}\PYG{n}{import\PYGZus{}aps\PYGZus{}8BM\PYGZus{}xanes\PYGZus{}dir}\PYG{p}{(}\PYG{l+s+s2}{\PYGZdq{}}\PYG{l+s+s2}{opearando\PYGZus{}exp1/}\PYG{l+s+s2}{\PYGZdq{}}\PYG{p}{,}
                            \PYG{n}{hdf\PYGZus{}filename}\PYG{o}{=}\PYG{l+s+s2}{\PYGZdq{}}\PYG{l+s+s2}{operando\PYGZus{}experiments.h5}\PYG{l+s+s2}{\PYGZdq{}}\PYG{p}{)}
\end{sphinxVerbatim}


\section{SSRL Beamline 6-2c - Directory of XRM Files}
\label{\detokenize{importing:ssrl-beamline-6-2c-directory-of-xrm-files}}
In-house XANES scan scripts often save a directory full of \sphinxcode{.xrm}
files with the metadata coding in the filenames. From the Xradia TXM
at SSRL beamline 6-2c, this XANES scan script can be generated with
the in-house generator, and the results can then be imported with
{\hyperref[\detokenize{xanespy:xanespy.importers.import_ssrl_xanes_dir}]{\sphinxcrossref{\sphinxcode{import\_ssrl\_xanes\_dir()}}}}. The list of
energies is automatically extracted from the filenames. The reference
frames will also be identified in the directory.

Example usage:

\begin{sphinxVerbatim}[commandchars=\\\{\}]
\PYG{k+kn}{import} \PYG{n+nn}{xanespy} \PYG{k}{as} \PYG{n+nn}{xp}

\PYG{c+c1}{\PYGZsh{} First a script should be created with sector\PYGZus{}8\PYGZus{}xanes\PYGZus{}script()}
\PYG{c+c1}{\PYGZsh{} Once the script is done, import the data with this function}
\PYG{n}{xp}\PYG{o}{.}\PYG{n}{import\PYGZus{}ssrl\PYGZus{}xanes\PYGZus{}dir}\PYG{p}{(}\PYG{l+s+s2}{\PYGZdq{}}\PYG{l+s+s2}{opearando\PYGZus{}exp1/}\PYG{l+s+s2}{\PYGZdq{}}\PYG{p}{,}
                         \PYG{n}{hdf\PYGZus{}filename}\PYG{o}{=}\PYG{l+s+s2}{\PYGZdq{}}\PYG{l+s+s2}{operando\PYGZus{}experiments.h5}\PYG{l+s+s2}{\PYGZdq{}}\PYG{p}{)}
\end{sphinxVerbatim}


\section{Ptychography from 5.3.2.1 (ALS)}
\label{\detokenize{importing:ptychography-from-5-3-2-1-als}}
The output of the nanosurveyor reconstruction algorithm at 5.3.2.1
saves the data in h5
files. \sphinxcode{import\_nanosurveyor\_frameset{}`()}
copies the reconstructed images and metadata from the individual files
and combines them into a new HDF5 file for XAS analysis. The original
CCD images are left in their original HDF5 files, so they should not
be discarded.

\begin{sphinxVerbatim}[commandchars=\\\{\}]
\PYG{k+kn}{import} \PYG{n+nn}{xanespy} \PYG{k}{as} \PYG{n+nn}{xp}

\PYG{c+c1}{\PYGZsh{} This function copies the reconstructed images to a new file.}
\PYG{n}{xp}\PYG{o}{.}\PYG{n}{import\PYGZus{}nanosurveyor\PYGZus{}frameset}\PYG{p}{(}\PYG{l+s+s1}{\PYGZsq{}}\PYG{l+s+s1}{NS\PYGZus{}160529047/}\PYG{l+s+s1}{\PYGZsq{}}\PYG{p}{)}
\end{sphinxVerbatim}

Given the slow nature of ptychography experiments, it may be necessary
to capture an XAS scan into multiple chunks. Passing \sphinxcode{append=True}
to the importer allows \sphinxstylestrong{datasets to be combined}:

\begin{sphinxVerbatim}[commandchars=\\\{\}]
\PYG{k+kn}{import} \PYG{n+nn}{xanespy} \PYG{k}{as} \PYG{n+nn}{xp}

\PYG{c+c1}{\PYGZsh{} The first data\PYGZhy{}set is imported like normal except that the}
\PYG{c+c1}{\PYGZsh{} groupname and filename to save under are explicit.}
\PYG{n}{xp}\PYG{o}{.}\PYG{n}{import\PYGZus{}nanosurveyor\PYGZus{}frameset}\PYG{p}{(}\PYG{l+s+s1}{\PYGZsq{}}\PYG{l+s+s1}{NS\PYGZus{}160529047/}\PYG{l+s+s1}{\PYGZsq{}}\PYG{p}{,}
                                \PYG{n}{hdf\PYGZus{}filename}\PYG{o}{=}\PYG{l+s+s1}{\PYGZsq{}}\PYG{l+s+s1}{my\PYGZus{}ptycho\PYGZus{}data.h5}\PYG{l+s+s1}{\PYGZsq{}}\PYG{p}{,}
                                \PYG{n}{hdf\PYGZus{}groupname}\PYG{o}{=}\PYG{l+s+s1}{\PYGZsq{}}\PYG{l+s+s1}{my\PYGZus{}combined\PYGZus{}experiment}\PYG{l+s+s1}{\PYGZsq{}}\PYG{p}{)}

\PYG{c+c1}{\PYGZsh{} Now subsequent scans get the {}`{}`append=True{}`{}` argument}
\PYG{n}{xp}\PYG{o}{.}\PYG{n}{import\PYGZus{}nanosurveyor\PYGZus{}frameset}\PYG{p}{(}\PYG{l+s+s1}{\PYGZsq{}}\PYG{l+s+s1}{NS\PYGZus{}160529048/}\PYG{l+s+s1}{\PYGZsq{}}\PYG{p}{,}
                                \PYG{n}{hdf\PYGZus{}filename}\PYG{o}{=}\PYG{l+s+s1}{\PYGZsq{}}\PYG{l+s+s1}{my\PYGZus{}ptycho\PYGZus{}data.h5}\PYG{l+s+s1}{\PYGZsq{}}\PYG{p}{,}
                                \PYG{n}{hdf\PYGZus{}groupname}\PYG{o}{=}\PYG{l+s+s1}{\PYGZsq{}}\PYG{l+s+s1}{my\PYGZus{}combined\PYGZus{}experiment}\PYG{l+s+s1}{\PYGZsq{}}\PYG{p}{,}
                                \PYG{n}{append}\PYG{o}{=}\PYG{k+kc}{True}\PYG{p}{)}
\PYG{n}{xp}\PYG{o}{.}\PYG{n}{import\PYGZus{}nanosurveyor\PYGZus{}frameset}\PYG{p}{(}\PYG{l+s+s1}{\PYGZsq{}}\PYG{l+s+s1}{NS\PYGZus{}160529049/}\PYG{l+s+s1}{\PYGZsq{}}\PYG{p}{,}
                                \PYG{n}{hdf\PYGZus{}filename}\PYG{o}{=}\PYG{l+s+s1}{\PYGZsq{}}\PYG{l+s+s1}{my\PYGZus{}ptycho\PYGZus{}data.h5}\PYG{l+s+s1}{\PYGZsq{}}\PYG{p}{,}
                                \PYG{n}{hdf\PYGZus{}groupname}\PYG{o}{=}\PYG{l+s+s1}{\PYGZsq{}}\PYG{l+s+s1}{my\PYGZus{}combined\PYGZus{}experiment}\PYG{l+s+s1}{\PYGZsq{}}\PYG{p}{,}
                                \PYG{n}{append}\PYG{o}{=}\PYG{k+kc}{True}\PYG{p}{)}
\end{sphinxVerbatim}

It may be necessary to only import a subset of the frames collected in
a given directory. For example, if the last frame drifted out of the
field-of-view and was re-collected in the next set of energies. The
arguments \sphinxcode{energy\_range} and \sphinxcode{exclude\_re} can be used to fine-tune
the list of importable files. See the documentation for
{\hyperref[\detokenize{xanespy:xanespy.importers.import_nanosurveyor_frameset}]{\sphinxcrossref{\sphinxcode{import\_nanosurveyor\_frameset()}}}} for more
details.


\section{Xradia Image Files (.xrm and .txrm)}
\label{\detokenize{importing:xradia-image-files-xrm-and-txrm}}
Xradia microscopes use the Microsoft OLE container format, which is
not easily read %
\begin{footnote}[1]\sphinxAtStartFootnote
If you’re shopping for a container format for your new data
storage project, I would recommend AGAINST Microsoft
OLE. This format stores data in raw binary, meaning that you
need to know the encoding and structure to get meaningful
data out. Instead, try \sphinxstylestrong{HDF5}: a nice open-source, well
documented, type-aware format with bindings in many
languages. It even plays nicely with numpy out of the box.
%
\end{footnote}. Individual scan files are generally not that
helpful anyway. But in case you need it, there are some adapters to
\sphinxcode{.xrm} and \sphinxcode{.txrm} files, namely
{\hyperref[\detokenize{xanespy:xanespy.xradia.XRMFile}]{\sphinxcrossref{\sphinxcode{xanespy.xradia.XRMFile}}}} and
{\hyperref[\detokenize{xanespy:xanespy.xradia.TXRMFile}]{\sphinxcrossref{\sphinxcode{xanespy.xradia.TXRMFile}}}}.

\begin{sphinxadmonition}{note}{Note:}
The specification for \sphinxcode{.xrm} files is not public, so these
classes are reverse-engineered and may not be (definitely aren’t)
perfect. If you encounter problems, please \sphinxhref{https://github.com/m3wolf/xanespy/issues}{submit an issue}.
\end{sphinxadmonition}

Opening xrm or txrm files is best done via the context manager:

\begin{sphinxVerbatim}[commandchars=\\\{\}]
\PYG{k+kn}{import} \PYG{n+nn}{xanespy} \PYG{k}{as} \PYG{n+nn}{xp}
\PYG{k+kn}{import} \PYG{n+nn}{numpy} \PYG{k}{as} \PYG{n+nn}{np}

\PYG{c+c1}{\PYGZsh{} Single\PYGZhy{}image xrm file}
\PYG{k}{with} \PYG{n}{xp}\PYG{o}{.}\PYG{n}{XRMFile}\PYG{p}{(}\PYG{l+s+s1}{\PYGZsq{}}\PYG{l+s+s1}{my\PYGZus{}txm\PYGZus{}image.xrm}\PYG{l+s+s1}{\PYGZsq{}}\PYG{p}{)} \PYG{k}{as} \PYG{n}{f}\PYG{p}{:}
    \PYG{n}{img} \PYG{o}{=} \PYG{n}{f}\PYG{o}{.}\PYG{n}{image\PYGZus{}data}\PYG{p}{(}\PYG{p}{)}
    \PYG{k}{assert} \PYG{n}{img}\PYG{o}{.}\PYG{n}{ndim} \PYG{o}{==} \PYG{l+m+mi}{2} \PYG{c+c1}{\PYGZsh{} (row, col)}

\PYG{c+c1}{\PYGZsh{} Multi\PYGZhy{}image txrm energy stack file}
\PYG{k}{with} \PYG{n}{xp}\PYG{o}{.}\PYG{n}{TXRMFile}\PYG{p}{(}\PYG{l+s+s1}{\PYGZsq{}}\PYG{l+s+s1}{my\PYGZus{}txm\PYGZus{}stack.txrm}\PYG{l+s+s1}{\PYGZsq{}}\PYG{p}{)} \PYG{k}{as} \PYG{n}{f}\PYG{p}{:}
    \PYG{c+c1}{\PYGZsh{} Get images one at a time by index}
    \PYG{n}{img} \PYG{o}{=} \PYG{n}{f}\PYG{o}{.}\PYG{n}{image\PYGZus{}data}\PYG{p}{(}\PYG{n}{idx}\PYG{o}{=}\PYG{l+m+mi}{0}\PYG{p}{)}
    \PYG{k}{assert} \PYG{n}{img}\PYG{o}{.}\PYG{n}{ndim} \PYG{o}{==} \PYG{l+m+mi}{2} \PYG{c+c1}{\PYGZsh{} (row, col)}

    \PYG{c+c1}{\PYGZsh{} Get images all at once in one big array}
    \PYG{n}{stack} \PYG{o}{=} \PYG{n}{f}\PYG{o}{.}\PYG{n}{image\PYGZus{}stack}\PYG{p}{(}\PYG{p}{)}
    \PYG{k}{assert} \PYG{n}{stack}\PYG{o}{.}\PYG{n}{ndim} \PYG{o}{==} \PYG{l+m+mi}{3} \PYG{c+c1}{\PYGZsh{} (prj, row, col)}
    \PYG{k}{assert} \PYG{n}{np}\PYG{o}{.}\PYG{n}{array\PYGZus{}equal}\PYG{p}{(}\PYG{n}{img}\PYG{p}{,} \PYG{n}{stack}\PYG{p}{[}\PYG{l+m+mi}{0}\PYG{p}{]}\PYG{p}{)}

    \PYG{c+c1}{\PYGZsh{} Get X\PYGZhy{}ray energies for the images}
    \PYG{n}{energies} \PYG{o}{=} \PYG{n}{f}\PYG{o}{.}\PYG{n}{energies}\PYG{p}{(}\PYG{p}{)}
    \PYG{k}{assert} \PYG{n+nb}{len}\PYG{p}{(}\PYG{n}{energies}\PYG{p}{)} \PYG{o}{==} \PYG{n}{stack}\PYG{o}{.}\PYG{n}{shape}\PYG{p}{[}\PYG{l+m+mi}{0}\PYG{p}{]}
\end{sphinxVerbatim}

The {\hyperref[\detokenize{xanespy:xanespy.xradia.XRMFile}]{\sphinxcrossref{\sphinxcode{XRMFile}}}} and
{\hyperref[\detokenize{xanespy:xanespy.xradia.TXRMFile}]{\sphinxcrossref{\sphinxcode{TXRMFile}}}} classes accept an optional
\sphinxcode{flavor} keyword argument. This option affects several pieces of
metadata. See the {\hyperref[\detokenize{xanespy:xanespy.xradia.XRMFile}]{\sphinxcrossref{\sphinxcode{XRMFile}}}} documentation
for details.


\chapter{Analyzing the Data}
\label{\detokenize{analysis::doc}}\label{\detokenize{analysis:analyzing-the-data}}

\section{Forking the Data}
\label{\detokenize{analysis:forking-the-data}}
After long computations it can be helpful to create a copy of the
dataset as a sort of checkpoint. To enable this, xanespy includes the
\sphinxcode{fork\_data\_group()} method: it creates a copy of the HDF group. The
active set can be changed by setting the \sphinxcode{data\_name} attribute on a
\sphinxcode{XanesFrameset} object. Most operations enabled by the
\sphinxcode{XanesFrameset()} class are not idempotent, so starting from a clean
dataset may be necessary:

\begin{sphinxVerbatim}[commandchars=\\\{\}]
\PYG{k+kn}{import} \PYG{n+nn}{xanespy}
\PYG{c+c1}{\PYGZsh{} Select an imported hdf file to use}
\PYG{n}{frameset} \PYG{o}{=} \PYG{n}{xanespy}\PYG{o}{.}\PYG{n}{XanesFrameset}\PYG{p}{(}\PYG{n}{hdf\PYGZus{}filename}\PYG{o}{=}\PYG{l+s+s2}{\PYGZdq{}}\PYG{l+s+s2}{...}\PYG{l+s+s2}{\PYGZdq{}}\PYG{p}{)}
\PYG{c+c1}{\PYGZsh{} Make sure we\PYGZsq{}re starting with clean data}
\PYG{n}{frameset}\PYG{o}{.}\PYG{n}{data\PYGZus{}name} \PYG{o}{=} \PYG{l+s+s2}{\PYGZdq{}}\PYG{l+s+s2}{imported}\PYG{l+s+s2}{\PYGZdq{}}

\PYG{c+c1}{\PYGZsh{} Create a copy as a checkpoint}
\PYG{n}{frameset}\PYG{o}{.}\PYG{n}{fork\PYGZus{}data\PYGZus{}group}\PYG{p}{(}\PYG{l+s+s2}{\PYGZdq{}}\PYG{l+s+s2}{aligned}\PYG{l+s+s2}{\PYGZdq{}}\PYG{p}{)}
\PYG{c+c1}{\PYGZsh{} Do some work that we\PYGZsq{}re not sure will succeed}
\PYG{n}{frameset}\PYG{o}{.}\PYG{n}{align\PYGZus{}frames}\PYG{p}{(}\PYG{n}{passes}\PYG{o}{=}\PYG{l+m+mi}{5}\PYG{p}{)}

\PYG{c+c1}{\PYGZsh{} If the alignment doesn\PYGZsq{}t work right,}
\PYG{c+c1}{\PYGZsh{} we can switch back to the original data and try again}
\PYG{n}{frameset}\PYG{o}{.}\PYG{n}{data\PYGZus{}name} \PYG{o}{=} \PYG{l+s+s2}{\PYGZdq{}}\PYG{l+s+s2}{imported}\PYG{l+s+s2}{\PYGZdq{}}
\PYG{n}{frameset}\PYG{o}{.}\PYG{n}{fork\PYGZus{}data\PYGZus{}group}\PYG{p}{(}\PYG{l+s+s2}{\PYGZdq{}}\PYG{l+s+s2}{aligned}\PYG{l+s+s2}{\PYGZdq{}}\PYG{p}{)}
\PYG{n}{frameset}\PYG{o}{.}\PYG{n}{align\PYGZus{}frames}\PYG{p}{(}\PYG{n}{passes}\PYG{o}{=}\PYG{l+m+mi}{3}\PYG{p}{)}
\end{sphinxVerbatim}

The {\hyperref[\detokenize{xanespy:xanespy.xanes_frameset.XanesFrameset.fork_data_group}]{\sphinxcrossref{\sphinxcode{fork\_data\_group()}}}}
method can be slow for large datasets. Xanespy will raise exceptions
for non-sensical requests for forking: trying to copy a group onto
itself, using a datagroup that doesn’t exist, etc.


\section{Frame Alignment}
\label{\detokenize{analysis:frame-alignment}}
In order to acquire reliable spectra, \sphinxstylestrong{it is important that the
frames be aligned properly}. Thermal expansion, motor slop, sample
damage and imperfect microscope alignment can all cause frames to be
misaligned. \sphinxstylestrong{It is often necessary to align the frames before
performing any of the subsequent steps.}

This is done with the \sphinxcode{xanespy.XanesFrameset().align\_frames()} method:

\begin{sphinxVerbatim}[commandchars=\\\{\}]
\PYG{k+kn}{import} \PYG{n+nn}{xanespy}
\PYG{c+c1}{\PYGZsh{} Select an imported hdf file to use}
\PYG{n}{frameset} \PYG{o}{=} \PYG{n}{xanespy}\PYG{o}{.}\PYG{n}{XanesFrameset}\PYG{p}{(}\PYG{n}{hdf\PYGZus{}filename}\PYG{o}{=}\PYG{l+s+s2}{\PYGZdq{}}\PYG{l+s+s2}{...}\PYG{l+s+s2}{\PYGZdq{}}\PYG{p}{)}

\PYG{c+c1}{\PYGZsh{} Run through five passes of the default phase correlation}
\PYG{n}{frameset}\PYG{o}{.}\PYG{n}{align\PYGZus{}frames}\PYG{p}{(}\PYG{n}{passes}\PYG{o}{=}\PYG{l+m+mi}{5}\PYG{p}{,} \PYG{n}{plot\PYGZus{}results}\PYG{o}{=}\PYG{k+kc}{True}\PYG{p}{)}
\end{sphinxVerbatim}
\begin{figure}[htbp]
\centering
\capstart

\noindent\sphinxincludegraphics{{alignment-boxplot}.svg}
\caption{With the \sphinxcode{plot\_results} argument, a box and whisker plot is
generated showing the distribution of corrections needed for
aligning each frame. Several passes help reduce the error.}\label{\detokenize{analysis:id1}}\end{figure}

The alignments are generally done with subpixel resolution, which
gives improved accuracy, but requires interpolation. To avoid problems
with accumulated error, a cumulative translation matrix is kept and
applied at the end to the original data. You can add your own
translation manually using the \sphinxcode{stage\_transformations()} method. If
\sphinxcode{align\_frames()} is called with \sphinxcode{commit=False}, then the alignment
parameters are added to \sphinxcode{stage\_transformations} but not
applied. Once all transformations are staged, the
\sphinxcode{apply\_transformations()} method will apply the cumulative
transformation matrix and (by default) save the result to disk.

If the starting alignment is particularly sporadic, a false minimum
can result in an exception or a very small image that doesn’t provide
useful information. In these cases, it may be necessary to first stage
a template registration then perform several passes of phase
correlation:

\begin{sphinxVerbatim}[commandchars=\\\{\}]
\PYG{n}{fs} \PYG{o}{=} \PYG{n}{XanesFrameset}\PYG{p}{(}\PYG{n}{hdf\PYGZus{}filename}\PYG{o}{=}\PYG{l+s+s2}{\PYGZdq{}}\PYG{l+s+s2}{...}\PYG{l+s+s2}{\PYGZdq{}}\PYG{p}{)}
\PYG{c+c1}{\PYGZsh{} Eg. use the 22nd energy and a range of the image as the template}
\PYG{n}{template} \PYG{o}{=} \PYG{n}{fs}\PYG{o}{.}\PYG{n}{frames}\PYG{p}{(}\PYG{p}{)}\PYG{p}{[}\PYG{l+m+mi}{21}\PYG{p}{,} \PYG{l+m+mi}{110}\PYG{p}{:}\PYG{l+m+mi}{425}\PYG{p}{,} \PYG{l+m+mi}{150}\PYG{p}{:}\PYG{l+m+mi}{450}\PYG{p}{]}
\PYG{n}{plt}\PYG{o}{.}\PYG{n}{imshow}\PYG{p}{(}\PYG{n}{template}\PYG{p}{,} \PYG{n}{cmap}\PYG{o}{=}\PYG{l+s+s2}{\PYGZdq{}}\PYG{l+s+s2}{gray}\PYG{l+s+s2}{\PYGZdq{}}\PYG{p}{)}

\PYG{n}{fs}\PYG{o}{.}\PYG{n}{fork\PYGZus{}data\PYGZus{}group}\PYG{p}{(}\PYG{l+s+s1}{\PYGZsq{}}\PYG{l+s+s1}{aligned}\PYG{l+s+s1}{\PYGZsq{}}\PYG{p}{)}

\PYG{n}{fs}\PYG{o}{.}\PYG{n}{align\PYGZus{}frames}\PYG{p}{(}\PYG{n}{method}\PYG{o}{=}\PYG{l+s+s2}{\PYGZdq{}}\PYG{l+s+s2}{template\PYGZus{}match}\PYG{l+s+s2}{\PYGZdq{}}\PYG{p}{,} \PYG{n}{template}\PYG{o}{=}\PYG{n}{template}\PYG{p}{,} \PYG{n}{commit}\PYG{o}{=}\PYG{k+kc}{False}\PYG{p}{)}
\PYG{n}{fs}\PYG{o}{.}\PYG{n}{align\PYGZus{}frames}\PYG{p}{(}\PYG{n}{passes}\PYG{o}{=}\PYG{l+m+mi}{5}\PYG{p}{,} \PYG{n}{commit}\PYG{o}{=}\PYG{k+kc}{True}\PYG{p}{)}
\end{sphinxVerbatim}


\section{Subtracting Surroundings}
\label{\detokenize{analysis:subtracting-surroundings}}
Some microscopes show differences in the absorbance of the whole
frame, including background material. This can be removed from each
frame, giving a better spectrum:

\begin{sphinxVerbatim}[commandchars=\\\{\}]
\PYG{n}{fs} \PYG{o}{=} \PYG{n}{XanesFrameset}\PYG{p}{(}\PYG{n}{hdf\PYGZus{}filename}\PYG{o}{=}\PYG{l+s+s2}{\PYGZdq{}}\PYG{l+s+s2}{...}\PYG{l+s+s2}{\PYGZdq{}}\PYG{p}{)}
\PYG{n}{fs}\PYG{o}{.}\PYG{n}{subtract\PYGZus{}surroundings}\PYG{p}{(}\PYG{p}{)}
\end{sphinxVerbatim}
\begin{figure}[htbp]
\centering
\capstart

\noindent\sphinxincludegraphics{{subtract-surroundings}.svg}
\caption{The effect of the \sphinxcode{subtract\_surroundings()} method.}\label{\detokenize{analysis:id2}}\end{figure}


\section{Spectrum Fitting - K-Edge}
\label{\detokenize{analysis:spectrum-fitting-k-edge}}

\section{Spectrum Fitting - L-Edge}
\label{\detokenize{analysis:spectrum-fitting-l-edge}}

\chapter{Visualization of Results}
\label{\detokenize{visualization::doc}}\label{\detokenize{visualization:visualization-of-results}}

\section{Plotting Maps}
\label{\detokenize{visualization:plotting-maps}}
After calculating the maps, there will be a number of options for
plotting. The {\hyperref[\detokenize{xanespy:xanespy.xanes_frameset.XanesFrameset}]{\sphinxcrossref{\sphinxcode{xanespy.xanes\_frameset.XanesFrameset}}}} object
has a number of methods for plotting the maps. Many of these method
use the \sphinxcode{map\_name} argument to The
{\hyperref[\detokenize{xanespy:xanespy.xanes_frameset.XanesFrameset.plot_map}]{\sphinxcrossref{\sphinxcode{xanespy.xanes\_frameset.XanesFrameset.plot\_map()}}}} method will
prepare a plot of any of the maps.


\section{Interactive (Qt) Viewer}
\label{\detokenize{visualization:interactive-qt-viewer}}
Xanespy includes a graphical user interface that allows for
interactive visualization of X-ray frames and maps. To launch the
viewer, prepare a \sphinxcode{XanesFrameset} object and run the \sphinxcode{qt\_viewer()}
method:

\begin{sphinxVerbatim}[commandchars=\\\{\}]
\PYG{n}{fs} \PYG{o}{=} \PYG{n}{XanesFrameset}\PYG{p}{(}\PYG{o}{.}\PYG{o}{.}\PYG{o}{.}\PYG{p}{)}
\PYG{n}{fs}\PYG{o}{.}\PYG{n}{qt\PYGZus{}viewer}\PYG{p}{(}\PYG{p}{)}
\end{sphinxVerbatim}

The data tree on the left of the window shows the possible datasets
than be viewed. Choosing an entry with type “frameset” will load and
plot the frames, spectra and histograms in the frame window. If a
“map” entry is selected, the map window will be launchedx and the
frames that went into making the map will be shown in the frame
window.

In the interest of encouraging reproducibility, the \sphinxstylestrong{ability to
export plots has been intentionally left out}. Any options selected
in the GUI can be passed into the \sphinxcode{plot\_map}, \sphinxcode{plot\_histogram} or
\sphinxcode{plot\_spectrum} methods of the frameset object. The name of the
entry in the data tree is given as the keyword argument
\sphinxcode{representation}.

\begin{sphinxadmonition}{warning}{Warning:}
The interactive viewer has no effect on the underlying
frameset or accompanying data, with one exception:
changing to a different top-level branch in the data tree
will change the \sphinxcode{XanesFrameset.data\_name}
attribute. Before proceeding with analysis, either revert
the \sphinxcode{data\_name} attribute manually or create a new
\sphinxcode{XanesFrameset} object.
\end{sphinxadmonition}
\begin{figure}[htbp]
\centering

\noindent\sphinxincludegraphics{{qt_viewer_screenshot-1}.png}
\end{figure}
\begin{figure}[htbp]
\centering
\capstart

\noindent\sphinxincludegraphics{{qt_viewer_screenshot-2}.png}
\caption{Screenshots of the Qt viewer. Frame window (top) and map window
(bottom).}\label{\detokenize{visualization:id1}}\end{figure}


\chapter{Accessing the Data Directly}
\label{\detokenize{data_stores::doc}}\label{\detokenize{data_stores:accessing-the-data-directly}}
While the {\hyperref[\detokenize{xanespy:xanespy.xanes_frameset.XanesFrameset}]{\sphinxcrossref{\sphinxcode{XanesFrameset}}}} class has
methods for common tasks, sometimes it is necessary to access the data
directly, as either numpy arrays or h5py datasets. The xanes\_frameset
has a {\hyperref[\detokenize{xanespy:xanespy.xanes_frameset.XanesFrameset.store}]{\sphinxcrossref{\sphinxcode{store()}}}} method
that returns an interface (\sphinxcode{TXMStore}) to the underlying HDF5 file.

\begin{sphinxadmonition}{warning}{Warning:}
The \sphinxcode{TXMStore} created by \sphinxcode{xanes\_frameset.store()} is
attached to an open HDF5 file. It is strongly recommended to use
the \sphinxcode{with} statement described below. Otherwise make sure to call
the store’s \sphinxcode{close()} method in a \sphinxcode{try...except} block. File
corruption is likely if not opened in this manner.
\end{sphinxadmonition}

Call the following to get access to the associated datasets. Properties of the interface will
return an HDF5 dataset in most cases.:

\begin{sphinxVerbatim}[commandchars=\\\{\}]
\PYG{k+kn}{import} \PYG{n+nn}{xanespy} \PYG{k}{as} \PYG{n+nn}{xp}
\PYG{n}{frameset} \PYG{o}{=} \PYG{n}{xp}\PYG{o}{.}\PYG{n}{XanesFrameset}\PYG{p}{(}\PYG{o}{.}\PYG{o}{.}\PYG{o}{.}\PYG{p}{)}

\PYG{c+c1}{\PYGZsh{} Open the TXMStore interface}
\PYG{k}{with} \PYG{n}{frameset}\PYG{o}{.}\PYG{n}{store}\PYG{p}{(}\PYG{p}{)} \PYG{k}{as} \PYG{n}{store}\PYG{p}{:}
    \PYG{c+c1}{\PYGZsh{} For example, the images are in (timestep, energy, row, column) order}
    \PYG{k}{assert} \PYG{n}{store}\PYG{o}{.}\PYG{n}{absorbances}\PYG{o}{.}\PYG{n}{shape} \PYG{o}{==} \PYG{p}{(}\PYG{l+m+mi}{10}\PYG{p}{,} \PYG{l+m+mi}{62}\PYG{p}{,} \PYG{l+m+mi}{1024}\PYG{p}{,} \PYG{l+m+mi}{1024}\PYG{p}{)}
    \PYG{c+c1}{\PYGZsh{} Energies are in (timestep, energy) order}
    \PYG{k}{assert} \PYG{n}{store}\PYG{o}{.}\PYG{n}{energies}\PYG{o}{.}\PYG{n}{shape} \PYG{o}{==} \PYG{p}{(}\PYG{l+m+mi}{10}\PYG{p}{,} \PYG{l+m+mi}{62}\PYG{p}{)}
\end{sphinxVerbatim}


\chapter{xanespy}
\label{\detokenize{modules::doc}}\label{\detokenize{modules:xanespy}}

\section{xanespy package}
\label{\detokenize{xanespy::doc}}\label{\detokenize{xanespy:xanespy-package}}

\subsection{Submodules}
\label{\detokenize{xanespy:submodules}}

\subsection{xanespy.beamlines module}
\label{\detokenize{xanespy:module-xanespy.beamlines}}\label{\detokenize{xanespy:xanespy-beamlines-module}}\index{xanespy.beamlines (module)}
Functions and classes that prepare experiments at specific
synchrotron beamlines.
\index{Detector (class in xanespy.beamlines)}

\begin{fulllineitems}
\phantomsection\label{\detokenize{xanespy:xanespy.beamlines.Detector}}\pysiglinewithargsret{\sphinxstrong{class }\sphinxcode{xanespy.beamlines.}\sphinxbfcode{Detector}}{\emph{start: xanespy.beamlines.ZoneplatePoint}, \emph{x\_step: int = None}, \emph{y\_step: int = None}, \emph{z\_step: int = None}, \emph{end: xanespy.beamlines.ZoneplatePoint = None}}{}
Bases: {\hyperref[\detokenize{xanespy:xanespy.beamlines.Zoneplate}]{\sphinxcrossref{\sphinxcode{xanespy.beamlines.Zoneplate}}}}

A calibration object for the position of the detector.

\end{fulllineitems}

\index{DetectorPoint (class in xanespy.beamlines)}

\begin{fulllineitems}
\phantomsection\label{\detokenize{xanespy:xanespy.beamlines.DetectorPoint}}\pysiglinewithargsret{\sphinxstrong{class }\sphinxcode{xanespy.beamlines.}\sphinxbfcode{DetectorPoint}}{\emph{x}, \emph{y}, \emph{z}, \emph{energy}}{}
Bases: \sphinxcode{tuple}
\index{energy (xanespy.beamlines.DetectorPoint attribute)}

\begin{fulllineitems}
\phantomsection\label{\detokenize{xanespy:xanespy.beamlines.DetectorPoint.energy}}\pysigline{\sphinxbfcode{energy}}
Alias for field number 3

\end{fulllineitems}

\index{x (xanespy.beamlines.DetectorPoint attribute)}

\begin{fulllineitems}
\phantomsection\label{\detokenize{xanespy:xanespy.beamlines.DetectorPoint.x}}\pysigline{\sphinxbfcode{x}}
Alias for field number 0

\end{fulllineitems}

\index{y (xanespy.beamlines.DetectorPoint attribute)}

\begin{fulllineitems}
\phantomsection\label{\detokenize{xanespy:xanespy.beamlines.DetectorPoint.y}}\pysigline{\sphinxbfcode{y}}
Alias for field number 1

\end{fulllineitems}

\index{z (xanespy.beamlines.DetectorPoint attribute)}

\begin{fulllineitems}
\phantomsection\label{\detokenize{xanespy:xanespy.beamlines.DetectorPoint.z}}\pysigline{\sphinxbfcode{z}}
Alias for field number 2

\end{fulllineitems}


\end{fulllineitems}

\index{Zoneplate (class in xanespy.beamlines)}

\begin{fulllineitems}
\phantomsection\label{\detokenize{xanespy:xanespy.beamlines.Zoneplate}}\pysiglinewithargsret{\sphinxstrong{class }\sphinxcode{xanespy.beamlines.}\sphinxbfcode{Zoneplate}}{\emph{start: xanespy.beamlines.ZoneplatePoint}, \emph{x\_step: int = None}, \emph{y\_step: int = None}, \emph{z\_step: int = None}, \emph{end: xanespy.beamlines.ZoneplatePoint = None}}{}
Bases: \sphinxcode{object}

Type of focusing optic using in X-ray microscopy. It must be moved
with changing energy to properly focus the beam. In order to
properly predict zoneplate positions, it needs either two
position-energy pairs or one position-energy pair and a
step. Passing two position-energy pairs is preffered because this
allows x, y and z to be set properly instead of just z.
\begin{quote}\begin{description}
\item[{Parameters}] \leavevmode\begin{itemize}
\item {} 
\sphinxstyleliteralstrong{start} (\sphinxstyleliteralemphasis{tuple}) \textendash{} The first zoneplate position-energy pair.

\item {} 
\sphinxstyleliteralstrong{z\_step} (\sphinxstyleliteralemphasis{int}\sphinxstyleliteralemphasis{, }\sphinxstyleliteralemphasis{optional}) \textendash{} Adjustment in z-position for every positive change of 1 eV
of beam energy.

\item {} 
\sphinxstyleliteralstrong{end} (\sphinxstyleliteralemphasis{tuple}\sphinxstyleliteralemphasis{, }\sphinxstyleliteralemphasis{optional}) \textendash{} The second zoneplate position-energy pair.

\end{itemize}

\end{description}\end{quote}
\index{position() (xanespy.beamlines.Zoneplate method)}

\begin{fulllineitems}
\phantomsection\label{\detokenize{xanespy:xanespy.beamlines.Zoneplate.position}}\pysiglinewithargsret{\sphinxbfcode{position}}{\emph{energy: float}}{}
Predict the x, y and z position of the zonplate for the given
energy.

\end{fulllineitems}


\end{fulllineitems}

\index{ZoneplatePoint (class in xanespy.beamlines)}

\begin{fulllineitems}
\phantomsection\label{\detokenize{xanespy:xanespy.beamlines.ZoneplatePoint}}\pysiglinewithargsret{\sphinxstrong{class }\sphinxcode{xanespy.beamlines.}\sphinxbfcode{ZoneplatePoint}}{\emph{x}, \emph{y}, \emph{z}, \emph{energy}}{}
Bases: \sphinxcode{tuple}
\index{energy (xanespy.beamlines.ZoneplatePoint attribute)}

\begin{fulllineitems}
\phantomsection\label{\detokenize{xanespy:xanespy.beamlines.ZoneplatePoint.energy}}\pysigline{\sphinxbfcode{energy}}
Alias for field number 3

\end{fulllineitems}

\index{x (xanespy.beamlines.ZoneplatePoint attribute)}

\begin{fulllineitems}
\phantomsection\label{\detokenize{xanespy:xanespy.beamlines.ZoneplatePoint.x}}\pysigline{\sphinxbfcode{x}}
Alias for field number 0

\end{fulllineitems}

\index{y (xanespy.beamlines.ZoneplatePoint attribute)}

\begin{fulllineitems}
\phantomsection\label{\detokenize{xanespy:xanespy.beamlines.ZoneplatePoint.y}}\pysigline{\sphinxbfcode{y}}
Alias for field number 1

\end{fulllineitems}

\index{z (xanespy.beamlines.ZoneplatePoint attribute)}

\begin{fulllineitems}
\phantomsection\label{\detokenize{xanespy:xanespy.beamlines.ZoneplatePoint.z}}\pysigline{\sphinxbfcode{z}}
Alias for field number 2

\end{fulllineitems}


\end{fulllineitems}

\index{monitor\_sector8() (in module xanespy.beamlines)}

\begin{fulllineitems}
\phantomsection\label{\detokenize{xanespy:xanespy.beamlines.monitor_sector8}}\pysiglinewithargsret{\sphinxcode{xanespy.beamlines.}\sphinxbfcode{monitor\_sector8}}{\emph{tsv\_filename}}{}
Monitors a list of files and displays them as they are collected
from the instrument. A matplotlib axes is displayed and is updated
with each new frame that is detected. This function will block
until all files in the file list are accounted for.
\begin{quote}\begin{description}
\item[{Parameters}] \leavevmode
\sphinxstyleliteralstrong{tsv\_filename} (\sphinxstyleliteralemphasis{string}) \textendash{} The name of the file that contains a tab-separated-values list
of filenames. This file is automatically generated by the
\sphinxcode{sector8\_xanes\_script} function.

\end{description}\end{quote}

\end{fulllineitems}

\index{sector8\_xanes\_script() (in module xanespy.beamlines)}

\begin{fulllineitems}
\phantomsection\label{\detokenize{xanespy:xanespy.beamlines.sector8_xanes_script}}\pysiglinewithargsret{\sphinxcode{xanespy.beamlines.}\sphinxbfcode{sector8\_xanes\_script}}{\emph{dest}, \emph{edge}, \emph{zoneplate}, \emph{detector}, \emph{sample\_positions}, \emph{names}, \emph{iterations=range(0}, \emph{1)}, \emph{binning=1}, \emph{exposure=30}, \emph{abba\_mode=True}}{}
Prepare an script file for running multiple consecutive XANES
framesets on the transmission x-ray micrscope at the Advanced
Photon Source beamline 8-BM-B. This function also creates a
tab-separated-values (tsv) file which contains each sample
filename and its associated meta-data. This can then be used for
real-time processing.
\begin{quote}\begin{description}
\item[{Parameters}] \leavevmode\begin{itemize}
\item {} 
\sphinxstyleliteralstrong{dest} \textendash{} file-like object that will hold the resulting script

\item {} 
\sphinxstyleliteralstrong{zoneplate} ({\hyperref[\detokenize{xanespy:xanespy.beamlines.Zoneplate}]{\sphinxcrossref{\sphinxstyleliteralemphasis{Zoneplate}}}}) \textendash{} Calibration details for the Fresnel zone-plate.

\item {} 
\sphinxstyleliteralstrong{detector} ({\hyperref[\detokenize{xanespy:xanespy.beamlines.Detector}]{\sphinxcrossref{\sphinxstyleliteralemphasis{Detector}}}}) \textendash{} Like zoneplate, but for detector.

\item {} 
\sphinxstyleliteralstrong{edge} ({\hyperref[\detokenize{xanespy:xanespy.edges.KEdge}]{\sphinxcrossref{\sphinxstyleliteralemphasis{KEdge}}}}) \textendash{} Description of the absorption edge.

\item {} 
\sphinxstyleliteralstrong{binning} (\sphinxstyleliteralemphasis{int}\sphinxstyleliteralemphasis{, }\sphinxstyleliteralemphasis{optional}) \textendash{} how many CCD pixels to combine into one image pixel
(eg. 2 means 2x2 CCD pixels become 1 image pixel.

\item {} 
\sphinxstyleliteralstrong{exposure} (\sphinxstyleliteralemphasis{float}) \textendash{} How many seconds to collect for per frame

\item {} 
\sphinxstyleliteralstrong{sample\_positions} \textendash{} Locations to move the x, y (and z) axes to in order to capture
the image.

\item {} 
\sphinxstyleliteralstrong{names} \textendash{} sample name to use in file names. Should match \sphinxtitleref{sample\_positions} in length.

\item {} 
\sphinxstyleliteralstrong{iterations} (\sphinxstyleliteralemphasis{iterable}) \textendash{} contains an identifier for each full set of xanes location with
reference.

\item {} 
\sphinxstyleliteralstrong{abba\_mode} (\sphinxstyleliteralemphasis{str}\sphinxstyleliteralemphasis{, }\sphinxstyleliteralemphasis{optional}) \textendash{} If True, script will locations forward and backwards to save
time. Eg: reference, sample, change-energy, sample, reference,
change-energy, etc. Not compatible with \sphinxtitleref{frame\_rest}
argument. If used, the reference frame should be first or last
in the order to make the process maximally efficient.

\end{itemize}

\end{description}\end{quote}

\end{fulllineitems}

\index{ssrl6\_xanes\_script() (in module xanespy.beamlines)}

\begin{fulllineitems}
\phantomsection\label{\detokenize{xanespy:xanespy.beamlines.ssrl6_xanes_script}}\pysiglinewithargsret{\sphinxcode{xanespy.beamlines.}\sphinxbfcode{ssrl6\_xanes\_script}}{\emph{dest, edge: edges.KEdge, zoneplate: xanespy.beamlines.Zoneplate, positions: typing.List{[}utilities.position{]}, reference\_position: utilities.position, iterations: typing.Iterable, iteration\_rest: int = 0, frame\_rest: int = 0, binning: int = 2, exposure=0.5, repetitions: int = 5, ref\_repetitions: int = 10, abba\_mode: bool = True}}{}
Prepare a script file for running multiple consecutive XANES
framesets on the transmission x-ray micrscope at the Advanced
Photon Source beamline 8-BM-B. Both \sphinxtitleref{iteration\_rest} and
\sphinxtitleref{frame\_rest} can be used to give the material time to recover from
X-ray damage.
\begin{quote}\begin{description}
\item[{Parameters}] \leavevmode\begin{itemize}
\item {} 
\sphinxstyleliteralstrong{dest} \textendash{} A file-like object that will hold the resulting script

\item {} 
\sphinxstyleliteralstrong{edge} ({\hyperref[\detokenize{xanespy:xanespy.edges.Edge}]{\sphinxcrossref{\sphinxstyleliteralemphasis{Edge}}}}) \textendash{} Description of the absorption edge.

\item {} 
\sphinxstyleliteralstrong{binning} (\sphinxstyleliteralemphasis{int}\sphinxstyleliteralemphasis{, }\sphinxstyleliteralemphasis{optional}) \textendash{} how many CCD pixels to combine into one image pixel (eg. 2 means
2x2 CCD pixels become 1 image pixel.

\item {} 
\sphinxstyleliteralstrong{exposure} (\sphinxstyleliteralemphasis{float}\sphinxstyleliteralemphasis{, }\sphinxstyleliteralemphasis{optional}) \textendash{} How many seconds to collect for per frame

\item {} 
\sphinxstyleliteralstrong{positions} \textendash{} Locations to move the x, y (and z) axes to in order to capture
the image.

\item {} 
\sphinxstyleliteralstrong{reference\_position} (\sphinxstyleliteralemphasis{tuple}) \textendash{} Single x, y, z location to capture a reference frame.

\item {} 
\sphinxstyleliteralstrong{iteration\_rest} (\sphinxstyleliteralemphasis{int}\sphinxstyleliteralemphasis{, }\sphinxstyleliteralemphasis{optional}) \textendash{} Time (in seconds) to wait between iterations. Beam will wait at
reference location before starting next XANES set.

\item {} 
\sphinxstyleliteralstrong{frame\_rest} (\sphinxstyleliteralemphasis{int}\sphinxstyleliteralemphasis{, }\sphinxstyleliteralemphasis{optional}) \textendash{} Time (in seconds) to wait between frames. Beam will wait at
reference location before starting next energy frame.

\item {} 
\sphinxstyleliteralstrong{zoneplate} ({\hyperref[\detokenize{xanespy:xanespy.beamlines.Zoneplate}]{\sphinxcrossref{\sphinxstyleliteralemphasis{Zoneplate}}}}) \textendash{} Calibration details for the Fresnel zone-plate.

\item {} 
\sphinxstyleliteralstrong{detector} ({\hyperref[\detokenize{xanespy:xanespy.beamlines.Detector}]{\sphinxcrossref{\sphinxstyleliteralemphasis{Detector}}}}) \textendash{} Like zoneplate, but for detector.

\item {} 
\sphinxstyleliteralstrong{iterations} (\sphinxstyleliteralemphasis{Iterable}) \textendash{} Contains an identifier for each XANES dataset.

\item {} 
\sphinxstyleliteralstrong{repetitions} (\sphinxstyleliteralemphasis{int}\sphinxstyleliteralemphasis{, }\sphinxstyleliteralemphasis{optional}) \textendash{} How many images to collect for each location/energy. These
frames will then be averaged during analysis.

\item {} 
\sphinxstyleliteralstrong{ref\_repetitions} (\sphinxstyleliteralemphasis{int}\sphinxstyleliteralemphasis{, }\sphinxstyleliteralemphasis{optional}) \textendash{} Same as \sphinxtitleref{repetitions} but for reference frames.

\item {} 
\sphinxstyleliteralstrong{abba\_mode} (\sphinxstyleliteralemphasis{bool}\sphinxstyleliteralemphasis{, }\sphinxstyleliteralemphasis{optional}) \textendash{} If True, script will alternate sample and reference locations
first to save time. Eg: reference, sample, change-energy,
sample, reference, change-energy, etc. Not compatible with
\sphinxtitleref{frame\_rest} argument.

\end{itemize}

\end{description}\end{quote}

\end{fulllineitems}

\index{write\_scaninfo\_header() (in module xanespy.beamlines)}

\begin{fulllineitems}
\phantomsection\label{\detokenize{xanespy:xanespy.beamlines.write_scaninfo_header}}\pysiglinewithargsret{\sphinxcode{xanespy.beamlines.}\sphinxbfcode{write\_scaninfo\_header}}{\emph{f}, \emph{abba\_mode}, \emph{repetitions}, \emph{ref\_repetitions}}{}
\end{fulllineitems}



\subsection{xanespy.edges module}
\label{\detokenize{xanespy:xanespy-edges-module}}\label{\detokenize{xanespy:module-xanespy.edges}}\index{xanespy.edges (module)}
Descriptions of X-ray energy absorption edge.
\index{Edge (class in xanespy.edges)}

\begin{fulllineitems}
\phantomsection\label{\detokenize{xanespy:xanespy.edges.Edge}}\pysigline{\sphinxstrong{class }\sphinxcode{xanespy.edges.}\sphinxbfcode{Edge}}
Bases: \sphinxcode{object}

An X-ray absorption edge. It is defined by a series of energy
ranges. All energies are assumed to be in units of
electron-volts. This class is intended to be extended into K-edge,
L-edge, etc.
\index{E\_0 (xanespy.edges.Edge attribute)}

\begin{fulllineitems}
\phantomsection\label{\detokenize{xanespy:xanespy.edges.Edge.E_0}}\pysigline{\sphinxbfcode{E\_0}}
\sphinxstyleemphasis{float} \textendash{} The energy of the absorption edge itself.

\end{fulllineitems}

\index{regions (xanespy.edges.Edge attribute)}

\begin{fulllineitems}
\phantomsection\label{\detokenize{xanespy:xanespy.edges.Edge.regions}}\pysigline{\sphinxbfcode{regions}}
\sphinxstyleemphasis{list of 3-tuples} \textendash{} All the energy regions. Each tuple is of the form (start, end,
step) and is inclusive at both ends.

\end{fulllineitems}

\index{name (xanespy.edges.Edge attribute)}

\begin{fulllineitems}
\phantomsection\label{\detokenize{xanespy:xanespy.edges.Edge.name}}\pysigline{\sphinxbfcode{name}}
\sphinxstyleemphasis{string} \textendash{} A human-readable name for this edge (eg “Ni K-edge”)

\end{fulllineitems}

\index{pre\_edge (xanespy.edges.Edge attribute)}

\begin{fulllineitems}
\phantomsection\label{\detokenize{xanespy:xanespy.edges.Edge.pre_edge}}\pysigline{\sphinxbfcode{pre\_edge}}
\sphinxstyleemphasis{2-tuple} \textendash{} Energy range (start, stop) that defines points below the edge
region, inclusive.

\end{fulllineitems}

\index{post\_edge (xanespy.edges.Edge attribute)}

\begin{fulllineitems}
\phantomsection\label{\detokenize{xanespy:xanespy.edges.Edge.post_edge}}\pysigline{\sphinxbfcode{post\_edge}}
\sphinxstyleemphasis{2-tuple} \textendash{} Energy range (start, stop) that defines points above the edge
region, inclusive.

\end{fulllineitems}

\index{map\_range (xanespy.edges.Edge attribute)}

\begin{fulllineitems}
\phantomsection\label{\detokenize{xanespy:xanespy.edges.Edge.map_range}}\pysigline{\sphinxbfcode{map\_range}}
\sphinxstyleemphasis{2-tuple} \textendash{} Energy range (start, stop) used for normalizing maps. If not
supplied, will be determine from pre- and post-edge arguments.

\end{fulllineitems}

\index{edge\_range (xanespy.edges.Edge attribute)}

\begin{fulllineitems}
\phantomsection\label{\detokenize{xanespy:xanespy.edges.Edge.edge_range}}\pysigline{\sphinxbfcode{edge\_range}}
\sphinxstyleemphasis{2-tuple} \textendash{} Energy range (start, stop) used to determine the official
beginning and edge of the edge itself.

\end{fulllineitems}

\index{E\_0 (xanespy.edges.Edge attribute)}

\begin{fulllineitems}
\pysigline{\sphinxbfcode{E\_0}\sphinxstrong{ = None}}
\end{fulllineitems}

\index{all\_energies() (xanespy.edges.Edge method)}

\begin{fulllineitems}
\phantomsection\label{\detokenize{xanespy:xanespy.edges.Edge.all_energies}}\pysiglinewithargsret{\sphinxbfcode{all\_energies}}{}{}
Combine all the regions into one array.
\begin{quote}\begin{description}
\item[{Returns}] \leavevmode
\sphinxstylestrong{energies} \textendash{} Flat array with the energies for this edge.

\item[{Return type}] \leavevmode
list

\end{description}\end{quote}

\end{fulllineitems}

\index{edge\_range (xanespy.edges.Edge attribute)}

\begin{fulllineitems}
\pysigline{\sphinxbfcode{edge\_range}\sphinxstrong{ = None}}
\end{fulllineitems}

\index{energies\_in\_range() (xanespy.edges.Edge method)}

\begin{fulllineitems}
\phantomsection\label{\detokenize{xanespy:xanespy.edges.Edge.energies_in_range}}\pysiglinewithargsret{\sphinxbfcode{energies\_in\_range}}{\emph{norm\_range=None}}{}
\end{fulllineitems}

\index{map\_range (xanespy.edges.Edge attribute)}

\begin{fulllineitems}
\pysigline{\sphinxbfcode{map\_range}\sphinxstrong{ = None}}
\end{fulllineitems}

\index{post\_edge (xanespy.edges.Edge attribute)}

\begin{fulllineitems}
\pysigline{\sphinxbfcode{post\_edge}\sphinxstrong{ = None}}
\end{fulllineitems}

\index{pre\_edge (xanespy.edges.Edge attribute)}

\begin{fulllineitems}
\pysigline{\sphinxbfcode{pre\_edge}\sphinxstrong{ = None}}
\end{fulllineitems}

\index{pre\_edge\_fit (xanespy.edges.Edge attribute)}

\begin{fulllineitems}
\phantomsection\label{\detokenize{xanespy:xanespy.edges.Edge.pre_edge_fit}}\pysigline{\sphinxbfcode{pre\_edge\_fit}\sphinxstrong{ = None}}
\end{fulllineitems}

\index{regions (xanespy.edges.Edge attribute)}

\begin{fulllineitems}
\pysigline{\sphinxbfcode{regions}\sphinxstrong{ = {[}{]}}}
\end{fulllineitems}


\end{fulllineitems}

\index{KEdge (class in xanespy.edges)}

\begin{fulllineitems}
\phantomsection\label{\detokenize{xanespy:xanespy.edges.KEdge}}\pysigline{\sphinxstrong{class }\sphinxcode{xanespy.edges.}\sphinxbfcode{KEdge}}
Bases: {\hyperref[\detokenize{xanespy:xanespy.edges.Edge}]{\sphinxcrossref{\sphinxcode{xanespy.edges.Edge}}}}

An X-ray absorption K-edge corresponding to a 1s transition.
\index{annotate\_spectrum() (xanespy.edges.KEdge method)}

\begin{fulllineitems}
\phantomsection\label{\detokenize{xanespy:xanespy.edges.KEdge.annotate_spectrum}}\pysiglinewithargsret{\sphinxbfcode{annotate\_spectrum}}{\emph{ax}}{}
Draw lines on the axes to indicate the position of the edge.

\end{fulllineitems}

\index{mask() (xanespy.edges.KEdge method)}

\begin{fulllineitems}
\phantomsection\label{\detokenize{xanespy:xanespy.edges.KEdge.mask}}\pysiglinewithargsret{\sphinxbfcode{mask}}{\emph{*args}, \emph{**kwargs}}{}
Return a numpy array mask for material that’s active at this
edge. Calculations are done in \sphinxcode{xanes\_math.l\_edge\_mask()}.

\end{fulllineitems}


\end{fulllineitems}

\index{LEdge (class in xanespy.edges)}

\begin{fulllineitems}
\phantomsection\label{\detokenize{xanespy:xanespy.edges.LEdge}}\pysigline{\sphinxstrong{class }\sphinxcode{xanespy.edges.}\sphinxbfcode{LEdge}}
Bases: {\hyperref[\detokenize{xanespy:xanespy.edges.Edge}]{\sphinxcrossref{\sphinxcode{xanespy.edges.Edge}}}}

An X-ray absorption K-edge corresponding to a 2s or 2p
transition.
\index{annotate\_spectrum() (xanespy.edges.LEdge method)}

\begin{fulllineitems}
\phantomsection\label{\detokenize{xanespy:xanespy.edges.LEdge.annotate_spectrum}}\pysiglinewithargsret{\sphinxbfcode{annotate\_spectrum}}{\emph{ax}}{}
Draw lines on the axes to indicate the position of the edge.

\end{fulllineitems}

\index{mask() (xanespy.edges.LEdge method)}

\begin{fulllineitems}
\phantomsection\label{\detokenize{xanespy:xanespy.edges.LEdge.mask}}\pysiglinewithargsret{\sphinxbfcode{mask}}{\emph{*args}, \emph{**kwargs}}{}
Return a numpy array mask for material that’s active at this
edge. Calculations are done in \sphinxtitleref{xanes\_math.l\_edge\_mask()}.

\end{fulllineitems}


\end{fulllineitems}

\index{LMOMnKEdge (class in xanespy.edges)}

\begin{fulllineitems}
\phantomsection\label{\detokenize{xanespy:xanespy.edges.LMOMnKEdge}}\pysigline{\sphinxstrong{class }\sphinxcode{xanespy.edges.}\sphinxbfcode{LMOMnKEdge}}
Bases: {\hyperref[\detokenize{xanespy:xanespy.edges.KEdge}]{\sphinxcrossref{\sphinxcode{xanespy.edges.KEdge}}}}
\index{regions (xanespy.edges.LMOMnKEdge attribute)}

\begin{fulllineitems}
\phantomsection\label{\detokenize{xanespy:xanespy.edges.LMOMnKEdge.regions}}\pysigline{\sphinxbfcode{regions}\sphinxstrong{ = {[}(6450, 6510, 20), (6524, 6542, 2), (6544, 6564, 1), (6566, 6568, 2), (6572, 6600, 4), (6610, 6650, 10), (6700, 6850, 50){]}}}
\end{fulllineitems}


\end{fulllineitems}

\index{NCACobaltLEdge (class in xanespy.edges)}

\begin{fulllineitems}
\phantomsection\label{\detokenize{xanespy:xanespy.edges.NCACobaltLEdge}}\pysigline{\sphinxstrong{class }\sphinxcode{xanespy.edges.}\sphinxbfcode{NCACobaltLEdge}}
Bases: {\hyperref[\detokenize{xanespy:xanespy.edges.LEdge}]{\sphinxcrossref{\sphinxcode{xanespy.edges.LEdge}}}}
\index{E\_0 (xanespy.edges.NCACobaltLEdge attribute)}

\begin{fulllineitems}
\phantomsection\label{\detokenize{xanespy:xanespy.edges.NCACobaltLEdge.E_0}}\pysigline{\sphinxbfcode{E\_0}\sphinxstrong{ = 793.2}}
\end{fulllineitems}

\index{edge\_range (xanespy.edges.NCACobaltLEdge attribute)}

\begin{fulllineitems}
\phantomsection\label{\detokenize{xanespy:xanespy.edges.NCACobaltLEdge.edge_range}}\pysigline{\sphinxbfcode{edge\_range}\sphinxstrong{ = (775, 785)}}
\end{fulllineitems}

\index{map\_range (xanespy.edges.NCACobaltLEdge attribute)}

\begin{fulllineitems}
\phantomsection\label{\detokenize{xanespy:xanespy.edges.NCACobaltLEdge.map_range}}\pysigline{\sphinxbfcode{map\_range}\sphinxstrong{ = (0, 1)}}
\end{fulllineitems}

\index{post\_edge (xanespy.edges.NCACobaltLEdge attribute)}

\begin{fulllineitems}
\phantomsection\label{\detokenize{xanespy:xanespy.edges.NCACobaltLEdge.post_edge}}\pysigline{\sphinxbfcode{post\_edge}\sphinxstrong{ = (785, 790)}}
\end{fulllineitems}

\index{pre\_edge (xanespy.edges.NCACobaltLEdge attribute)}

\begin{fulllineitems}
\phantomsection\label{\detokenize{xanespy:xanespy.edges.NCACobaltLEdge.pre_edge}}\pysigline{\sphinxbfcode{pre\_edge}\sphinxstrong{ = (770, 775)}}
\end{fulllineitems}

\index{regions (xanespy.edges.NCACobaltLEdge attribute)}

\begin{fulllineitems}
\phantomsection\label{\detokenize{xanespy:xanespy.edges.NCACobaltLEdge.regions}}\pysigline{\sphinxbfcode{regions}\sphinxstrong{ = {[}(770, 775, 1), (775, 785, 0.5), (785, 790, 1){]}}}
\end{fulllineitems}


\end{fulllineitems}

\index{NCANickelKEdge (class in xanespy.edges)}

\begin{fulllineitems}
\phantomsection\label{\detokenize{xanespy:xanespy.edges.NCANickelKEdge}}\pysigline{\sphinxstrong{class }\sphinxcode{xanespy.edges.}\sphinxbfcode{NCANickelKEdge}}
Bases: {\hyperref[\detokenize{xanespy:xanespy.edges.KEdge}]{\sphinxcrossref{\sphinxcode{xanespy.edges.KEdge}}}}
\index{E\_0 (xanespy.edges.NCANickelKEdge attribute)}

\begin{fulllineitems}
\phantomsection\label{\detokenize{xanespy:xanespy.edges.NCANickelKEdge.E_0}}\pysigline{\sphinxbfcode{E\_0}\sphinxstrong{ = 8333}}
\end{fulllineitems}

\index{edge\_range (xanespy.edges.NCANickelKEdge attribute)}

\begin{fulllineitems}
\phantomsection\label{\detokenize{xanespy:xanespy.edges.NCANickelKEdge.edge_range}}\pysigline{\sphinxbfcode{edge\_range}\sphinxstrong{ = (8341, 8358)}}
\end{fulllineitems}

\index{map\_range (xanespy.edges.NCANickelKEdge attribute)}

\begin{fulllineitems}
\phantomsection\label{\detokenize{xanespy:xanespy.edges.NCANickelKEdge.map_range}}\pysigline{\sphinxbfcode{map\_range}\sphinxstrong{ = (8341, 8360)}}
\end{fulllineitems}

\index{post\_edge (xanespy.edges.NCANickelKEdge attribute)}

\begin{fulllineitems}
\phantomsection\label{\detokenize{xanespy:xanespy.edges.NCANickelKEdge.post_edge}}\pysigline{\sphinxbfcode{post\_edge}\sphinxstrong{ = (8400, 8640)}}
\end{fulllineitems}

\index{pre\_edge (xanespy.edges.NCANickelKEdge attribute)}

\begin{fulllineitems}
\phantomsection\label{\detokenize{xanespy:xanespy.edges.NCANickelKEdge.pre_edge}}\pysigline{\sphinxbfcode{pre\_edge}\sphinxstrong{ = (8249, 8320)}}
\end{fulllineitems}

\index{regions (xanespy.edges.NCANickelKEdge attribute)}

\begin{fulllineitems}
\phantomsection\label{\detokenize{xanespy:xanespy.edges.NCANickelKEdge.regions}}\pysigline{\sphinxbfcode{regions}\sphinxstrong{ = {[}(8250, 8310, 20), (8324, 8344, 2), (8344, 8356, 1), (8356, 8360, 2), (8360, 8400, 4), (8400, 8440, 8), (8440, 8640, 50){]}}}
\end{fulllineitems}

\index{shell (xanespy.edges.NCANickelKEdge attribute)}

\begin{fulllineitems}
\phantomsection\label{\detokenize{xanespy:xanespy.edges.NCANickelKEdge.shell}}\pysigline{\sphinxbfcode{shell}\sphinxstrong{ = ‘K’}}
\end{fulllineitems}


\end{fulllineitems}

\index{NCANickelKEdge61 (class in xanespy.edges)}

\begin{fulllineitems}
\phantomsection\label{\detokenize{xanespy:xanespy.edges.NCANickelKEdge61}}\pysigline{\sphinxstrong{class }\sphinxcode{xanespy.edges.}\sphinxbfcode{NCANickelKEdge61}}
Bases: {\hyperref[\detokenize{xanespy:xanespy.edges.NCANickelKEdge}]{\sphinxcrossref{\sphinxcode{xanespy.edges.NCANickelKEdge}}}}
\index{regions (xanespy.edges.NCANickelKEdge61 attribute)}

\begin{fulllineitems}
\phantomsection\label{\detokenize{xanespy:xanespy.edges.NCANickelKEdge61.regions}}\pysigline{\sphinxbfcode{regions}\sphinxstrong{ = {[}(8250, 8310, 15), (8324, 8360, 1), (8360, 8400, 4), (8400, 8440, 8), (8440, 8640, 50){]}}}
\end{fulllineitems}


\end{fulllineitems}

\index{NCANickelKEdge62 (class in xanespy.edges)}

\begin{fulllineitems}
\phantomsection\label{\detokenize{xanespy:xanespy.edges.NCANickelKEdge62}}\pysigline{\sphinxstrong{class }\sphinxcode{xanespy.edges.}\sphinxbfcode{NCANickelKEdge62}}
Bases: {\hyperref[\detokenize{xanespy:xanespy.edges.NCANickelKEdge}]{\sphinxcrossref{\sphinxcode{xanespy.edges.NCANickelKEdge}}}}
\index{regions (xanespy.edges.NCANickelKEdge62 attribute)}

\begin{fulllineitems}
\phantomsection\label{\detokenize{xanespy:xanespy.edges.NCANickelKEdge62.regions}}\pysigline{\sphinxbfcode{regions}\sphinxstrong{ = {[}(8250, 8310, 15), (8324, 8360, 1), (8360, 8400, 4), (8400, 8440, 8), (8440, 8690, 50){]}}}
\end{fulllineitems}


\end{fulllineitems}

\index{NCANickelLEdge (class in xanespy.edges)}

\begin{fulllineitems}
\phantomsection\label{\detokenize{xanespy:xanespy.edges.NCANickelLEdge}}\pysigline{\sphinxstrong{class }\sphinxcode{xanespy.edges.}\sphinxbfcode{NCANickelLEdge}}
Bases: {\hyperref[\detokenize{xanespy:xanespy.edges.LEdge}]{\sphinxcrossref{\sphinxcode{xanespy.edges.LEdge}}}}
\index{E\_0 (xanespy.edges.NCANickelLEdge attribute)}

\begin{fulllineitems}
\phantomsection\label{\detokenize{xanespy:xanespy.edges.NCANickelLEdge.E_0}}\pysigline{\sphinxbfcode{E\_0}\sphinxstrong{ = 853}}
\end{fulllineitems}

\index{edge\_range (xanespy.edges.NCANickelLEdge attribute)}

\begin{fulllineitems}
\phantomsection\label{\detokenize{xanespy:xanespy.edges.NCANickelLEdge.edge_range}}\pysigline{\sphinxbfcode{edge\_range}\sphinxstrong{ = (848, 857)}}
\end{fulllineitems}

\index{map\_range (xanespy.edges.NCANickelLEdge attribute)}

\begin{fulllineitems}
\phantomsection\label{\detokenize{xanespy:xanespy.edges.NCANickelLEdge.map_range}}\pysigline{\sphinxbfcode{map\_range}\sphinxstrong{ = (0, 1)}}
\end{fulllineitems}

\index{post\_edge (xanespy.edges.NCANickelLEdge attribute)}

\begin{fulllineitems}
\phantomsection\label{\detokenize{xanespy:xanespy.edges.NCANickelLEdge.post_edge}}\pysigline{\sphinxbfcode{post\_edge}\sphinxstrong{ = (857, 862)}}
\end{fulllineitems}

\index{pre\_edge (xanespy.edges.NCANickelLEdge attribute)}

\begin{fulllineitems}
\phantomsection\label{\detokenize{xanespy:xanespy.edges.NCANickelLEdge.pre_edge}}\pysigline{\sphinxbfcode{pre\_edge}\sphinxstrong{ = (844, 848)}}
\end{fulllineitems}

\index{regions (xanespy.edges.NCANickelLEdge attribute)}

\begin{fulllineitems}
\phantomsection\label{\detokenize{xanespy:xanespy.edges.NCANickelLEdge.regions}}\pysigline{\sphinxbfcode{regions}\sphinxstrong{ = {[}(844, 848, 1), (849, 856, 0.25), (857, 862, 1){]}}}
\end{fulllineitems}


\end{fulllineitems}

\index{NMCNickelKEdge29 (class in xanespy.edges)}

\begin{fulllineitems}
\phantomsection\label{\detokenize{xanespy:xanespy.edges.NMCNickelKEdge29}}\pysigline{\sphinxstrong{class }\sphinxcode{xanespy.edges.}\sphinxbfcode{NMCNickelKEdge29}}
Bases: {\hyperref[\detokenize{xanespy:xanespy.edges.NCANickelKEdge}]{\sphinxcrossref{\sphinxcode{xanespy.edges.NCANickelKEdge}}}}
\index{regions (xanespy.edges.NMCNickelKEdge29 attribute)}

\begin{fulllineitems}
\phantomsection\label{\detokenize{xanespy:xanespy.edges.NMCNickelKEdge29.regions}}\pysigline{\sphinxbfcode{regions}\sphinxstrong{ = {[}(8250, 8310, 20), (8324, 8346, 6), (8346, 8358, 1), (8360, 8400, 10), (8400, 8440, 15), (8440, 8640, 100){]}}}
\end{fulllineitems}


\end{fulllineitems}



\subsection{xanespy.exceptions module}
\label{\detokenize{xanespy:xanespy-exceptions-module}}\label{\detokenize{xanespy:module-xanespy.exceptions}}\index{xanespy.exceptions (module)}
Define classes for more fine-grained control over exception
handling.
\index{CreateGroupError}

\begin{fulllineitems}
\phantomsection\label{\detokenize{xanespy:xanespy.exceptions.CreateGroupError}}\pysigline{\sphinxstrong{exception }\sphinxcode{xanespy.exceptions.}\sphinxbfcode{CreateGroupError}}
Bases: \sphinxcode{ValueError}

Tried to import a TXM frameset into a group but the corresponding
HDF group already exists or is otherwise inaccessible.

\end{fulllineitems}

\index{DataFormatError}

\begin{fulllineitems}
\phantomsection\label{\detokenize{xanespy:xanespy.exceptions.DataFormatError}}\pysigline{\sphinxstrong{exception }\sphinxcode{xanespy.exceptions.}\sphinxbfcode{DataFormatError}}
Bases: \sphinxcode{RuntimeError}

The raw data are arranged in a way that the importers or TXM classes do
not understand.

\end{fulllineitems}

\index{DataNotFoundError}

\begin{fulllineitems}
\phantomsection\label{\detokenize{xanespy:xanespy.exceptions.DataNotFoundError}}\pysigline{\sphinxstrong{exception }\sphinxcode{xanespy.exceptions.}\sphinxbfcode{DataNotFoundError}}
Bases: \sphinxcode{FileNotFoundError}

Expected a directory containing data but found none.

\end{fulllineitems}

\index{DatasetExistsError}

\begin{fulllineitems}
\phantomsection\label{\detokenize{xanespy:xanespy.exceptions.DatasetExistsError}}\pysigline{\sphinxstrong{exception }\sphinxcode{xanespy.exceptions.}\sphinxbfcode{DatasetExistsError}}
Bases: \sphinxcode{RuntimeError}

Trying to save a new dataset but one already exists with the given
path.

\end{fulllineitems}

\index{FileExistsError}

\begin{fulllineitems}
\phantomsection\label{\detokenize{xanespy:xanespy.exceptions.FileExistsError}}\pysigline{\sphinxstrong{exception }\sphinxcode{xanespy.exceptions.}\sphinxbfcode{FileExistsError}}
Bases: \sphinxcode{OSError}

Tried to import a TXM frameset but the corresponding HDF file
already exists.

\end{fulllineitems}

\index{FilenameParseError}

\begin{fulllineitems}
\phantomsection\label{\detokenize{xanespy:xanespy.exceptions.FilenameParseError}}\pysigline{\sphinxstrong{exception }\sphinxcode{xanespy.exceptions.}\sphinxbfcode{FilenameParseError}}
Bases: \sphinxcode{ValueError}

The parameters in the filename do not match the naming scheme
associated with this flavor.

\end{fulllineitems}

\index{FrameFileNotFound}

\begin{fulllineitems}
\phantomsection\label{\detokenize{xanespy:xanespy.exceptions.FrameFileNotFound}}\pysigline{\sphinxstrong{exception }\sphinxcode{xanespy.exceptions.}\sphinxbfcode{FrameFileNotFound}}
Bases: \sphinxcode{OSError}

Expected to load a TXM frame file but it doesn’t exist.

\end{fulllineitems}

\index{GroupKeyError}

\begin{fulllineitems}
\phantomsection\label{\detokenize{xanespy:xanespy.exceptions.GroupKeyError}}\pysigline{\sphinxstrong{exception }\sphinxcode{xanespy.exceptions.}\sphinxbfcode{GroupKeyError}}
Bases: \sphinxcode{KeyError}

Tried to load or create an HDF group but failed. Examples include:
the group doesn’t exist, is ambiguous or already exists when being
created.

\end{fulllineitems}

\index{HDFScopeError}

\begin{fulllineitems}
\phantomsection\label{\detokenize{xanespy:xanespy.exceptions.HDFScopeError}}\pysigline{\sphinxstrong{exception }\sphinxcode{xanespy.exceptions.}\sphinxbfcode{HDFScopeError}}
Bases: \sphinxcode{ValueError}

Tried to pass an HDF scope that is not recognized.

\end{fulllineitems}

\index{NoParticleError}

\begin{fulllineitems}
\phantomsection\label{\detokenize{xanespy:xanespy.exceptions.NoParticleError}}\pysigline{\sphinxstrong{exception }\sphinxcode{xanespy.exceptions.}\sphinxbfcode{NoParticleError}}
Bases: \sphinxcode{Exception}

\end{fulllineitems}

\index{RefinementError}

\begin{fulllineitems}
\phantomsection\label{\detokenize{xanespy:xanespy.exceptions.RefinementError}}\pysigline{\sphinxstrong{exception }\sphinxcode{xanespy.exceptions.}\sphinxbfcode{RefinementError}}
Bases: \sphinxcode{RuntimeError}

\end{fulllineitems}

\index{XanesMathError}

\begin{fulllineitems}
\phantomsection\label{\detokenize{xanespy:xanespy.exceptions.XanesMathError}}\pysigline{\sphinxstrong{exception }\sphinxcode{xanespy.exceptions.}\sphinxbfcode{XanesMathError}}
Bases: \sphinxcode{RuntimeError}

\end{fulllineitems}



\subsection{xanespy.importers module}
\label{\detokenize{xanespy:module-xanespy.importers}}\label{\detokenize{xanespy:xanespy-importers-module}}\index{xanespy.importers (module)}\index{decode\_aps\_params() (in module xanespy.importers)}

\begin{fulllineitems}
\phantomsection\label{\detokenize{xanespy:xanespy.importers.decode_aps_params}}\pysiglinewithargsret{\sphinxcode{xanespy.importers.}\sphinxbfcode{decode\_aps\_params}}{\emph{filename}}{}
Accept the filename of an XRM file and return sample parameters as
a dictionary.

\end{fulllineitems}

\index{decode\_ssrl\_params() (in module xanespy.importers)}

\begin{fulllineitems}
\phantomsection\label{\detokenize{xanespy:xanespy.importers.decode_ssrl_params}}\pysiglinewithargsret{\sphinxcode{xanespy.importers.}\sphinxbfcode{decode\_ssrl\_params}}{\emph{filename}}{}
Accept the filename of an XRM file and return sample parameters as
a dictionary.

\end{fulllineitems}

\index{import\_aps\_8BM\_xanes\_dir() (in module xanespy.importers)}

\begin{fulllineitems}
\phantomsection\label{\detokenize{xanespy:xanespy.importers.import_aps_8BM_xanes_dir}}\pysiglinewithargsret{\sphinxcode{xanespy.importers.}\sphinxbfcode{import\_aps\_8BM\_xanes\_dir}}{\emph{directory}, \emph{hdf\_filename}, \emph{quiet=False}}{}
\end{fulllineitems}

\index{import\_aps\_8BM\_xanes\_file() (in module xanespy.importers)}

\begin{fulllineitems}
\phantomsection\label{\detokenize{xanespy:xanespy.importers.import_aps_8BM_xanes_file}}\pysiglinewithargsret{\sphinxcode{xanespy.importers.}\sphinxbfcode{import\_aps\_8BM\_xanes\_file}}{\emph{filename}, \emph{ref\_filename}, \emph{hdf\_filename}, \emph{groupname=None}}{}
Extract an entire xanes framestack from one xradia file.

A single TXRM file can contain multiple frames at different
energies. This function will import such a file along with the
corresponding reference frames into an HDF file. If the given
\sphinxcode{groupname} exists in \sphinxcode{hdf\_filename}, it will be overwritten
and a \sphinxcode{RuntimeWarning} will be issued.
\begin{quote}\begin{description}
\item[{Parameters}] \leavevmode\begin{itemize}
\item {} 
\sphinxstyleliteralstrong{filename} (\sphinxstyleliteralemphasis{str}) \textendash{} File path to the txrm file to import.

\item {} 
\sphinxstyleliteralstrong{ref\_filename} (\sphinxstyleliteralemphasis{str}) \textendash{} File path to the txrm file that contains the white-field
reference images. This will be used to calculate optical depth
from transmitted intensity.

\item {} 
\sphinxstyleliteralstrong{hdf\_filename} (\sphinxstyleliteralemphasis{str}) \textendash{} File path to the destination HDF5 file that will receive the
imported data.

\item {} 
\sphinxstyleliteralstrong{groupname} (\sphinxstyleliteralemphasis{str}\sphinxstyleliteralemphasis{, }\sphinxstyleliteralemphasis{optional}) \textendash{} The name for the top-level HDF group. If omitted, a group name
will be generated from the \sphinxcode{filename} parameter.

\end{itemize}

\end{description}\end{quote}

\end{fulllineitems}

\index{import\_frameset() (in module xanespy.importers)}

\begin{fulllineitems}
\phantomsection\label{\detokenize{xanespy:xanespy.importers.import_frameset}}\pysiglinewithargsret{\sphinxcode{xanespy.importers.}\sphinxbfcode{import\_frameset}}{\emph{directory}, \emph{flavor}, \emph{hdf\_filename}, \emph{return\_val=None}}{}
Import all files in the given directory collected at an X-ray
microscope beamline.

Images are assumed to full-field transmission X-ray micrographs.

If \sphinxcode{return\_val} is “group”, the return value for this function
will be a data group in an \sphinxstylestrong{open} HDF5 file. The underlying file
should be explicitly closed to avoid corruption. This is done
automatically if \sphinxcode{return\_val} is None (default).
\begin{quote}\begin{description}
\item[{Parameters}] \leavevmode\begin{itemize}
\item {} 
\sphinxstyleliteralstrong{directory} (\sphinxstyleliteralemphasis{str}) \textendash{} A valid path to a directory containing the frame data to import.

\item {} 
\sphinxstyleliteralstrong{flavor} (\sphinxstyleliteralemphasis{str}) \textendash{} Indicates what type of naming conventions and data structure to
assume. See documentation for \sphinxcode{xanespy.xradia.XRMFile} for
possible choice.

\item {} 
\sphinxstyleliteralstrong{hdf\_filename} (\sphinxstyleliteralemphasis{str}) \textendash{} Where to save the output to. An exception is throw if this file
already exists.

\item {} 
\sphinxstyleliteralstrong{return\_val} (\sphinxstyleliteralemphasis{str}) \textendash{} Request a specific return value.
- None: No return value
- “group”: The open HDF5 group for this experiment.

\end{itemize}

\end{description}\end{quote}

\end{fulllineitems}

\index{import\_nanosurveyor\_frameset() (in module xanespy.importers)}

\begin{fulllineitems}
\phantomsection\label{\detokenize{xanespy:xanespy.importers.import_nanosurveyor_frameset}}\pysiglinewithargsret{\sphinxcode{xanespy.importers.}\sphinxbfcode{import\_nanosurveyor\_frameset}}{\emph{directory: str}, \emph{quiet=False}, \emph{hdf\_filename=None}, \emph{hdf\_groupname=None}, \emph{energy\_range=None}, \emph{exclude\_re=None}, \emph{append=False}}{}
Import a set of images from reconstructed ptychography scanning microscope data.

This generates ptychography chemical maps based on data collected
at ALS beamline 5.3.2.1. The arguments \sphinxcode{energy\_range} and
\sphinxcode{exclude\_re} can be used to fine-tune the set of imported
file. For example: passing \sphinxcode{exclude\_re='(019\textbar{}017)'} will import
everything except scans 019 and 017.
\begin{quote}\begin{description}
\item[{Parameters}] \leavevmode\begin{itemize}
\item {} 
\sphinxstyleliteralstrong{directory} (\sphinxstyleliteralemphasis{str}) \textendash{} Directory where to look for results. It should contain .cxi
files that are the output of the ptychography reconstruction.”

\item {} 
\sphinxstyleliteralstrong{quiet} (\sphinxstyleliteralemphasis{Bool}\sphinxstyleliteralemphasis{, }\sphinxstyleliteralemphasis{optional}) \textendash{} If truthy, progress bars will not be shown.

\item {} 
\sphinxstyleliteralstrong{hdf\_filename} (\sphinxstyleliteralemphasis{str}\sphinxstyleliteralemphasis{, }\sphinxstyleliteralemphasis{optional}) \textendash{} HDF File used to store computed results. If omitted or None, the
\sphinxtitleref{directory} basename is used

\item {} 
\sphinxstyleliteralstrong{hdf\_groupname} (\sphinxstyleliteralemphasis{str}\sphinxstyleliteralemphasis{, }\sphinxstyleliteralemphasis{optional}) \textendash{} Name to use for the hdf group of this dataset. If omitted or
None, the \sphinxtitleref{directory} basename is used. Raises an exception if
the group already exists in the HDF file.

\item {} 
\sphinxstyleliteralstrong{energy\_range} (\sphinxstyleliteralemphasis{2-tuple}\sphinxstyleliteralemphasis{, }\sphinxstyleliteralemphasis{optional}) \textendash{} A 2-tuple with the (min, max) energy to be imported. This is
useful if only a subset of the available data is usable. Values
are assumed to be in electron-volts.

\item {} 
\sphinxstyleliteralstrong{exclude\_re} (\sphinxstyleliteralemphasis{str}\sphinxstyleliteralemphasis{, }\sphinxstyleliteralemphasis{optional}) \textendash{} Any filenames matching this regular expression will not be
imported. A string or compiled re object can be given.

\item {} 
\sphinxstyleliteralstrong{append} (\sphinxstyleliteralemphasis{bool}\sphinxstyleliteralemphasis{, }\sphinxstyleliteralemphasis{optional}) \textendash{} If True, any existing dataset will be added to, rather
than replaced (default False)

\end{itemize}

\end{description}\end{quote}

\end{fulllineitems}

\index{import\_ssrl\_xanes\_dir() (in module xanespy.importers)}

\begin{fulllineitems}
\phantomsection\label{\detokenize{xanespy:xanespy.importers.import_ssrl_xanes_dir}}\pysiglinewithargsret{\sphinxcode{xanespy.importers.}\sphinxbfcode{import\_ssrl\_xanes\_dir}}{\emph{directory}, \emph{hdf\_filename}}{}
Import all files in the given directory collected at SSRL beamline
6-2c and process into framesets. Images are assumed to full-field
transmission X-ray micrographs and repetitions will be
averaged. Passed on to \sphinxcode{xanespy.importers.import\_frameset}
\begin{quote}\begin{description}
\item[{Parameters}] \leavevmode\begin{itemize}
\item {} 
\sphinxstyleliteralstrong{directory} (\sphinxstyleliteralemphasis{str}) \textendash{} Where to look for files to import.

\item {} 
\sphinxstyleliteralstrong{hdf\_filename} (\sphinxstyleliteralemphasis{str}) \textendash{} Path to an HDF5 to receive the data.

\end{itemize}

\end{description}\end{quote}

\end{fulllineitems}

\index{import\_stxm\_frameset() (in module xanespy.importers)}

\begin{fulllineitems}
\phantomsection\label{\detokenize{xanespy:xanespy.importers.import_stxm_frameset}}\pysiglinewithargsret{\sphinxcode{xanespy.importers.}\sphinxbfcode{import\_stxm\_frameset}}{\emph{directory: str}, \emph{quiet=False}, \emph{hdf\_filename=None}, \emph{hdf\_groupname=None}, \emph{energy\_range=None}, \emph{exclude\_re=None}, \emph{append=False}}{}
Import a set of images from scanning microscope data.

This generates Scanning Tranmission X-ray Microscopy chemical maps
based on data collected at ALS beamline 5.3.2.1
\begin{quote}\begin{description}
\item[{Parameters}] \leavevmode\begin{itemize}
\item {} 
\sphinxstyleliteralstrong{directory} (\sphinxstyleliteralemphasis{str}) \textendash{} Directory where to look for results. It should contain .hdr and
xim files that are the output of the ptychography
reconstruction.”

\item {} 
\sphinxstyleliteralstrong{quiet} (\sphinxstyleliteralemphasis{Bool}\sphinxstyleliteralemphasis{, }\sphinxstyleliteralemphasis{optional}) \textendash{} If truthy, progress bars will not be shown.

\item {} 
\sphinxstyleliteralstrong{hdf\_filename} (\sphinxstyleliteralemphasis{str}\sphinxstyleliteralemphasis{, }\sphinxstyleliteralemphasis{optional}) \textendash{} HDF File used to store computed results. If omitted or None, the
\sphinxtitleref{directory} basename is used

\item {} 
\sphinxstyleliteralstrong{hdf\_groupname} (\sphinxstyleliteralemphasis{str}\sphinxstyleliteralemphasis{, }\sphinxstyleliteralemphasis{optional}) \textendash{} Name to use for the hdf group of this dataset. If omitted or
None, the \sphinxtitleref{directory} basename is used. Raises an exception if
the group already exists in the HDF file.

\item {} 
\sphinxstyleliteralstrong{energy\_range} (\sphinxstyleliteralemphasis{2-tuple}\sphinxstyleliteralemphasis{, }\sphinxstyleliteralemphasis{optional}) \textendash{} A 2-tuple with the (min, max) energy to be imported. This is
useful if only a subset of the available data is usable. Values
are assumed to be in electron-volts.

\item {} 
\sphinxstyleliteralstrong{exclude\_re} (\sphinxstyleliteralemphasis{str}\sphinxstyleliteralemphasis{, }\sphinxstyleliteralemphasis{optional}) \textendash{} Any filenames matching this regular expression will not be
imported. A string or compiled re object can be given.

\item {} 
\sphinxstyleliteralstrong{append} (\sphinxstyleliteralemphasis{bool}\sphinxstyleliteralemphasis{, }\sphinxstyleliteralemphasis{optional}) \textendash{} If True, any existing dataset will be added to, rather
than replaced (default False)

\end{itemize}

\end{description}\end{quote}

\end{fulllineitems}

\index{magnification\_correction() (in module xanespy.importers)}

\begin{fulllineitems}
\phantomsection\label{\detokenize{xanespy:xanespy.importers.magnification_correction}}\pysiglinewithargsret{\sphinxcode{xanespy.importers.}\sphinxbfcode{magnification\_correction}}{\emph{frames}, \emph{pixel\_sizes}}{}
Correct for changes in magnification at different energies.

As the X-ray energy increases, the focal length of the zone plate
changes and so the image is zoomed-out at higher energies. This
method applies a correction to each frame to make the
magnification similar to that of the first frame. Some beamlines
correct for this automatically during acquisition and don’t need
this function: APS 8-BM-B, 32-ID-C.
\begin{quote}\begin{description}
\item[{Parameters}] \leavevmode\begin{itemize}
\item {} 
\sphinxstyleliteralstrong{frames} (\sphinxstyleliteralemphasis{np.ndarray}) \textendash{} Numpy array of image frames that need to be corrected.

\item {} 
\sphinxstyleliteralstrong{pixel\_sizes} (\sphinxstyleliteralemphasis{np.ndarray}) \textendash{} Numpy array of pixel sizes corresponding to entries in \sphinxtitleref{frames}.

\end{itemize}

\item[{Returns}] \leavevmode
\sphinxstylestrong{(scales2D, translations)} \textendash{} An array of scale factors to use for applying a correction to
each frame. Translations show how much to move each frame array by to re-center it.

\item[{Return type}] \leavevmode
(np.ndarray, np.ndarray)

\end{description}\end{quote}

\end{fulllineitems}

\index{read\_metadata() (in module xanespy.importers)}

\begin{fulllineitems}
\phantomsection\label{\detokenize{xanespy:xanespy.importers.read_metadata}}\pysiglinewithargsret{\sphinxcode{xanespy.importers.}\sphinxbfcode{read\_metadata}}{\emph{filenames}, \emph{flavor}}{}
Take a list of filenames and return a pandas dataframe with all the
metadata.
\begin{quote}\begin{description}
\item[{Parameters}] \leavevmode\begin{itemize}
\item {} 
\sphinxstyleliteralstrong{filenames} (\sphinxstyleliteralemphasis{iterable}) \textendash{} Iterable of filenames to use for extracting metadata.

\item {} 
\sphinxstyleliteralstrong{flavor} (\sphinxstyleliteralemphasis{str}) \textendash{} Same as in \sphinxcode{import\_frameset}.

\end{itemize}

\end{description}\end{quote}

\end{fulllineitems}



\subsection{xanespy.plots module}
\label{\detokenize{xanespy:xanespy-plots-module}}\label{\detokenize{xanespy:module-xanespy.plots}}\index{xanespy.plots (module)}
Helper functions for setting up and displaying plots using matplotlib.
\index{big\_axes() (in module xanespy.plots)}

\begin{fulllineitems}
\phantomsection\label{\detokenize{xanespy:xanespy.plots.big_axes}}\pysiglinewithargsret{\sphinxcode{xanespy.plots.}\sphinxbfcode{big\_axes}}{}{}
Return a new Axes object, but larger than the default.

\end{fulllineitems}

\index{draw\_colorbar() (in module xanespy.plots)}

\begin{fulllineitems}
\phantomsection\label{\detokenize{xanespy:xanespy.plots.draw_colorbar}}\pysiglinewithargsret{\sphinxcode{xanespy.plots.}\sphinxbfcode{draw\_colorbar}}{\emph{ax}, \emph{cmap}, \emph{norm}, \emph{energies}, \emph{orientation=’vertical’}, \emph{*args}, \emph{**kwargs}}{}
Draw a colorbar on the side of a mapping axes to show the range of
colors used. Returns the newly created colorbar object.
\begin{quote}\begin{description}
\item[{Parameters}] \leavevmode\begin{itemize}
\item {} 
\sphinxstyleliteralstrong{ax} (\sphinxstyleliteralemphasis{-}) \textendash{} 

\item {} 
\sphinxstyleliteralstrong{cmap} (\sphinxstyleliteralemphasis{-}) \textendash{} to use.

\item {} 
\sphinxstyleliteralstrong{norm} (\sphinxstyleliteralemphasis{-}) \textendash{} use.

\item {} 
\sphinxstyleliteralstrong{energies} (\sphinxstyleliteralemphasis{-}) \textendash{} colorbar.

\end{itemize}

\end{description}\end{quote}

\end{fulllineitems}

\index{draw\_histogram\_colorbar() (in module xanespy.plots)}

\begin{fulllineitems}
\phantomsection\label{\detokenize{xanespy:xanespy.plots.draw_histogram_colorbar}}\pysiglinewithargsret{\sphinxcode{xanespy.plots.}\sphinxbfcode{draw\_histogram\_colorbar}}{\emph{ax}, \emph{*args}, \emph{**kwargs}}{}
Similar to \sphinxtitleref{draw\_colorbar()} with some special formatting options
to put it along the X-axis of the axes.

\end{fulllineitems}

\index{dual\_axes() (in module xanespy.plots)}

\begin{fulllineitems}
\phantomsection\label{\detokenize{xanespy:xanespy.plots.dual_axes}}\pysiglinewithargsret{\sphinxcode{xanespy.plots.}\sphinxbfcode{dual\_axes}}{\emph{fig=None}, \emph{longdim=13.8}, \emph{shortdim=6.9}, \emph{orientation=’horizontal’}}{}
Two new axes for mapping, side-by-side.
\begin{quote}\begin{description}
\item[{Parameters}] \leavevmode\begin{itemize}
\item {} 
\sphinxstyleliteralstrong{longdim} (\sphinxstyleliteralemphasis{float}) \textendash{} Size in inches for the long dimension. If orientation is
“vertical”, this will be the height.

\item {} 
\sphinxstyleliteralstrong{shortdim} (\sphinxstyleliteralemphasis{float}) \textendash{} Size in inches for the short dimension. If orientation is
“vertical”, this will be the width.

\end{itemize}

\end{description}\end{quote}

\end{fulllineitems}

\index{new\_axes() (in module xanespy.plots)}

\begin{fulllineitems}
\phantomsection\label{\detokenize{xanespy:xanespy.plots.new_axes}}\pysiglinewithargsret{\sphinxcode{xanespy.plots.}\sphinxbfcode{new\_axes}}{\emph{height=5}, \emph{width=None}}{}
Create a new set of matplotlib axes for plotting. Height in inches.

\end{fulllineitems}

\index{new\_image\_axes() (in module xanespy.plots)}

\begin{fulllineitems}
\phantomsection\label{\detokenize{xanespy:xanespy.plots.new_image_axes}}\pysiglinewithargsret{\sphinxcode{xanespy.plots.}\sphinxbfcode{new\_image\_axes}}{\emph{height=5}, \emph{width=5}}{}
Square axes with ticks on the outside.

\end{fulllineitems}

\index{plot\_composite\_map() (in module xanespy.plots)}

\begin{fulllineitems}
\phantomsection\label{\detokenize{xanespy:xanespy.plots.plot_composite_map}}\pysiglinewithargsret{\sphinxcode{xanespy.plots.}\sphinxbfcode{plot\_composite\_map}}{\emph{data}, \emph{ax=None}, \emph{origin=’upper’}, \emph{*args}, \emph{**kwargs}}{}
Plot an RGB composite map on the given axes.

\end{fulllineitems}

\index{plot\_pixel\_spectra() (in module xanespy.plots)}

\begin{fulllineitems}
\phantomsection\label{\detokenize{xanespy:xanespy.plots.plot_pixel_spectra}}\pysiglinewithargsret{\sphinxcode{xanespy.plots.}\sphinxbfcode{plot\_pixel\_spectra}}{\emph{pixels}, \emph{extent}, \emph{spectra}, \emph{energies}, \emph{map\_ax}, \emph{spectra\_ax=None}, \emph{step\_size=0}}{}
Highlight certain pixels in an already-plotted map and plot their
spectra. The map should already have been plotted.
\begin{quote}\begin{description}
\item[{Parameters}] \leavevmode\begin{itemize}
\item {} 
\sphinxstyleliteralstrong{pixels} (\sphinxstyleliteralemphasis{-}) \textendash{} to highlight and plot.

\item {} 
\sphinxstyleliteralstrong{extent} (\sphinxstyleliteralemphasis{-}) \textendash{} positions to (row, column) positions.

\item {} 
\sphinxstyleliteralstrong{spectra} (\sphinxstyleliteralemphasis{-}) \textendash{} in \sphinxtitleref{pixels} will use this array to get spectra. Shape is assumed
to be (row, column, energy).

\item {} 
\sphinxstyleliteralstrong{energies} (\sphinxstyleliteralemphasis{-}) \textendash{} plotting spectra.

\item {} 
\sphinxstyleliteralstrong{map\_ax} (\sphinxstyleliteralemphasis{-}) \textendash{} pixels.

\item {} 
\sphinxstyleliteralstrong{spectra\_ax} (\sphinxstyleliteralemphasis{-}) \textendash{} None (default) a new axes will be created.

\item {} 
\sphinxstyleliteralstrong{step\_size} (\sphinxstyleliteralemphasis{-}) \textendash{} directly on top of each other.

\end{itemize}

\end{description}\end{quote}

\end{fulllineitems}

\index{plot\_txm\_histogram() (in module xanespy.plots)}

\begin{fulllineitems}
\phantomsection\label{\detokenize{xanespy:xanespy.plots.plot_txm_histogram}}\pysiglinewithargsret{\sphinxcode{xanespy.plots.}\sphinxbfcode{plot\_txm\_histogram}}{\emph{data}, \emph{ax=None}, \emph{norm=None}, \emph{bins=None}, \emph{cmap=’plasma’}, \emph{add\_cbar=True}, \emph{*args}, \emph{**kwargs}}{}
Take an array of data values and show a histogram with some
color-coding related to normalization value.

Returns: The matplotlib axes object used for plotting.
\begin{quote}\begin{description}
\item[{Parameters}] \leavevmode\begin{itemize}
\item {} 
\sphinxstyleliteralstrong{data} (\sphinxstyleliteralemphasis{np.ndarray}) \textendash{} An array of values to plot on the histogram.

\item {} 
\sphinxstyleliteralstrong{ax} (\sphinxstyleliteralemphasis{optional}) \textendash{} Matplotlib Axes instance to receive the plot. If None, a new axes
will created.

\item {} 
\sphinxstyleliteralstrong{norm} (\sphinxstyleliteralemphasis{optional}) \textendash{} Matplotlib Normalize instance with the colormap range.

\item {} 
\sphinxstyleliteralstrong{bins} (\sphinxstyleliteralemphasis{optional}) \textendash{} Bins to pass to the matplotlib hist() routine. If None
(default), we will choose based on dtype of the data: integers
will yield 1-wide bins, anything else will give 256 bins.

\item {} 
\sphinxstyleliteralstrong{cmap} (\sphinxstyleliteralemphasis{str}\sphinxstyleliteralemphasis{, }\sphinxstyleliteralemphasis{optional}) \textendash{} Matplotlib colormap for coloring the bars.

\item {} 
\sphinxstyleliteralstrong{add\_cbar} (\sphinxstyleliteralemphasis{bool}\sphinxstyleliteralemphasis{, }\sphinxstyleliteralemphasis{optional}) \textendash{} Boolean to decide whether to add a colorbar along the
bottom axis or not.

\item {} 
\sphinxstyleliteralstrong{*args} \textendash{} Positional arguments passed to matplotlib’s \sphinxtitleref{hist} call.

\item {} 
\sphinxstyleliteralstrong{*kwargs} \textendash{} Keyword arguments passed to matplotlib’s \sphinxtitleref{hist} call.

\end{itemize}

\end{description}\end{quote}

\end{fulllineitems}

\index{plot\_txm\_intermediates() (in module xanespy.plots)}

\begin{fulllineitems}
\phantomsection\label{\detokenize{xanespy:xanespy.plots.plot_txm_intermediates}}\pysiglinewithargsret{\sphinxcode{xanespy.plots.}\sphinxbfcode{plot\_txm\_intermediates}}{\emph{images}}{}
Accept a dictionary of images and plots them each on its own axes
using matplotlib’s \sphinxtitleref{imshow}. This is a complement to routines that
operate on a microscopy frame and optionally return all the
intermediate calculated frames.

\end{fulllineitems}

\index{plot\_txm\_map() (in module xanespy.plots)}

\begin{fulllineitems}
\phantomsection\label{\detokenize{xanespy:xanespy.plots.plot_txm_map}}\pysiglinewithargsret{\sphinxcode{xanespy.plots.}\sphinxbfcode{plot\_txm\_map}}{\emph{data}, \emph{edge=None}, \emph{norm=None}, \emph{ax=None}, \emph{cmap=’plasma’}, \emph{origin=’upper’}, \emph{vmin=None}, \emph{vmax=None}, \emph{*args}, \emph{**kwargs}}{}
\end{fulllineitems}

\index{plot\_xanes\_spectrum() (in module xanespy.plots)}

\begin{fulllineitems}
\phantomsection\label{\detokenize{xanespy:xanespy.plots.plot_xanes_spectrum}}\pysiglinewithargsret{\sphinxcode{xanespy.plots.}\sphinxbfcode{plot\_xanes\_spectrum}}{\emph{spectrum}, \emph{energies}, \emph{norm=\textless{}matplotlib.colors.Normalize object\textgreater{}}, \emph{show\_fit=False}, \emph{ax=None}, \emph{ax2=None}, \emph{linestyle=’:’}, \emph{color=’blue’}, \emph{cmap=’plasma’}, \emph{*args}, \emph{**kwargs}}{}
Plot a XANES spectrum on an axes. Applies some color formatting if
\sphinxtitleref{edge} is a valid XANES Edge object.
\begin{quote}\begin{description}
\item[{Parameters}] \leavevmode\begin{itemize}
\item {} 
\sphinxstyleliteralstrong{spectrum} (\sphinxstyleliteralemphasis{-}) \textendash{} 

\item {} 
\sphinxstyleliteralstrong{energies} (\sphinxstyleliteralemphasis{-}) \textendash{} 

\item {} 
\sphinxstyleliteralstrong{norm} (\sphinxstyleliteralemphasis{-}) \textendash{} range. This will be used to annotate the plot if it is give.

\item {} 
\sphinxstyleliteralstrong{show\_fit} (\sphinxstyleliteralemphasis{-}) \textendash{} 

\item {} 
\sphinxstyleliteralstrong{ax} (\sphinxstyleliteralemphasis{-}) \textendash{} will be generated.

\item {} 
\sphinxstyleliteralstrong{ax2} (\sphinxstyleliteralemphasis{-}) \textendash{} the data are complex.

\item {} 
\sphinxstyleliteralstrong{linestyle} (\sphinxstyleliteralemphasis{-}) \textendash{} 

\item {} 
\sphinxstyleliteralstrong{cmap} (\sphinxstyleliteralemphasis{-}) \textendash{} 

\item {} 
\sphinxstyleliteralstrong{color} (\sphinxstyleliteralemphasis{-}) \textendash{} or “y” will decide based on the numerical value, \sphinxtitleref{norm} and
\sphinxtitleref{cmap} arguments. Anything else will be passed as a color spec
to the matplotlib commands.

\end{itemize}

\end{description}\end{quote}

\end{fulllineitems}

\index{remove\_extra\_spines() (in module xanespy.plots)}

\begin{fulllineitems}
\phantomsection\label{\detokenize{xanespy:xanespy.plots.remove_extra_spines}}\pysiglinewithargsret{\sphinxcode{xanespy.plots.}\sphinxbfcode{remove\_extra\_spines}}{\emph{ax}}{}
Removes the right and top borders from the axes.

\end{fulllineitems}

\index{set\_axes\_color() (in module xanespy.plots)}

\begin{fulllineitems}
\phantomsection\label{\detokenize{xanespy:xanespy.plots.set_axes_color}}\pysiglinewithargsret{\sphinxcode{xanespy.plots.}\sphinxbfcode{set\_axes\_color}}{\emph{ax}, \emph{color}}{}
Set the axes, tick marks, etc of \sphinxtitleref{ax} to mpl color \sphinxtitleref{color}. Also,
“doegreen” has special significance as the color associated with the
US department of energy.

\end{fulllineitems}

\index{set\_outside\_ticks() (in module xanespy.plots)}

\begin{fulllineitems}
\phantomsection\label{\detokenize{xanespy:xanespy.plots.set_outside_ticks}}\pysiglinewithargsret{\sphinxcode{xanespy.plots.}\sphinxbfcode{set\_outside\_ticks}}{\emph{ax}}{}
Convert all the axes so that the ticks are on the outside and don’t
obscure data.

\end{fulllineitems}



\subsection{xanespy.qt\_frame\_view module}
\label{\detokenize{xanespy:module-xanespy.qt_frame_view}}\label{\detokenize{xanespy:xanespy-qt-frame-view-module}}\index{xanespy.qt\_frame\_view (module)}\index{FrameAnimation (class in xanespy.qt\_frame\_view)}

\begin{fulllineitems}
\phantomsection\label{\detokenize{xanespy:xanespy.qt_frame_view.FrameAnimation}}\pysiglinewithargsret{\sphinxstrong{class }\sphinxcode{xanespy.qt\_frame\_view.}\sphinxbfcode{FrameAnimation}}{\emph{fig}, \emph{artists}, \emph{*args}, \emph{**kwargs}}{}
Bases: \sphinxcode{matplotlib.animation.ArtistAnimation}

Performs the animation for scrolling through frames arbitarily.
\index{stop() (xanespy.qt\_frame\_view.FrameAnimation method)}

\begin{fulllineitems}
\phantomsection\label{\detokenize{xanespy:xanespy.qt_frame_view.FrameAnimation.stop}}\pysiglinewithargsret{\sphinxbfcode{stop}}{}{}
\end{fulllineitems}


\end{fulllineitems}

\index{FrameChangeSource (class in xanespy.qt\_frame\_view)}

\begin{fulllineitems}
\phantomsection\label{\detokenize{xanespy:xanespy.qt_frame_view.FrameChangeSource}}\pysiglinewithargsret{\sphinxstrong{class }\sphinxcode{xanespy.qt\_frame\_view.}\sphinxbfcode{FrameChangeSource}}{\emph{view}, \emph{*args}, \emph{**kwargs}}{}
Bases: \sphinxcode{PyQt5.QtCore.QObject}
\index{add\_callback() (xanespy.qt\_frame\_view.FrameChangeSource method)}

\begin{fulllineitems}
\phantomsection\label{\detokenize{xanespy:xanespy.qt_frame_view.FrameChangeSource.add_callback}}\pysiglinewithargsret{\sphinxbfcode{add\_callback}}{\emph{func}, \emph{*args}, \emph{**kwargs}}{}
\end{fulllineitems}

\index{callbacks (xanespy.qt\_frame\_view.FrameChangeSource attribute)}

\begin{fulllineitems}
\phantomsection\label{\detokenize{xanespy:xanespy.qt_frame_view.FrameChangeSource.callbacks}}\pysigline{\sphinxbfcode{callbacks}\sphinxstrong{ = {[}{]}}}
\end{fulllineitems}

\index{remove\_callback() (xanespy.qt\_frame\_view.FrameChangeSource method)}

\begin{fulllineitems}
\phantomsection\label{\detokenize{xanespy:xanespy.qt_frame_view.FrameChangeSource.remove_callback}}\pysiglinewithargsret{\sphinxbfcode{remove\_callback}}{\emph{func}, \emph{*args}, \emph{**kwargs}}{}
\end{fulllineitems}

\index{start() (xanespy.qt\_frame\_view.FrameChangeSource method)}

\begin{fulllineitems}
\phantomsection\label{\detokenize{xanespy:xanespy.qt_frame_view.FrameChangeSource.start}}\pysiglinewithargsret{\sphinxbfcode{start}}{}{}
\end{fulllineitems}

\index{stop() (xanespy.qt\_frame\_view.FrameChangeSource method)}

\begin{fulllineitems}
\phantomsection\label{\detokenize{xanespy:xanespy.qt_frame_view.FrameChangeSource.stop}}\pysiglinewithargsret{\sphinxbfcode{stop}}{}{}
\end{fulllineitems}


\end{fulllineitems}

\index{QtFrameView (class in xanespy.qt\_frame\_view)}

\begin{fulllineitems}
\phantomsection\label{\detokenize{xanespy:xanespy.qt_frame_view.QtFrameView}}\pysigline{\sphinxstrong{class }\sphinxcode{xanespy.qt\_frame\_view.}\sphinxbfcode{QtFrameView}}
Bases: \sphinxcode{PyQt5.QtCore.QObject}
\index{add\_hdf\_tree\_item() (xanespy.qt\_frame\_view.QtFrameView method)}

\begin{fulllineitems}
\phantomsection\label{\detokenize{xanespy:xanespy.qt_frame_view.QtFrameView.add_hdf_tree_item}}\pysiglinewithargsret{\sphinxbfcode{add\_hdf\_tree\_item}}{\emph{item}}{}
\end{fulllineitems}

\index{clear\_axes() (xanespy.qt\_frame\_view.QtFrameView method)}

\begin{fulllineitems}
\phantomsection\label{\detokenize{xanespy:xanespy.qt_frame_view.QtFrameView.clear_axes}}\pysiglinewithargsret{\sphinxbfcode{clear\_axes}}{}{}
\end{fulllineitems}

\index{connect\_signals() (xanespy.qt\_frame\_view.QtFrameView method)}

\begin{fulllineitems}
\phantomsection\label{\detokenize{xanespy:xanespy.qt_frame_view.QtFrameView.connect_signals}}\pysiglinewithargsret{\sphinxbfcode{connect\_signals}}{\emph{presenter}}{}
\end{fulllineitems}

\index{create\_canvas() (xanespy.qt\_frame\_view.QtFrameView method)}

\begin{fulllineitems}
\phantomsection\label{\detokenize{xanespy:xanespy.qt_frame_view.QtFrameView.create_canvas}}\pysiglinewithargsret{\sphinxbfcode{create\_canvas}}{}{}
\end{fulllineitems}

\index{create\_status\_bar() (xanespy.qt\_frame\_view.QtFrameView method)}

\begin{fulllineitems}
\phantomsection\label{\detokenize{xanespy:xanespy.qt_frame_view.QtFrameView.create_status_bar}}\pysiglinewithargsret{\sphinxbfcode{create\_status\_bar}}{}{}
\end{fulllineitems}

\index{disable\_frame\_controls() (xanespy.qt\_frame\_view.QtFrameView method)}

\begin{fulllineitems}
\phantomsection\label{\detokenize{xanespy:xanespy.qt_frame_view.QtFrameView.disable_frame_controls}}\pysiglinewithargsret{\sphinxbfcode{disable\_frame\_controls}}{\emph{status}}{}
\end{fulllineitems}

\index{disable\_plotting\_controls() (xanespy.qt\_frame\_view.QtFrameView method)}

\begin{fulllineitems}
\phantomsection\label{\detokenize{xanespy:xanespy.qt_frame_view.QtFrameView.disable_plotting_controls}}\pysiglinewithargsret{\sphinxbfcode{disable\_plotting\_controls}}{\emph{status}}{}
\end{fulllineitems}

\index{draw\_frames (xanespy.qt\_frame\_view.QtFrameView attribute)}

\begin{fulllineitems}
\phantomsection\label{\detokenize{xanespy:xanespy.qt_frame_view.QtFrameView.draw_frames}}\pysigline{\sphinxbfcode{draw\_frames}}
\end{fulllineitems}

\index{draw\_histogram (xanespy.qt\_frame\_view.QtFrameView attribute)}

\begin{fulllineitems}
\phantomsection\label{\detokenize{xanespy:xanespy.qt_frame_view.QtFrameView.draw_histogram}}\pysigline{\sphinxbfcode{draw\_histogram}}
\end{fulllineitems}

\index{draw\_spectrum() (xanespy.qt\_frame\_view.QtFrameView method)}

\begin{fulllineitems}
\phantomsection\label{\detokenize{xanespy:xanespy.qt_frame_view.QtFrameView.draw_spectrum}}\pysiglinewithargsret{\sphinxbfcode{draw\_spectrum}}{\emph{spectrum}, \emph{energies}, \emph{norm}, \emph{cmap}, \emph{edge\_range}}{}
\end{fulllineitems}

\index{edge\_ax (xanespy.qt\_frame\_view.QtFrameView attribute)}

\begin{fulllineitems}
\phantomsection\label{\detokenize{xanespy:xanespy.qt_frame_view.QtFrameView.edge_ax}}\pysigline{\sphinxbfcode{edge\_ax}\sphinxstrong{ = None}}
\end{fulllineitems}

\index{expand\_hdf\_tree (xanespy.qt\_frame\_view.QtFrameView attribute)}

\begin{fulllineitems}
\phantomsection\label{\detokenize{xanespy:xanespy.qt_frame_view.QtFrameView.expand_hdf_tree}}\pysigline{\sphinxbfcode{expand\_hdf\_tree}}
\end{fulllineitems}

\index{fig (xanespy.qt\_frame\_view.QtFrameView attribute)}

\begin{fulllineitems}
\phantomsection\label{\detokenize{xanespy:xanespy.qt_frame_view.QtFrameView.fig}}\pysigline{\sphinxbfcode{fig}\sphinxstrong{ = None}}
\end{fulllineitems}

\index{frame\_changed (xanespy.qt\_frame\_view.QtFrameView attribute)}

\begin{fulllineitems}
\phantomsection\label{\detokenize{xanespy:xanespy.qt_frame_view.QtFrameView.frame_changed}}\pysigline{\sphinxbfcode{frame\_changed}}
\end{fulllineitems}

\index{frame\_controls (xanespy.qt\_frame\_view.QtFrameView attribute)}

\begin{fulllineitems}
\phantomsection\label{\detokenize{xanespy:xanespy.qt_frame_view.QtFrameView.frame_controls}}\pysigline{\sphinxbfcode{frame\_controls}}
Gives a list of all the UI buttons that are associated with
changing the currently active frame.

\end{fulllineitems}

\index{hist\_ax (xanespy.qt\_frame\_view.QtFrameView attribute)}

\begin{fulllineitems}
\phantomsection\label{\detokenize{xanespy:xanespy.qt_frame_view.QtFrameView.hist_ax}}\pysigline{\sphinxbfcode{hist\_ax}\sphinxstrong{ = None}}
\end{fulllineitems}

\index{hist\_cb (xanespy.qt\_frame\_view.QtFrameView attribute)}

\begin{fulllineitems}
\phantomsection\label{\detokenize{xanespy:xanespy.qt_frame_view.QtFrameView.hist_cb}}\pysigline{\sphinxbfcode{hist\_cb}\sphinxstrong{ = None}}
\end{fulllineitems}

\index{plotting\_controls (xanespy.qt\_frame\_view.QtFrameView attribute)}

\begin{fulllineitems}
\phantomsection\label{\detokenize{xanespy:xanespy.qt_frame_view.QtFrameView.plotting_controls}}\pysigline{\sphinxbfcode{plotting\_controls}}
Gives a list of all the UI elements that are associated with
changing how the frameset is plotted.

\end{fulllineitems}

\index{select\_active\_hdf\_item() (xanespy.qt\_frame\_view.QtFrameView method)}

\begin{fulllineitems}
\phantomsection\label{\detokenize{xanespy:xanespy.qt_frame_view.QtFrameView.select_active_hdf_item}}\pysiglinewithargsret{\sphinxbfcode{select\_active\_hdf\_item}}{\emph{item}}{}
\end{fulllineitems}

\index{set\_cmap() (xanespy.qt\_frame\_view.QtFrameView method)}

\begin{fulllineitems}
\phantomsection\label{\detokenize{xanespy:xanespy.qt_frame_view.QtFrameView.set_cmap}}\pysiglinewithargsret{\sphinxbfcode{set\_cmap}}{\emph{cmap}}{}
\end{fulllineitems}

\index{set\_cmap\_list() (xanespy.qt\_frame\_view.QtFrameView method)}

\begin{fulllineitems}
\phantomsection\label{\detokenize{xanespy:xanespy.qt_frame_view.QtFrameView.set_cmap_list}}\pysiglinewithargsret{\sphinxbfcode{set\_cmap\_list}}{\emph{cmap\_list}}{}
\end{fulllineitems}

\index{set\_component() (xanespy.qt\_frame\_view.QtFrameView method)}

\begin{fulllineitems}
\phantomsection\label{\detokenize{xanespy:xanespy.qt_frame_view.QtFrameView.set_component}}\pysiglinewithargsret{\sphinxbfcode{set\_component}}{\emph{comp}}{}
\end{fulllineitems}

\index{set\_component\_list() (xanespy.qt\_frame\_view.QtFrameView method)}

\begin{fulllineitems}
\phantomsection\label{\detokenize{xanespy:xanespy.qt_frame_view.QtFrameView.set_component_list}}\pysiglinewithargsret{\sphinxbfcode{set\_component\_list}}{\emph{component\_list}}{}
\end{fulllineitems}

\index{set\_drawing\_status() (xanespy.qt\_frame\_view.QtFrameView method)}

\begin{fulllineitems}
\phantomsection\label{\detokenize{xanespy:xanespy.qt_frame_view.QtFrameView.set_drawing_status}}\pysiglinewithargsret{\sphinxbfcode{set\_drawing\_status}}{\emph{status}}{}
\end{fulllineitems}

\index{set\_slider\_max() (xanespy.qt\_frame\_view.QtFrameView method)}

\begin{fulllineitems}
\phantomsection\label{\detokenize{xanespy:xanespy.qt_frame_view.QtFrameView.set_slider_max}}\pysiglinewithargsret{\sphinxbfcode{set\_slider\_max}}{\emph{val}}{}
\end{fulllineitems}

\index{set\_status\_cursor() (xanespy.qt\_frame\_view.QtFrameView method)}

\begin{fulllineitems}
\phantomsection\label{\detokenize{xanespy:xanespy.qt_frame_view.QtFrameView.set_status_cursor}}\pysiglinewithargsret{\sphinxbfcode{set\_status\_cursor}}{\emph{msg}}{}
\end{fulllineitems}

\index{set\_status\_energy() (xanespy.qt\_frame\_view.QtFrameView method)}

\begin{fulllineitems}
\phantomsection\label{\detokenize{xanespy:xanespy.qt_frame_view.QtFrameView.set_status_energy}}\pysiglinewithargsret{\sphinxbfcode{set\_status\_energy}}{\emph{msg}}{}
\end{fulllineitems}

\index{set\_status\_index() (xanespy.qt\_frame\_view.QtFrameView method)}

\begin{fulllineitems}
\phantomsection\label{\detokenize{xanespy:xanespy.qt_frame_view.QtFrameView.set_status_index}}\pysiglinewithargsret{\sphinxbfcode{set\_status\_index}}{\emph{msg}}{}
\end{fulllineitems}

\index{set\_status\_pixel() (xanespy.qt\_frame\_view.QtFrameView method)}

\begin{fulllineitems}
\phantomsection\label{\detokenize{xanespy:xanespy.qt_frame_view.QtFrameView.set_status_pixel}}\pysiglinewithargsret{\sphinxbfcode{set\_status\_pixel}}{\emph{msg}}{}
\end{fulllineitems}

\index{set\_status\_shape() (xanespy.qt\_frame\_view.QtFrameView method)}

\begin{fulllineitems}
\phantomsection\label{\detokenize{xanespy:xanespy.qt_frame_view.QtFrameView.set_status_shape}}\pysiglinewithargsret{\sphinxbfcode{set\_status\_shape}}{\emph{msg}}{}
\end{fulllineitems}

\index{set\_status\_unit() (xanespy.qt\_frame\_view.QtFrameView method)}

\begin{fulllineitems}
\phantomsection\label{\detokenize{xanespy:xanespy.qt_frame_view.QtFrameView.set_status_unit}}\pysiglinewithargsret{\sphinxbfcode{set\_status\_unit}}{\emph{msg}}{}
\end{fulllineitems}

\index{set\_status\_value() (xanespy.qt\_frame\_view.QtFrameView method)}

\begin{fulllineitems}
\phantomsection\label{\detokenize{xanespy:xanespy.qt_frame_view.QtFrameView.set_status_value}}\pysiglinewithargsret{\sphinxbfcode{set\_status\_value}}{\emph{msg}}{}
\end{fulllineitems}

\index{set\_timestep() (xanespy.qt\_frame\_view.QtFrameView method)}

\begin{fulllineitems}
\phantomsection\label{\detokenize{xanespy:xanespy.qt_frame_view.QtFrameView.set_timestep}}\pysiglinewithargsret{\sphinxbfcode{set\_timestep}}{\emph{idx}}{}
\end{fulllineitems}

\index{set\_timestep\_list() (xanespy.qt\_frame\_view.QtFrameView method)}

\begin{fulllineitems}
\phantomsection\label{\detokenize{xanespy:xanespy.qt_frame_view.QtFrameView.set_timestep_list}}\pysiglinewithargsret{\sphinxbfcode{set\_timestep\_list}}{\emph{timestep\_list}}{}
\end{fulllineitems}

\index{set\_ui\_enabled() (xanespy.qt\_frame\_view.QtFrameView method)}

\begin{fulllineitems}
\phantomsection\label{\detokenize{xanespy:xanespy.qt_frame_view.QtFrameView.set_ui_enabled}}\pysiglinewithargsret{\sphinxbfcode{set\_ui\_enabled}}{\emph{enable=True}}{}
Turn on (default) or off the main interactive elements in the
frame window. Useful for indicating blocking operations.

\end{fulllineitems}

\index{set\_vmax() (xanespy.qt\_frame\_view.QtFrameView method)}

\begin{fulllineitems}
\phantomsection\label{\detokenize{xanespy:xanespy.qt_frame_view.QtFrameView.set_vmax}}\pysiglinewithargsret{\sphinxbfcode{set\_vmax}}{\emph{val}}{}
\end{fulllineitems}

\index{set\_vmax\_decimals() (xanespy.qt\_frame\_view.QtFrameView method)}

\begin{fulllineitems}
\phantomsection\label{\detokenize{xanespy:xanespy.qt_frame_view.QtFrameView.set_vmax_decimals}}\pysiglinewithargsret{\sphinxbfcode{set\_vmax\_decimals}}{\emph{val}}{}
\end{fulllineitems}

\index{set\_vmax\_minimum() (xanespy.qt\_frame\_view.QtFrameView method)}

\begin{fulllineitems}
\phantomsection\label{\detokenize{xanespy:xanespy.qt_frame_view.QtFrameView.set_vmax_minimum}}\pysiglinewithargsret{\sphinxbfcode{set\_vmax\_minimum}}{\emph{val}}{}
\end{fulllineitems}

\index{set\_vmax\_step() (xanespy.qt\_frame\_view.QtFrameView method)}

\begin{fulllineitems}
\phantomsection\label{\detokenize{xanespy:xanespy.qt_frame_view.QtFrameView.set_vmax_step}}\pysiglinewithargsret{\sphinxbfcode{set\_vmax\_step}}{\emph{val}}{}
\end{fulllineitems}

\index{set\_vmin() (xanespy.qt\_frame\_view.QtFrameView method)}

\begin{fulllineitems}
\phantomsection\label{\detokenize{xanespy:xanespy.qt_frame_view.QtFrameView.set_vmin}}\pysiglinewithargsret{\sphinxbfcode{set\_vmin}}{\emph{val}}{}
\end{fulllineitems}

\index{set\_vmin\_decimals() (xanespy.qt\_frame\_view.QtFrameView method)}

\begin{fulllineitems}
\phantomsection\label{\detokenize{xanespy:xanespy.qt_frame_view.QtFrameView.set_vmin_decimals}}\pysiglinewithargsret{\sphinxbfcode{set\_vmin\_decimals}}{\emph{val}}{}
\end{fulllineitems}

\index{set\_vmin\_maximum() (xanespy.qt\_frame\_view.QtFrameView method)}

\begin{fulllineitems}
\phantomsection\label{\detokenize{xanespy:xanespy.qt_frame_view.QtFrameView.set_vmin_maximum}}\pysiglinewithargsret{\sphinxbfcode{set\_vmin\_maximum}}{\emph{val}}{}
\end{fulllineitems}

\index{set\_vmin\_step() (xanespy.qt\_frame\_view.QtFrameView method)}

\begin{fulllineitems}
\phantomsection\label{\detokenize{xanespy:xanespy.qt_frame_view.QtFrameView.set_vmin_step}}\pysiglinewithargsret{\sphinxbfcode{set\_vmin\_step}}{\emph{val}}{}
\end{fulllineitems}

\index{set\_window\_title() (xanespy.qt\_frame\_view.QtFrameView method)}

\begin{fulllineitems}
\phantomsection\label{\detokenize{xanespy:xanespy.qt_frame_view.QtFrameView.set_window_title}}\pysiglinewithargsret{\sphinxbfcode{set\_window\_title}}{\emph{title}}{}
\end{fulllineitems}

\index{setup() (xanespy.qt\_frame\_view.QtFrameView method)}

\begin{fulllineitems}
\phantomsection\label{\detokenize{xanespy:xanespy.qt_frame_view.QtFrameView.setup}}\pysiglinewithargsret{\sphinxbfcode{setup}}{}{}
\end{fulllineitems}

\index{show() (xanespy.qt\_frame\_view.QtFrameView method)}

\begin{fulllineitems}
\phantomsection\label{\detokenize{xanespy:xanespy.qt_frame_view.QtFrameView.show}}\pysiglinewithargsret{\sphinxbfcode{show}}{}{}
\end{fulllineitems}

\index{show\_status\_message() (xanespy.qt\_frame\_view.QtFrameView method)}

\begin{fulllineitems}
\phantomsection\label{\detokenize{xanespy:xanespy.qt_frame_view.QtFrameView.show_status_message}}\pysiglinewithargsret{\sphinxbfcode{show\_status\_message}}{\emph{message}}{}
\end{fulllineitems}

\index{spectrum\_ax (xanespy.qt\_frame\_view.QtFrameView attribute)}

\begin{fulllineitems}
\phantomsection\label{\detokenize{xanespy:xanespy.qt_frame_view.QtFrameView.spectrum_ax}}\pysigline{\sphinxbfcode{spectrum\_ax}\sphinxstrong{ = None}}
\end{fulllineitems}

\index{ui (xanespy.qt\_frame\_view.QtFrameView attribute)}

\begin{fulllineitems}
\phantomsection\label{\detokenize{xanespy:xanespy.qt_frame_view.QtFrameView.ui}}\pysigline{\sphinxbfcode{ui}\sphinxstrong{ = None}}
\end{fulllineitems}

\index{use\_busy\_cursor() (xanespy.qt\_frame\_view.QtFrameView method)}

\begin{fulllineitems}
\phantomsection\label{\detokenize{xanespy:xanespy.qt_frame_view.QtFrameView.use_busy_cursor}}\pysiglinewithargsret{\sphinxbfcode{use\_busy\_cursor}}{\emph{status}}{}
\end{fulllineitems}

\index{window (xanespy.qt\_frame\_view.QtFrameView attribute)}

\begin{fulllineitems}
\phantomsection\label{\detokenize{xanespy:xanespy.qt_frame_view.QtFrameView.window}}\pysigline{\sphinxbfcode{window}\sphinxstrong{ = None}}
\end{fulllineitems}


\end{fulllineitems}



\subsection{xanespy.qt\_frameset\_presenter module}
\label{\detokenize{xanespy:xanespy-qt-frameset-presenter-module}}\label{\detokenize{xanespy:module-xanespy.qt_frameset_presenter}}\index{xanespy.qt\_frameset\_presenter (module)}\index{QtFramesetPresenter (class in xanespy.qt\_frameset\_presenter)}

\begin{fulllineitems}
\phantomsection\label{\detokenize{xanespy:xanespy.qt_frameset_presenter.QtFramesetPresenter}}\pysiglinewithargsret{\sphinxstrong{class }\sphinxcode{xanespy.qt\_frameset\_presenter.}\sphinxbfcode{QtFramesetPresenter}}{\emph{frameset}, \emph{frame\_view}, \emph{*args}, \emph{**kwargs}}{}
Bases: \sphinxcode{PyQt5.QtCore.QObject}

Presenter for showing XanesFrameset frames and maps via Qt.
\index{app\_ready (xanespy.qt\_frameset\_presenter.QtFramesetPresenter attribute)}

\begin{fulllineitems}
\phantomsection\label{\detokenize{xanespy:xanespy.qt_frameset_presenter.QtFramesetPresenter.app_ready}}\pysigline{\sphinxbfcode{app\_ready}}
\sphinxstyleemphasis{pyqtSignal} \textendash{} Emitted when the application has been created and is ready for drawing.

\end{fulllineitems}

\index{busy\_status\_changed (xanespy.qt\_frameset\_presenter.QtFramesetPresenter attribute)}

\begin{fulllineitems}
\phantomsection\label{\detokenize{xanespy:xanespy.qt_frameset_presenter.QtFramesetPresenter.busy_status_changed}}\pysigline{\sphinxbfcode{busy\_status\_changed}}
\sphinxstyleemphasis{pyqtSignal} \textendash{} Emitted when processing has started or ended.

\end{fulllineitems}

\index{map\_data\_changed (xanespy.qt\_frameset\_presenter.QtFramesetPresenter attribute)}

\begin{fulllineitems}
\phantomsection\label{\detokenize{xanespy:xanespy.qt_frameset_presenter.QtFramesetPresenter.map_data_changed}}\pysigline{\sphinxbfcode{map\_data\_changed}}
\sphinxstyleemphasis{pyqtSignal} \textendash{} Emitted when the map data is different and should be re-plotted.

\end{fulllineitems}

\index{map\_data\_cleared (xanespy.qt\_frameset\_presenter.QtFramesetPresenter attribute)}

\begin{fulllineitems}
\phantomsection\label{\detokenize{xanespy:xanespy.qt_frameset_presenter.QtFramesetPresenter.map_data_cleared}}\pysigline{\sphinxbfcode{map\_data\_cleared}}
\sphinxstyleemphasis{pyqtSignal} \textendash{} Emitted when no map data is available and plots can be cleared.

\end{fulllineitems}

\index{active\_component (xanespy.qt\_frameset\_presenter.QtFramesetPresenter attribute)}

\begin{fulllineitems}
\phantomsection\label{\detokenize{xanespy:xanespy.qt_frameset_presenter.QtFramesetPresenter.active_component}}\pysigline{\sphinxbfcode{active\_component}\sphinxstrong{ = ‘real’}}
\end{fulllineitems}

\index{active\_frame (xanespy.qt\_frameset\_presenter.QtFramesetPresenter attribute)}

\begin{fulllineitems}
\phantomsection\label{\detokenize{xanespy:xanespy.qt_frameset_presenter.QtFramesetPresenter.active_frame}}\pysigline{\sphinxbfcode{active\_frame}\sphinxstrong{ = 0}}
\end{fulllineitems}

\index{active\_frames() (xanespy.qt\_frameset\_presenter.QtFramesetPresenter method)}

\begin{fulllineitems}
\phantomsection\label{\detokenize{xanespy:xanespy.qt_frameset_presenter.QtFramesetPresenter.active_frames}}\pysiglinewithargsret{\sphinxbfcode{active\_frames}}{}{}
\end{fulllineitems}

\index{active\_map() (xanespy.qt\_frameset\_presenter.QtFramesetPresenter method)}

\begin{fulllineitems}
\phantomsection\label{\detokenize{xanespy:xanespy.qt_frameset_presenter.QtFramesetPresenter.active_map}}\pysiglinewithargsret{\sphinxbfcode{active\_map}}{}{}
Returns the active map array if possible. If it doesn’t exist,
return None.

\end{fulllineitems}

\index{active\_representation (xanespy.qt\_frameset\_presenter.QtFramesetPresenter attribute)}

\begin{fulllineitems}
\phantomsection\label{\detokenize{xanespy:xanespy.qt_frameset_presenter.QtFramesetPresenter.active_representation}}\pysigline{\sphinxbfcode{active\_representation}\sphinxstrong{ = None}}
\end{fulllineitems}

\index{active\_timestep (xanespy.qt\_frameset\_presenter.QtFramesetPresenter attribute)}

\begin{fulllineitems}
\phantomsection\label{\detokenize{xanespy:xanespy.qt_frameset_presenter.QtFramesetPresenter.active_timestep}}\pysigline{\sphinxbfcode{active\_timestep}\sphinxstrong{ = 0}}
\end{fulllineitems}

\index{add\_frame\_view() (xanespy.qt\_frameset\_presenter.QtFramesetPresenter method)}

\begin{fulllineitems}
\phantomsection\label{\detokenize{xanespy:xanespy.qt_frameset_presenter.QtFramesetPresenter.add_frame_view}}\pysiglinewithargsret{\sphinxbfcode{add\_frame\_view}}{\emph{view}, \emph{threaded=True}}{}
Attach a view to this presenter.
\begin{quote}\begin{description}
\item[{Parameters}] \leavevmode\begin{itemize}
\item {} 
\sphinxstyleliteralstrong{view} (\sphinxstyleliteralemphasis{QObject}) \textendash{} The view will be connected to signals that describe changes in
frame data.

\item {} 
\sphinxstyleliteralstrong{threaded} (\sphinxstyleliteralemphasis{bool}\sphinxstyleliteralemphasis{, }\sphinxstyleliteralemphasis{optional}) \textendash{} If true, this view will be added to its own thread before
signals get connected

\end{itemize}

\end{description}\end{quote}

\end{fulllineitems}

\index{add\_map\_view() (xanespy.qt\_frameset\_presenter.QtFramesetPresenter method)}

\begin{fulllineitems}
\phantomsection\label{\detokenize{xanespy:xanespy.qt_frameset_presenter.QtFramesetPresenter.add_map_view}}\pysiglinewithargsret{\sphinxbfcode{add\_map\_view}}{\emph{view}, \emph{threaded=True}}{}
Attach a view to this presenter.
\begin{quote}\begin{description}
\item[{Parameters}] \leavevmode\begin{itemize}
\item {} 
\sphinxstyleliteralstrong{view} (\sphinxstyleliteralemphasis{QObject}) \textendash{} The view will be connected to signals that describe changes in
map data.

\item {} 
\sphinxstyleliteralstrong{threaded} (\sphinxstyleliteralemphasis{bool}\sphinxstyleliteralemphasis{, }\sphinxstyleliteralemphasis{optional}) \textendash{} If true, this view will be added to its own thread before
signals get connected, giving a snapier UI. Disabling makes
testing more straightforward.

\end{itemize}

\end{description}\end{quote}

\end{fulllineitems}

\index{animate\_frames() (xanespy.qt\_frameset\_presenter.QtFramesetPresenter method)}

\begin{fulllineitems}
\phantomsection\label{\detokenize{xanespy:xanespy.qt_frameset_presenter.QtFramesetPresenter.animate_frames}}\pysiglinewithargsret{\sphinxbfcode{animate\_frames}}{}{}
\end{fulllineitems}

\index{app\_ready (xanespy.qt\_frameset\_presenter.QtFramesetPresenter attribute)}

\begin{fulllineitems}
\pysigline{\sphinxbfcode{app\_ready}}
\end{fulllineitems}

\index{build\_hdf\_tree() (xanespy.qt\_frameset\_presenter.QtFramesetPresenter method)}

\begin{fulllineitems}
\phantomsection\label{\detokenize{xanespy:xanespy.qt_frameset_presenter.QtFramesetPresenter.build_hdf_tree}}\pysiglinewithargsret{\sphinxbfcode{build\_hdf\_tree}}{}{}
Build the items and insert them into the view’s HDF tree based on
the structure of the frameset’s HDF file.

\end{fulllineitems}

\index{busy\_status\_changed (xanespy.qt\_frameset\_presenter.QtFramesetPresenter attribute)}

\begin{fulllineitems}
\pysigline{\sphinxbfcode{busy\_status\_changed}}
\end{fulllineitems}

\index{change\_cmap() (xanespy.qt\_frameset\_presenter.QtFramesetPresenter method)}

\begin{fulllineitems}
\phantomsection\label{\detokenize{xanespy:xanespy.qt_frameset_presenter.QtFramesetPresenter.change_cmap}}\pysiglinewithargsret{\sphinxbfcode{change\_cmap}}{\emph{new\_cmap}}{}
\end{fulllineitems}

\index{change\_component() (xanespy.qt\_frameset\_presenter.QtFramesetPresenter method)}

\begin{fulllineitems}
\phantomsection\label{\detokenize{xanespy:xanespy.qt_frameset_presenter.QtFramesetPresenter.change_component}}\pysiglinewithargsret{\sphinxbfcode{change\_component}}{\emph{new\_comp}}{}
\end{fulllineitems}

\index{change\_hdf\_group() (xanespy.qt\_frameset\_presenter.QtFramesetPresenter method)}

\begin{fulllineitems}
\phantomsection\label{\detokenize{xanespy:xanespy.qt_frameset_presenter.QtFramesetPresenter.change_hdf_group}}\pysiglinewithargsret{\sphinxbfcode{change\_hdf\_group}}{\emph{new\_item}, \emph{old\_item}}{}
\end{fulllineitems}

\index{change\_map\_cmap() (xanespy.qt\_frameset\_presenter.QtFramesetPresenter method)}

\begin{fulllineitems}
\phantomsection\label{\detokenize{xanespy:xanespy.qt_frameset_presenter.QtFramesetPresenter.change_map_cmap}}\pysiglinewithargsret{\sphinxbfcode{change\_map\_cmap}}{\emph{new\_cmap}}{}
\end{fulllineitems}

\index{cmap\_list\_changed (xanespy.qt\_frameset\_presenter.QtFramesetPresenter attribute)}

\begin{fulllineitems}
\phantomsection\label{\detokenize{xanespy:xanespy.qt_frameset_presenter.QtFramesetPresenter.cmap_list_changed}}\pysigline{\sphinxbfcode{cmap\_list\_changed}}
\end{fulllineitems}

\index{component\_list\_changed (xanespy.qt\_frameset\_presenter.QtFramesetPresenter attribute)}

\begin{fulllineitems}
\phantomsection\label{\detokenize{xanespy:xanespy.qt_frameset_presenter.QtFramesetPresenter.component_list_changed}}\pysigline{\sphinxbfcode{component\_list\_changed}}
\end{fulllineitems}

\index{connect\_signals() (xanespy.qt\_frameset\_presenter.QtFramesetPresenter method)}

\begin{fulllineitems}
\phantomsection\label{\detokenize{xanespy:xanespy.qt_frameset_presenter.QtFramesetPresenter.connect_signals}}\pysiglinewithargsret{\sphinxbfcode{connect\_signals}}{}{}
\end{fulllineitems}

\index{create\_app() (xanespy.qt\_frameset\_presenter.QtFramesetPresenter method)}

\begin{fulllineitems}
\phantomsection\label{\detokenize{xanespy:xanespy.qt_frameset_presenter.QtFramesetPresenter.create_app}}\pysiglinewithargsret{\sphinxbfcode{create\_app}}{}{}
\end{fulllineitems}

\index{draw\_frame\_histogram() (xanespy.qt\_frameset\_presenter.QtFramesetPresenter method)}

\begin{fulllineitems}
\phantomsection\label{\detokenize{xanespy:xanespy.qt_frameset_presenter.QtFramesetPresenter.draw_frame_histogram}}\pysiglinewithargsret{\sphinxbfcode{draw\_frame\_histogram}}{}{}
\end{fulllineitems}

\index{draw\_frame\_spectra() (xanespy.qt\_frameset\_presenter.QtFramesetPresenter method)}

\begin{fulllineitems}
\phantomsection\label{\detokenize{xanespy:xanespy.qt_frameset_presenter.QtFramesetPresenter.draw_frame_spectra}}\pysiglinewithargsret{\sphinxbfcode{draw\_frame\_spectra}}{}{}
\end{fulllineitems}

\index{first\_frame() (xanespy.qt\_frameset\_presenter.QtFramesetPresenter method)}

\begin{fulllineitems}
\phantomsection\label{\detokenize{xanespy:xanespy.qt_frameset_presenter.QtFramesetPresenter.first_frame}}\pysiglinewithargsret{\sphinxbfcode{first\_frame}}{}{}
\end{fulllineitems}

\index{frame\_cmap (xanespy.qt\_frameset\_presenter.QtFramesetPresenter attribute)}

\begin{fulllineitems}
\phantomsection\label{\detokenize{xanespy:xanespy.qt_frameset_presenter.QtFramesetPresenter.frame_cmap}}\pysigline{\sphinxbfcode{frame\_cmap}\sphinxstrong{ = ‘copper’}}
\end{fulllineitems}

\index{frame\_norm() (xanespy.qt\_frameset\_presenter.QtFramesetPresenter method)}

\begin{fulllineitems}
\phantomsection\label{\detokenize{xanespy:xanespy.qt_frameset_presenter.QtFramesetPresenter.frame_norm}}\pysiglinewithargsret{\sphinxbfcode{frame\_norm}}{}{}
\end{fulllineitems}

\index{frame\_pixel (xanespy.qt\_frameset\_presenter.QtFramesetPresenter attribute)}

\begin{fulllineitems}
\phantomsection\label{\detokenize{xanespy:xanespy.qt_frameset_presenter.QtFramesetPresenter.frame_pixel}}\pysigline{\sphinxbfcode{frame\_pixel}\sphinxstrong{ = None}}
\end{fulllineitems}

\index{hover\_frame\_pixel() (xanespy.qt\_frameset\_presenter.QtFramesetPresenter method)}

\begin{fulllineitems}
\phantomsection\label{\detokenize{xanespy:xanespy.qt_frameset_presenter.QtFramesetPresenter.hover_frame_pixel}}\pysiglinewithargsret{\sphinxbfcode{hover\_frame\_pixel}}{\emph{xy}}{}
\end{fulllineitems}

\index{last\_frame() (xanespy.qt\_frameset\_presenter.QtFramesetPresenter method)}

\begin{fulllineitems}
\phantomsection\label{\detokenize{xanespy:xanespy.qt_frameset_presenter.QtFramesetPresenter.last_frame}}\pysiglinewithargsret{\sphinxbfcode{last\_frame}}{}{}
\end{fulllineitems}

\index{launch() (xanespy.qt\_frameset\_presenter.QtFramesetPresenter method)}

\begin{fulllineitems}
\phantomsection\label{\detokenize{xanespy:xanespy.qt_frameset_presenter.QtFramesetPresenter.launch}}\pysiglinewithargsret{\sphinxbfcode{launch}}{}{}
\end{fulllineitems}

\index{map\_cmap (xanespy.qt\_frameset\_presenter.QtFramesetPresenter attribute)}

\begin{fulllineitems}
\phantomsection\label{\detokenize{xanespy:xanespy.qt_frameset_presenter.QtFramesetPresenter.map_cmap}}\pysigline{\sphinxbfcode{map\_cmap}\sphinxstrong{ = ‘plasma’}}
\end{fulllineitems}

\index{map\_cursor\_changed (xanespy.qt\_frameset\_presenter.QtFramesetPresenter attribute)}

\begin{fulllineitems}
\phantomsection\label{\detokenize{xanespy:xanespy.qt_frameset_presenter.QtFramesetPresenter.map_cursor_changed}}\pysigline{\sphinxbfcode{map\_cursor\_changed}}
\end{fulllineitems}

\index{map\_data\_changed (xanespy.qt\_frameset\_presenter.QtFramesetPresenter attribute)}

\begin{fulllineitems}
\pysigline{\sphinxbfcode{map\_data\_changed}}
\end{fulllineitems}

\index{map\_data\_cleared (xanespy.qt\_frameset\_presenter.QtFramesetPresenter attribute)}

\begin{fulllineitems}
\pysigline{\sphinxbfcode{map\_data\_cleared}}
\end{fulllineitems}

\index{map\_limits\_changed (xanespy.qt\_frameset\_presenter.QtFramesetPresenter attribute)}

\begin{fulllineitems}
\phantomsection\label{\detokenize{xanespy:xanespy.qt_frameset_presenter.QtFramesetPresenter.map_limits_changed}}\pysigline{\sphinxbfcode{map\_limits\_changed}}
\end{fulllineitems}

\index{map\_norm() (xanespy.qt\_frameset\_presenter.QtFramesetPresenter method)}

\begin{fulllineitems}
\phantomsection\label{\detokenize{xanespy:xanespy.qt_frameset_presenter.QtFramesetPresenter.map_norm}}\pysiglinewithargsret{\sphinxbfcode{map\_norm}}{}{}
\end{fulllineitems}

\index{map\_pixel\_changed (xanespy.qt\_frameset\_presenter.QtFramesetPresenter attribute)}

\begin{fulllineitems}
\phantomsection\label{\detokenize{xanespy:xanespy.qt_frameset_presenter.QtFramesetPresenter.map_pixel_changed}}\pysigline{\sphinxbfcode{map\_pixel\_changed}}
\end{fulllineitems}

\index{map\_spectrum\_changed (xanespy.qt\_frameset\_presenter.QtFramesetPresenter attribute)}

\begin{fulllineitems}
\phantomsection\label{\detokenize{xanespy:xanespy.qt_frameset_presenter.QtFramesetPresenter.map_spectrum_changed}}\pysigline{\sphinxbfcode{map\_spectrum\_changed}}
\end{fulllineitems}

\index{mean\_spectrum\_changed (xanespy.qt\_frameset\_presenter.QtFramesetPresenter attribute)}

\begin{fulllineitems}
\phantomsection\label{\detokenize{xanespy:xanespy.qt_frameset_presenter.QtFramesetPresenter.mean_spectrum_changed}}\pysigline{\sphinxbfcode{mean\_spectrum\_changed}}
\end{fulllineitems}

\index{move\_map\_pixel() (xanespy.qt\_frameset\_presenter.QtFramesetPresenter method)}

\begin{fulllineitems}
\phantomsection\label{\detokenize{xanespy:xanespy.qt_frameset_presenter.QtFramesetPresenter.move_map_pixel}}\pysiglinewithargsret{\sphinxbfcode{move\_map\_pixel}}{\emph{vert}, \emph{horiz}}{}
Move the active pixel by the given amount in each direction.

If the current pixel is not active, this is a no-op.

\end{fulllineitems}

\index{move\_slider() (xanespy.qt\_frameset\_presenter.QtFramesetPresenter method)}

\begin{fulllineitems}
\phantomsection\label{\detokenize{xanespy:xanespy.qt_frameset_presenter.QtFramesetPresenter.move_slider}}\pysiglinewithargsret{\sphinxbfcode{move\_slider}}{\emph{new\_value}}{}
\end{fulllineitems}

\index{next\_frame() (xanespy.qt\_frameset\_presenter.QtFramesetPresenter method)}

\begin{fulllineitems}
\phantomsection\label{\detokenize{xanespy:xanespy.qt_frameset_presenter.QtFramesetPresenter.next_frame}}\pysiglinewithargsret{\sphinxbfcode{next\_frame}}{}{}
\end{fulllineitems}

\index{num\_frames (xanespy.qt\_frameset\_presenter.QtFramesetPresenter attribute)}

\begin{fulllineitems}
\phantomsection\label{\detokenize{xanespy:xanespy.qt_frameset_presenter.QtFramesetPresenter.num_frames}}\pysigline{\sphinxbfcode{num\_frames}\sphinxstrong{ = 0}}
\end{fulllineitems}

\index{play\_frames() (xanespy.qt\_frameset\_presenter.QtFramesetPresenter method)}

\begin{fulllineitems}
\phantomsection\label{\detokenize{xanespy:xanespy.qt_frameset_presenter.QtFramesetPresenter.play_frames}}\pysiglinewithargsret{\sphinxbfcode{play\_frames}}{\emph{start}}{}
\end{fulllineitems}

\index{prepare\_ui() (xanespy.qt\_frameset\_presenter.QtFramesetPresenter method)}

\begin{fulllineitems}
\phantomsection\label{\detokenize{xanespy:xanespy.qt_frameset_presenter.QtFramesetPresenter.prepare_ui}}\pysiglinewithargsret{\sphinxbfcode{prepare\_ui}}{}{}
\end{fulllineitems}

\index{previous\_frame() (xanespy.qt\_frameset\_presenter.QtFramesetPresenter method)}

\begin{fulllineitems}
\phantomsection\label{\detokenize{xanespy:xanespy.qt_frameset_presenter.QtFramesetPresenter.previous_frame}}\pysiglinewithargsret{\sphinxbfcode{previous\_frame}}{}{}
\end{fulllineitems}

\index{process\_events (xanespy.qt\_frameset\_presenter.QtFramesetPresenter attribute)}

\begin{fulllineitems}
\phantomsection\label{\detokenize{xanespy:xanespy.qt_frameset_presenter.QtFramesetPresenter.process_events}}\pysigline{\sphinxbfcode{process\_events}}
\end{fulllineitems}

\index{refresh\_frames() (xanespy.qt\_frameset\_presenter.QtFramesetPresenter method)}

\begin{fulllineitems}
\phantomsection\label{\detokenize{xanespy:xanespy.qt_frameset_presenter.QtFramesetPresenter.refresh_frames}}\pysiglinewithargsret{\sphinxbfcode{refresh\_frames}}{}{}
\end{fulllineitems}

\index{reset\_frame\_range() (xanespy.qt\_frameset\_presenter.QtFramesetPresenter method)}

\begin{fulllineitems}
\phantomsection\label{\detokenize{xanespy:xanespy.qt_frameset_presenter.QtFramesetPresenter.reset_frame_range}}\pysiglinewithargsret{\sphinxbfcode{reset\_frame\_range}}{}{}
Reset the frame plotting vmin and vmax based on the currently
selected data. VMin will be the 1nd percentile and VMax will
be the 99th percentile. The results can be accessed by calling
the \sphinxtitleref{frame\_norm()} method.
\begin{quote}\begin{description}
\item[{Returns}] \leavevmode
\begin{itemize}
\item {} 
\sphinxstylestrong{(vmin, vmax)} (\sphinxstyleemphasis{(float, float)}) \textendash{} A 2-tuple with the new min and max range.

\item {} 
\sphinxstyleemphasis{None} \textendash{} If current representation is not a valid map, None will be
returned.

\end{itemize}


\end{description}\end{quote}

\end{fulllineitems}

\index{reset\_map\_range() (xanespy.qt\_frameset\_presenter.QtFramesetPresenter method)}

\begin{fulllineitems}
\phantomsection\label{\detokenize{xanespy:xanespy.qt_frameset_presenter.QtFramesetPresenter.reset_map_range}}\pysiglinewithargsret{\sphinxbfcode{reset\_map\_range}}{}{}
Reset the map plotting vmin and vmax based on the currently
selected data. VMin will be the 1st percentile and VMax will
be the 99th percentile. This method also caches the values so
they can be retrieved via the \sphinxtitleref{map\_norm()} method. If the
current representation is not valid map data, then the cached
vmin and vmax will be returned with no other changes.
\begin{quote}\begin{description}
\item[{Returns}] \leavevmode
\sphinxstylestrong{(vmin, vmax)} \textendash{} A 2-tuple with the current min and max range.

\item[{Return type}] \leavevmode
(float, float)

\end{description}\end{quote}

\end{fulllineitems}

\index{set\_frame\_vmax() (xanespy.qt\_frameset\_presenter.QtFramesetPresenter method)}

\begin{fulllineitems}
\phantomsection\label{\detokenize{xanespy:xanespy.qt_frameset_presenter.QtFramesetPresenter.set_frame_vmax}}\pysiglinewithargsret{\sphinxbfcode{set\_frame\_vmax}}{\emph{new\_value}}{}
\end{fulllineitems}

\index{set\_frame\_vmin() (xanespy.qt\_frameset\_presenter.QtFramesetPresenter method)}

\begin{fulllineitems}
\phantomsection\label{\detokenize{xanespy:xanespy.qt_frameset_presenter.QtFramesetPresenter.set_frame_vmin}}\pysiglinewithargsret{\sphinxbfcode{set\_frame\_vmin}}{\emph{new\_value}}{}
\end{fulllineitems}

\index{set\_map\_cursor() (xanespy.qt\_frameset\_presenter.QtFramesetPresenter method)}

\begin{fulllineitems}
\phantomsection\label{\detokenize{xanespy:xanespy.qt_frameset_presenter.QtFramesetPresenter.set_map_cursor}}\pysiglinewithargsret{\sphinxbfcode{set\_map\_cursor}}{\emph{x}, \emph{y}}{}
\end{fulllineitems}

\index{set\_map\_pixel() (xanespy.qt\_frameset\_presenter.QtFramesetPresenter method)}

\begin{fulllineitems}
\phantomsection\label{\detokenize{xanespy:xanespy.qt_frameset_presenter.QtFramesetPresenter.set_map_pixel}}\pysiglinewithargsret{\sphinxbfcode{set\_map\_pixel}}{\emph{x}, \emph{y}}{}
\end{fulllineitems}

\index{set\_map\_vmax() (xanespy.qt\_frameset\_presenter.QtFramesetPresenter method)}

\begin{fulllineitems}
\phantomsection\label{\detokenize{xanespy:xanespy.qt_frameset_presenter.QtFramesetPresenter.set_map_vmax}}\pysiglinewithargsret{\sphinxbfcode{set\_map\_vmax}}{\emph{new\_value}}{}
\end{fulllineitems}

\index{set\_map\_vmin() (xanespy.qt\_frameset\_presenter.QtFramesetPresenter method)}

\begin{fulllineitems}
\phantomsection\label{\detokenize{xanespy:xanespy.qt_frameset_presenter.QtFramesetPresenter.set_map_vmin}}\pysiglinewithargsret{\sphinxbfcode{set\_map\_vmin}}{\emph{new\_value}}{}
\end{fulllineitems}

\index{set\_play\_speed() (xanespy.qt\_frameset\_presenter.QtFramesetPresenter method)}

\begin{fulllineitems}
\phantomsection\label{\detokenize{xanespy:xanespy.qt_frameset_presenter.QtFramesetPresenter.set_play_speed}}\pysiglinewithargsret{\sphinxbfcode{set\_play\_speed}}{\emph{new\_speed}}{}
Change how fast the play timer ticks. Input speeds should be in the
range of 0, 30 with 30 being the fastest. This is converted to timer
intervals between 1ms and 1000ms on a exponential scale.

\end{fulllineitems}

\index{set\_timestep() (xanespy.qt\_frameset\_presenter.QtFramesetPresenter method)}

\begin{fulllineitems}
\phantomsection\label{\detokenize{xanespy:xanespy.qt_frameset_presenter.QtFramesetPresenter.set_timestep}}\pysiglinewithargsret{\sphinxbfcode{set\_timestep}}{\emph{new\_timestep}}{}
\end{fulllineitems}

\index{show\_spectrum\_fit (xanespy.qt\_frameset\_presenter.QtFramesetPresenter attribute)}

\begin{fulllineitems}
\phantomsection\label{\detokenize{xanespy:xanespy.qt_frameset_presenter.QtFramesetPresenter.show_spectrum_fit}}\pysigline{\sphinxbfcode{show\_spectrum\_fit}\sphinxstrong{ = False}}
\end{fulllineitems}

\index{timer (xanespy.qt\_frameset\_presenter.QtFramesetPresenter attribute)}

\begin{fulllineitems}
\phantomsection\label{\detokenize{xanespy:xanespy.qt_frameset_presenter.QtFramesetPresenter.timer}}\pysigline{\sphinxbfcode{timer}\sphinxstrong{ = None}}
\end{fulllineitems}

\index{toggle\_edge\_mask() (xanespy.qt\_frameset\_presenter.QtFramesetPresenter method)}

\begin{fulllineitems}
\phantomsection\label{\detokenize{xanespy:xanespy.qt_frameset_presenter.QtFramesetPresenter.toggle_edge_mask}}\pysiglinewithargsret{\sphinxbfcode{toggle\_edge\_mask}}{\emph{state}}{}
\end{fulllineitems}

\index{toggle\_spectrum\_fit() (xanespy.qt\_frameset\_presenter.QtFramesetPresenter method)}

\begin{fulllineitems}
\phantomsection\label{\detokenize{xanespy:xanespy.qt_frameset_presenter.QtFramesetPresenter.toggle_spectrum_fit}}\pysiglinewithargsret{\sphinxbfcode{toggle\_spectrum\_fit}}{\emph{state}}{}
\end{fulllineitems}

\index{update\_frame\_range\_limits() (xanespy.qt\_frameset\_presenter.QtFramesetPresenter method)}

\begin{fulllineitems}
\phantomsection\label{\detokenize{xanespy:xanespy.qt_frameset_presenter.QtFramesetPresenter.update_frame_range_limits}}\pysiglinewithargsret{\sphinxbfcode{update\_frame\_range\_limits}}{}{}
Check that the (min, max, step) of the frame spinboxes are
reasonable. This should be called when the spinbox values change.

\end{fulllineitems}

\index{update\_map\_limits() (xanespy.qt\_frameset\_presenter.QtFramesetPresenter method)}

\begin{fulllineitems}
\phantomsection\label{\detokenize{xanespy:xanespy.qt_frameset_presenter.QtFramesetPresenter.update_map_limits}}\pysiglinewithargsret{\sphinxbfcode{update\_map\_limits}}{}{}
\end{fulllineitems}

\index{update\_maps() (xanespy.qt\_frameset\_presenter.QtFramesetPresenter method)}

\begin{fulllineitems}
\phantomsection\label{\detokenize{xanespy:xanespy.qt_frameset_presenter.QtFramesetPresenter.update_maps}}\pysiglinewithargsret{\sphinxbfcode{update\_maps}}{}{}
Send the current mapping data to the \sphinxtitleref{map\_data\_changed} signal.

This method should be called after anything changes the visual
representation of the map. Examples:
\begin{itemize}
\item {} 
Changing the active representation

\item {} 
Changing the colormap

\item {} 
Changing the map normalization limits

\item {} 
Changing whether the XAS edge mask is applied

\end{itemize}

\end{fulllineitems}

\index{update\_spectra() (xanespy.qt\_frameset\_presenter.QtFramesetPresenter method)}

\begin{fulllineitems}
\phantomsection\label{\detokenize{xanespy:xanespy.qt_frameset_presenter.QtFramesetPresenter.update_spectra}}\pysiglinewithargsret{\sphinxbfcode{update\_spectra}}{}{}
Get the most recent data for mean and single-pixel spectra and send
them out to the signals \sphinxtitleref{mean\_spectrum\_changed} and
\sphinxtitleref{pixel\_sepctrum\_changed}.

\end{fulllineitems}

\index{update\_status\_frame() (xanespy.qt\_frameset\_presenter.QtFramesetPresenter method)}

\begin{fulllineitems}
\phantomsection\label{\detokenize{xanespy:xanespy.qt_frameset_presenter.QtFramesetPresenter.update_status_frame}}\pysiglinewithargsret{\sphinxbfcode{update\_status\_frame}}{\emph{new\_frame}}{}
Create a string (and send it to the UI) that indicates the energy
of the requested frame.

\end{fulllineitems}

\index{update\_status\_shape() (xanespy.qt\_frameset\_presenter.QtFramesetPresenter method)}

\begin{fulllineitems}
\phantomsection\label{\detokenize{xanespy:xanespy.qt_frameset_presenter.QtFramesetPresenter.update_status_shape}}\pysiglinewithargsret{\sphinxbfcode{update\_status\_shape}}{}{}
\end{fulllineitems}

\index{update\_status\_unit() (xanespy.qt\_frameset\_presenter.QtFramesetPresenter method)}

\begin{fulllineitems}
\phantomsection\label{\detokenize{xanespy:xanespy.qt_frameset_presenter.QtFramesetPresenter.update_status_unit}}\pysiglinewithargsret{\sphinxbfcode{update\_status\_unit}}{}{}
\end{fulllineitems}

\index{update\_status\_value() (xanespy.qt\_frameset\_presenter.QtFramesetPresenter method)}

\begin{fulllineitems}
\phantomsection\label{\detokenize{xanespy:xanespy.qt_frameset_presenter.QtFramesetPresenter.update_status_value}}\pysiglinewithargsret{\sphinxbfcode{update\_status\_value}}{}{}
\end{fulllineitems}

\index{use\_edge\_mask (xanespy.qt\_frameset\_presenter.QtFramesetPresenter attribute)}

\begin{fulllineitems}
\phantomsection\label{\detokenize{xanespy:xanespy.qt_frameset_presenter.QtFramesetPresenter.use_edge_mask}}\pysigline{\sphinxbfcode{use\_edge\_mask}\sphinxstrong{ = False}}
\end{fulllineitems}


\end{fulllineitems}



\subsection{xanespy.qt\_map\_view module}
\label{\detokenize{xanespy:xanespy-qt-map-view-module}}\label{\detokenize{xanespy:module-xanespy.qt_map_view}}\index{xanespy.qt\_map\_view (module)}\index{QtMapView (class in xanespy.qt\_map\_view)}

\begin{fulllineitems}
\phantomsection\label{\detokenize{xanespy:xanespy.qt_map_view.QtMapView}}\pysigline{\sphinxstrong{class }\sphinxcode{xanespy.qt\_map\_view.}\sphinxbfcode{QtMapView}}
Bases: \sphinxcode{PyQt5.QtCore.QObject}

A Qt view for a frameset map. It should be controlled by a
presenter.
\index{cmap\_changed (xanespy.qt\_map\_view.QtMapView attribute)}

\begin{fulllineitems}
\phantomsection\label{\detokenize{xanespy:xanespy.qt_map_view.QtMapView.cmap_changed}}\pysigline{\sphinxbfcode{cmap\_changed}}
\sphinxstyleemphasis{signal} \textendash{} Fires when the user changes the colormap via the UI

\end{fulllineitems}

\index{limits\_applied (xanespy.qt\_map\_view.QtMapView attribute)}

\begin{fulllineitems}
\phantomsection\label{\detokenize{xanespy:xanespy.qt_map_view.QtMapView.limits_applied}}\pysigline{\sphinxbfcode{limits\_applied}}
\sphinxstyleemphasis{signal} \textendash{} Fires when the user requests that data be redrawn with new limits.

\end{fulllineitems}

\index{limits\_reset (xanespy.qt\_map\_view.QtMapView attribute)}

\begin{fulllineitems}
\phantomsection\label{\detokenize{xanespy:xanespy.qt_map_view.QtMapView.limits_reset}}\pysigline{\sphinxbfcode{limits\_reset}}
\sphinxstyleemphasis{signal} \textendash{} Fires when the user asks that the norm limits be reset to the data.

\end{fulllineitems}

\index{cmap\_changed (xanespy.qt\_map\_view.QtMapView attribute)}

\begin{fulllineitems}
\pysigline{\sphinxbfcode{cmap\_changed}}
\end{fulllineitems}

\index{connect\_presenter() (xanespy.qt\_map\_view.QtMapView method)}

\begin{fulllineitems}
\phantomsection\label{\detokenize{xanespy:xanespy.qt_map_view.QtMapView.connect_presenter}}\pysiglinewithargsret{\sphinxbfcode{connect\_presenter}}{\emph{presenter}}{}
Connect to signals for changed presenter state.

\end{fulllineitems}

\index{create\_canvas() (xanespy.qt\_map\_view.QtMapView method)}

\begin{fulllineitems}
\phantomsection\label{\detokenize{xanespy:xanespy.qt_map_view.QtMapView.create_canvas}}\pysiglinewithargsret{\sphinxbfcode{create\_canvas}}{}{}
\end{fulllineitems}

\index{crosshairs (xanespy.qt\_map\_view.QtMapView attribute)}

\begin{fulllineitems}
\phantomsection\label{\detokenize{xanespy:xanespy.qt_map_view.QtMapView.crosshairs}}\pysigline{\sphinxbfcode{crosshairs}\sphinxstrong{ = None}}
\end{fulllineitems}

\index{edge\_mask\_toggled (xanespy.qt\_map\_view.QtMapView attribute)}

\begin{fulllineitems}
\phantomsection\label{\detokenize{xanespy:xanespy.qt_map_view.QtMapView.edge_mask_toggled}}\pysigline{\sphinxbfcode{edge\_mask\_toggled}}
\end{fulllineitems}

\index{fig (xanespy.qt\_map\_view.QtMapView attribute)}

\begin{fulllineitems}
\phantomsection\label{\detokenize{xanespy:xanespy.qt_map_view.QtMapView.fig}}\pysigline{\sphinxbfcode{fig}\sphinxstrong{ = None}}
\end{fulllineitems}

\index{hide() (xanespy.qt\_map\_view.QtMapView method)}

\begin{fulllineitems}
\phantomsection\label{\detokenize{xanespy:xanespy.qt_map_view.QtMapView.hide}}\pysiglinewithargsret{\sphinxbfcode{hide}}{}{}
\end{fulllineitems}

\index{keyboard\_nav() (xanespy.qt\_map\_view.QtMapView method)}

\begin{fulllineitems}
\phantomsection\label{\detokenize{xanespy:xanespy.qt_map_view.QtMapView.keyboard_nav}}\pysiglinewithargsret{\sphinxbfcode{keyboard\_nav}}{\emph{event}}{}
\end{fulllineitems}

\index{latest\_cmap (xanespy.qt\_map\_view.QtMapView attribute)}

\begin{fulllineitems}
\phantomsection\label{\detokenize{xanespy:xanespy.qt_map_view.QtMapView.latest_cmap}}\pysigline{\sphinxbfcode{latest\_cmap}\sphinxstrong{ = ‘plasma’}}
\end{fulllineitems}

\index{limits\_applied (xanespy.qt\_map\_view.QtMapView attribute)}

\begin{fulllineitems}
\pysigline{\sphinxbfcode{limits\_applied}}
\end{fulllineitems}

\index{limits\_reset (xanespy.qt\_map\_view.QtMapView attribute)}

\begin{fulllineitems}
\pysigline{\sphinxbfcode{limits\_reset}}
\end{fulllineitems}

\index{map\_clicked (xanespy.qt\_map\_view.QtMapView attribute)}

\begin{fulllineitems}
\phantomsection\label{\detokenize{xanespy:xanespy.qt_map_view.QtMapView.map_clicked}}\pysigline{\sphinxbfcode{map\_clicked}}
\end{fulllineitems}

\index{map\_hovered (xanespy.qt\_map\_view.QtMapView attribute)}

\begin{fulllineitems}
\phantomsection\label{\detokenize{xanespy:xanespy.qt_map_view.QtMapView.map_hovered}}\pysigline{\sphinxbfcode{map\_hovered}}
\end{fulllineitems}

\index{map\_moved (xanespy.qt\_map\_view.QtMapView attribute)}

\begin{fulllineitems}
\phantomsection\label{\detokenize{xanespy:xanespy.qt_map_view.QtMapView.map_moved}}\pysigline{\sphinxbfcode{map\_moved}}
\end{fulllineitems}

\index{map\_vmax\_changed (xanespy.qt\_map\_view.QtMapView attribute)}

\begin{fulllineitems}
\phantomsection\label{\detokenize{xanespy:xanespy.qt_map_view.QtMapView.map_vmax_changed}}\pysigline{\sphinxbfcode{map\_vmax\_changed}}
\end{fulllineitems}

\index{map\_vmin\_changed (xanespy.qt\_map\_view.QtMapView attribute)}

\begin{fulllineitems}
\phantomsection\label{\detokenize{xanespy:xanespy.qt_map_view.QtMapView.map_vmin_changed}}\pysigline{\sphinxbfcode{map\_vmin\_changed}}
\end{fulllineitems}

\index{mouse\_clicked\_canvas() (xanespy.qt\_map\_view.QtMapView method)}

\begin{fulllineitems}
\phantomsection\label{\detokenize{xanespy:xanespy.qt_map_view.QtMapView.mouse_clicked_canvas}}\pysiglinewithargsret{\sphinxbfcode{mouse\_clicked\_canvas}}{\emph{mouse\_event}}{}
\end{fulllineitems}

\index{mouse\_in\_canvas() (xanespy.qt\_map\_view.QtMapView method)}

\begin{fulllineitems}
\phantomsection\label{\detokenize{xanespy:xanespy.qt_map_view.QtMapView.mouse_in_canvas}}\pysiglinewithargsret{\sphinxbfcode{mouse\_in\_canvas}}{\emph{mouse\_event}}{}
\end{fulllineitems}

\index{plot\_histogram\_data() (xanespy.qt\_map\_view.QtMapView method)}

\begin{fulllineitems}
\phantomsection\label{\detokenize{xanespy:xanespy.qt_map_view.QtMapView.plot_histogram_data}}\pysiglinewithargsret{\sphinxbfcode{plot\_histogram\_data}}{\emph{map\_data}, \emph{norm}, \emph{cmap}, \emph{extent}}{}
\end{fulllineitems}

\index{plot\_map\_data() (xanespy.qt\_map\_view.QtMapView method)}

\begin{fulllineitems}
\phantomsection\label{\detokenize{xanespy:xanespy.qt_map_view.QtMapView.plot_map_data}}\pysiglinewithargsret{\sphinxbfcode{plot\_map\_data}}{\emph{map\_data}, \emph{norm}, \emph{cmap}, \emph{extent}}{}
\end{fulllineitems}

\index{plot\_spectrum() (xanespy.qt\_map\_view.QtMapView method)}

\begin{fulllineitems}
\phantomsection\label{\detokenize{xanespy:xanespy.qt_map_view.QtMapView.plot_spectrum}}\pysiglinewithargsret{\sphinxbfcode{plot\_spectrum}}{\emph{spectrum}, \emph{fitted\_spectrum}, \emph{norm}, \emph{cmap}, \emph{edge\_range}}{}
\end{fulllineitems}

\index{redraw\_canvas() (xanespy.qt\_map\_view.QtMapView method)}

\begin{fulllineitems}
\phantomsection\label{\detokenize{xanespy:xanespy.qt_map_view.QtMapView.redraw_canvas}}\pysiglinewithargsret{\sphinxbfcode{redraw\_canvas}}{}{}
\end{fulllineitems}

\index{redraw\_crosshairs() (xanespy.qt\_map\_view.QtMapView method)}

\begin{fulllineitems}
\phantomsection\label{\detokenize{xanespy:xanespy.qt_map_view.QtMapView.redraw_crosshairs}}\pysiglinewithargsret{\sphinxbfcode{redraw\_crosshairs}}{\emph{xy}}{}
Draw a set of crosshairs on the map at location given by \sphinxtitleref{xy}.

\end{fulllineitems}

\index{set\_cmap\_list() (xanespy.qt\_map\_view.QtMapView method)}

\begin{fulllineitems}
\phantomsection\label{\detokenize{xanespy:xanespy.qt_map_view.QtMapView.set_cmap_list}}\pysiglinewithargsret{\sphinxbfcode{set\_cmap\_list}}{\emph{new\_list}}{}
\end{fulllineitems}

\index{set\_map\_limits() (xanespy.qt\_map\_view.QtMapView method)}

\begin{fulllineitems}
\phantomsection\label{\detokenize{xanespy:xanespy.qt_map_view.QtMapView.set_map_limits}}\pysiglinewithargsret{\sphinxbfcode{set\_map\_limits}}{\emph{vmin}, \emph{vmax}, \emph{step}, \emph{decimals}}{}
\end{fulllineitems}

\index{setup\_ui() (xanespy.qt\_map\_view.QtMapView method)}

\begin{fulllineitems}
\phantomsection\label{\detokenize{xanespy:xanespy.qt_map_view.QtMapView.setup_ui}}\pysiglinewithargsret{\sphinxbfcode{setup\_ui}}{}{}
\end{fulllineitems}

\index{show() (xanespy.qt\_map\_view.QtMapView method)}

\begin{fulllineitems}
\phantomsection\label{\detokenize{xanespy:xanespy.qt_map_view.QtMapView.show}}\pysiglinewithargsret{\sphinxbfcode{show}}{}{}
\end{fulllineitems}

\index{spectrum\_fit\_toggled (xanespy.qt\_map\_view.QtMapView attribute)}

\begin{fulllineitems}
\phantomsection\label{\detokenize{xanespy:xanespy.qt_map_view.QtMapView.spectrum_fit_toggled}}\pysigline{\sphinxbfcode{spectrum\_fit\_toggled}}
\end{fulllineitems}

\index{ui (xanespy.qt\_map\_view.QtMapView attribute)}

\begin{fulllineitems}
\phantomsection\label{\detokenize{xanespy:xanespy.qt_map_view.QtMapView.ui}}\pysigline{\sphinxbfcode{ui}\sphinxstrong{ = None}}
\end{fulllineitems}

\index{update\_crosshair\_labels() (xanespy.qt\_map\_view.QtMapView method)}

\begin{fulllineitems}
\phantomsection\label{\detokenize{xanespy:xanespy.qt_map_view.QtMapView.update_crosshair_labels}}\pysiglinewithargsret{\sphinxbfcode{update\_crosshair\_labels}}{\emph{xy}, \emph{pixel}, \emph{value}}{}
\end{fulllineitems}

\index{update\_cursor\_labels() (xanespy.qt\_map\_view.QtMapView method)}

\begin{fulllineitems}
\phantomsection\label{\detokenize{xanespy:xanespy.qt_map_view.QtMapView.update_cursor_labels}}\pysiglinewithargsret{\sphinxbfcode{update\_cursor\_labels}}{\emph{xy}, \emph{pixel}, \emph{value}}{}
\end{fulllineitems}

\index{window (xanespy.qt\_map\_view.QtMapView attribute)}

\begin{fulllineitems}
\phantomsection\label{\detokenize{xanespy:xanespy.qt_map_view.QtMapView.window}}\pysigline{\sphinxbfcode{window}\sphinxstrong{ = None}}
\end{fulllineitems}


\end{fulllineitems}



\subsection{xanespy.txmstore module}
\label{\detokenize{xanespy:xanespy-txmstore-module}}\label{\detokenize{xanespy:module-xanespy.txmstore}}\index{xanespy.txmstore (module)}
Tools for accessing TXM data stored in an HDF5 file.
\index{TXMStore (class in xanespy.txmstore)}

\begin{fulllineitems}
\phantomsection\label{\detokenize{xanespy:xanespy.txmstore.TXMStore}}\pysiglinewithargsret{\sphinxstrong{class }\sphinxcode{xanespy.txmstore.}\sphinxbfcode{TXMStore}}{\emph{hdf\_filename: str}, \emph{parent\_name: str}, \emph{data\_name=None}, \emph{mode=’r’}}{}
Bases: \sphinxcode{object}

Wrapper around HDF5 file that stores TXM data. It has a series of
properties that return the corresponding HDF5 dataset object; the
TXMStore().attribute.value pattern can be used to get pure numpy
arrays. These objects should be used as a context manager to ensure
that the file is closed, especially if using a writing mode:
\begin{quote}
\begin{description}
\item[{with TXMStore() as store:}] \leavevmode
\# Do stuff with store here

\end{description}
\end{quote}
\begin{quote}\begin{description}
\item[{Parameters}] \leavevmode\begin{itemize}
\item {} 
\sphinxstyleliteralstrong{hdf\_filename} (\sphinxstyleliteralemphasis{str}) \textendash{} Path to the HDF file to be used.

\item {} 
\sphinxstyleliteralstrong{parent\_name} (\sphinxstyleliteralemphasis{str}) \textendash{} Name of the top-level HDF5 group.

\item {} 
\sphinxstyleliteralstrong{data\_name} (\sphinxstyleliteralemphasis{str}) \textendash{} Name of the second level HDF5 group, used for specific data
iterations (eg. imported, aligned)

\item {} 
\sphinxstyleliteralstrong{mode} (\sphinxstyleliteralemphasis{str}) \textendash{} Eg. ‘r’ for read-only, ‘r+’ for read-write. Passed directly to
h5py.File constructor.

\end{itemize}

\end{description}\end{quote}
\index{VERSION (xanespy.txmstore.TXMStore attribute)}

\begin{fulllineitems}
\phantomsection\label{\detokenize{xanespy:xanespy.txmstore.TXMStore.VERSION}}\pysigline{\sphinxbfcode{VERSION}\sphinxstrong{ = 1}}
\end{fulllineitems}

\index{absorbance\_mean (xanespy.txmstore.TXMStore attribute)}

\begin{fulllineitems}
\phantomsection\label{\detokenize{xanespy:xanespy.txmstore.TXMStore.absorbance_mean}}\pysigline{\sphinxbfcode{absorbance\_mean}}
\end{fulllineitems}

\index{absorbances (xanespy.txmstore.TXMStore attribute)}

\begin{fulllineitems}
\phantomsection\label{\detokenize{xanespy:xanespy.txmstore.TXMStore.absorbances}}\pysigline{\sphinxbfcode{absorbances}}
\end{fulllineitems}

\index{close() (xanespy.txmstore.TXMStore method)}

\begin{fulllineitems}
\phantomsection\label{\detokenize{xanespy:xanespy.txmstore.TXMStore.close}}\pysiglinewithargsret{\sphinxbfcode{close}}{}{}
\end{fulllineitems}

\index{cluster\_map (xanespy.txmstore.TXMStore attribute)}

\begin{fulllineitems}
\phantomsection\label{\detokenize{xanespy:xanespy.txmstore.TXMStore.cluster_map}}\pysigline{\sphinxbfcode{cluster\_map}}
\end{fulllineitems}

\index{data\_group() (xanespy.txmstore.TXMStore method)}

\begin{fulllineitems}
\phantomsection\label{\detokenize{xanespy:xanespy.txmstore.TXMStore.data_group}}\pysiglinewithargsret{\sphinxbfcode{data\_group}}{}{}
Retrieve the currently active second-level HDF5 group object for
this file and groupname. Ex. “imported” or “aligned\_frames”.

\end{fulllineitems}

\index{data\_name (xanespy.txmstore.TXMStore attribute)}

\begin{fulllineitems}
\phantomsection\label{\detokenize{xanespy:xanespy.txmstore.TXMStore.data_name}}\pysigline{\sphinxbfcode{data\_name}}
\end{fulllineitems}

\index{data\_tree() (xanespy.txmstore.TXMStore method)}

\begin{fulllineitems}
\phantomsection\label{\detokenize{xanespy:xanespy.txmstore.TXMStore.data_tree}}\pysiglinewithargsret{\sphinxbfcode{data\_tree}}{}{}
Create a tree of the possible groups this store could access. The
first level is samples, then data\_groups (ie. same sample but
different analysis status), then representations. Maps are not
included in this tree.

\end{fulllineitems}

\index{energies (xanespy.txmstore.TXMStore attribute)}

\begin{fulllineitems}
\phantomsection\label{\detokenize{xanespy:xanespy.txmstore.TXMStore.energies}}\pysigline{\sphinxbfcode{energies}}
\end{fulllineitems}

\index{filenames (xanespy.txmstore.TXMStore attribute)}

\begin{fulllineitems}
\phantomsection\label{\detokenize{xanespy:xanespy.txmstore.TXMStore.filenames}}\pysigline{\sphinxbfcode{filenames}}
\end{fulllineitems}

\index{fit\_parameters (xanespy.txmstore.TXMStore attribute)}

\begin{fulllineitems}
\phantomsection\label{\detokenize{xanespy:xanespy.txmstore.TXMStore.fit_parameters}}\pysigline{\sphinxbfcode{fit\_parameters}}
\end{fulllineitems}

\index{fork\_data\_group() (xanespy.txmstore.TXMStore method)}

\begin{fulllineitems}
\phantomsection\label{\detokenize{xanespy:xanespy.txmstore.TXMStore.fork_data_group}}\pysiglinewithargsret{\sphinxbfcode{fork\_data\_group}}{\emph{dest}, \emph{src=None}}{}
Turn on different active data group for this store. This method
deletes the existing group and copies symlinks from the
current one.

\end{fulllineitems}

\index{get\_dataset() (xanespy.txmstore.TXMStore method)}

\begin{fulllineitems}
\phantomsection\label{\detokenize{xanespy:xanespy.txmstore.TXMStore.get_dataset}}\pysiglinewithargsret{\sphinxbfcode{get\_dataset}}{\emph{name}}{}
Attempt to open the requested dataset.
\begin{quote}\begin{description}
\item[{Returns}] \leavevmode
\sphinxstylestrong{data} \textendash{} An open HDF5 dataset

\item[{Return type}] \leavevmode
hyp5.Dataset

\item[{Raises}] \leavevmode
\sphinxcode{exceptions.GroupKeyError} \textendash{} If the dataset does not exist in the file.

\end{description}\end{quote}

\end{fulllineitems}

\index{get\_frames() (xanespy.txmstore.TXMStore method)}

\begin{fulllineitems}
\phantomsection\label{\detokenize{xanespy:xanespy.txmstore.TXMStore.get_frames}}\pysiglinewithargsret{\sphinxbfcode{get\_frames}}{\emph{name}}{}
Get a set of frames, specified by the value of \sphinxtitleref{name}.

\end{fulllineitems}

\index{get\_map() (xanespy.txmstore.TXMStore method)}

\begin{fulllineitems}
\phantomsection\label{\detokenize{xanespy:xanespy.txmstore.TXMStore.get_map}}\pysiglinewithargsret{\sphinxbfcode{get\_map}}{\emph{name}}{}
Get a map of the frames, specified by the value of \sphinxtitleref{name}.

\end{fulllineitems}

\index{has\_dataset() (xanespy.txmstore.TXMStore method)}

\begin{fulllineitems}
\phantomsection\label{\detokenize{xanespy:xanespy.txmstore.TXMStore.has_dataset}}\pysiglinewithargsret{\sphinxbfcode{has\_dataset}}{\emph{name}}{}
Return a boolean indicated whether this dataset exists in the HDF
file.

\end{fulllineitems}

\index{intensities (xanespy.txmstore.TXMStore attribute)}

\begin{fulllineitems}
\phantomsection\label{\detokenize{xanespy:xanespy.txmstore.TXMStore.intensities}}\pysigline{\sphinxbfcode{intensities}}
\end{fulllineitems}

\index{latest\_data\_name (xanespy.txmstore.TXMStore attribute)}

\begin{fulllineitems}
\phantomsection\label{\detokenize{xanespy:xanespy.txmstore.TXMStore.latest_data_name}}\pysigline{\sphinxbfcode{latest\_data\_name}}
\end{fulllineitems}

\index{original\_positions (xanespy.txmstore.TXMStore attribute)}

\begin{fulllineitems}
\phantomsection\label{\detokenize{xanespy:xanespy.txmstore.TXMStore.original_positions}}\pysigline{\sphinxbfcode{original\_positions}}
\end{fulllineitems}

\index{parent\_group() (xanespy.txmstore.TXMStore method)}

\begin{fulllineitems}
\phantomsection\label{\detokenize{xanespy:xanespy.txmstore.TXMStore.parent_group}}\pysiglinewithargsret{\sphinxbfcode{parent\_group}}{}{}
Retrieve the top-level HDF5 group object for this file and
groupname.

\end{fulllineitems}

\index{particle\_labels (xanespy.txmstore.TXMStore attribute)}

\begin{fulllineitems}
\phantomsection\label{\detokenize{xanespy:xanespy.txmstore.TXMStore.particle_labels}}\pysigline{\sphinxbfcode{particle\_labels}}
\end{fulllineitems}

\index{pixel\_sizes (xanespy.txmstore.TXMStore attribute)}

\begin{fulllineitems}
\phantomsection\label{\detokenize{xanespy:xanespy.txmstore.TXMStore.pixel_sizes}}\pysigline{\sphinxbfcode{pixel\_sizes}}
\end{fulllineitems}

\index{pixel\_unit (xanespy.txmstore.TXMStore attribute)}

\begin{fulllineitems}
\phantomsection\label{\detokenize{xanespy:xanespy.txmstore.TXMStore.pixel_unit}}\pysigline{\sphinxbfcode{pixel\_unit}}
\end{fulllineitems}

\index{references (xanespy.txmstore.TXMStore attribute)}

\begin{fulllineitems}
\phantomsection\label{\detokenize{xanespy:xanespy.txmstore.TXMStore.references}}\pysigline{\sphinxbfcode{references}}
\end{fulllineitems}

\index{relative\_positions (xanespy.txmstore.TXMStore attribute)}

\begin{fulllineitems}
\phantomsection\label{\detokenize{xanespy:xanespy.txmstore.TXMStore.relative_positions}}\pysigline{\sphinxbfcode{relative\_positions}}
(x, y, z) position values for each frame.

\end{fulllineitems}

\index{replace\_dataset() (xanespy.txmstore.TXMStore method)}

\begin{fulllineitems}
\phantomsection\label{\detokenize{xanespy:xanespy.txmstore.TXMStore.replace_dataset}}\pysiglinewithargsret{\sphinxbfcode{replace\_dataset}}{\emph{name}, \emph{data}, \emph{context=None}, \emph{attrs=\{\}}, \emph{compression=None}, \emph{*args}, \emph{**kwargs}}{}
Wrapper for h5py.create\_dataset that removes the existing dataset
if it exists.
\begin{quote}\begin{description}
\item[{Parameters}] \leavevmode\begin{itemize}
\item {} 
\sphinxstyleliteralstrong{name} (\sphinxstyleliteralemphasis{str}) \textendash{} HDF5 groupname name to give this dataset.

\item {} 
\sphinxstyleliteralstrong{data} (\sphinxstyleliteralemphasis{np.ndarray}) \textendash{} Numpy array of data to be saved.

\item {} 
\sphinxstyleliteralstrong{context} (\sphinxstyleliteralemphasis{str}\sphinxstyleliteralemphasis{, }\sphinxstyleliteralemphasis{optional}) \textendash{} Specifies what kind of data is stored. Eg. “frameset”,
“metadata”, “map”.

\item {} 
\sphinxstyleliteralstrong{attrs} (\sphinxstyleliteralemphasis{dict}\sphinxstyleliteralemphasis{, }\sphinxstyleliteralemphasis{optional}) \textendash{} Dictionary containing HDF5 metadata attributes to be set on
the resulting dataset.

\item {} 
\sphinxstyleliteralstrong{*args} \textendash{} Arguments to pass to h5py’s \sphinxcode{create\_dataset} method.

\item {} 
\sphinxstyleliteralstrong{**kwargs} \textendash{} Keyword arguments to pass to h5py’s \sphinxcode{create\_dataset} method.

\end{itemize}

\end{description}\end{quote}

\end{fulllineitems}

\index{set\_frames() (xanespy.txmstore.TXMStore method)}

\begin{fulllineitems}
\phantomsection\label{\detokenize{xanespy:xanespy.txmstore.TXMStore.set_frames}}\pysiglinewithargsret{\sphinxbfcode{set\_frames}}{\emph{name}, \emph{val}}{}
Set data for a set of frames, specificied by the value of \sphinxtitleref{name}.

\end{fulllineitems}

\index{signal\_map (xanespy.txmstore.TXMStore attribute)}

\begin{fulllineitems}
\phantomsection\label{\detokenize{xanespy:xanespy.txmstore.TXMStore.signal_map}}\pysigline{\sphinxbfcode{signal\_map}}
\end{fulllineitems}

\index{signal\_method (xanespy.txmstore.TXMStore attribute)}

\begin{fulllineitems}
\phantomsection\label{\detokenize{xanespy:xanespy.txmstore.TXMStore.signal_method}}\pysigline{\sphinxbfcode{signal\_method}}
String describing how the previously extracted signals were
calculated.

\end{fulllineitems}

\index{signal\_weights (xanespy.txmstore.TXMStore attribute)}

\begin{fulllineitems}
\phantomsection\label{\detokenize{xanespy:xanespy.txmstore.TXMStore.signal_weights}}\pysigline{\sphinxbfcode{signal\_weights}}
Get the pixel weights of the previously extracted signals using any
one of a variety of decomposition methods, saved as
\sphinxtitleref{signal\_method}.

\end{fulllineitems}

\index{signals (xanespy.txmstore.TXMStore attribute)}

\begin{fulllineitems}
\phantomsection\label{\detokenize{xanespy:xanespy.txmstore.TXMStore.signals}}\pysigline{\sphinxbfcode{signals}}
Get the previously extracted signals using any one of a variety of
decomposition methods, saved as \sphinxtitleref{signal\_method}.

\end{fulllineitems}

\index{timestamps (xanespy.txmstore.TXMStore attribute)}

\begin{fulllineitems}
\phantomsection\label{\detokenize{xanespy:xanespy.txmstore.TXMStore.timestamps}}\pysigline{\sphinxbfcode{timestamps}}
\end{fulllineitems}

\index{timestep\_names (xanespy.txmstore.TXMStore attribute)}

\begin{fulllineitems}
\phantomsection\label{\detokenize{xanespy:xanespy.txmstore.TXMStore.timestep_names}}\pysigline{\sphinxbfcode{timestep\_names}}
\end{fulllineitems}

\index{whiteline\_fit (xanespy.txmstore.TXMStore attribute)}

\begin{fulllineitems}
\phantomsection\label{\detokenize{xanespy:xanespy.txmstore.TXMStore.whiteline_fit}}\pysigline{\sphinxbfcode{whiteline\_fit}}
\end{fulllineitems}

\index{whiteline\_max (xanespy.txmstore.TXMStore attribute)}

\begin{fulllineitems}
\phantomsection\label{\detokenize{xanespy:xanespy.txmstore.TXMStore.whiteline_max}}\pysigline{\sphinxbfcode{whiteline\_max}}
\end{fulllineitems}


\end{fulllineitems}



\subsection{xanespy.utilities module}
\label{\detokenize{xanespy:xanespy-utilities-module}}\label{\detokenize{xanespy:module-xanespy.utilities}}\index{xanespy.utilities (module)}
A collection of classes and functions that arent’ specific to any
one type of measurement. Also, it defines some namedtuples for
describing coordinates.
\index{Extent (in module xanespy.utilities)}

\begin{fulllineitems}
\phantomsection\label{\detokenize{xanespy:xanespy.utilities.Extent}}\pysigline{\sphinxcode{xanespy.utilities.}\sphinxbfcode{Extent}}
alias of \sphinxcode{extent}

\end{fulllineitems}

\index{Pixel (class in xanespy.utilities)}

\begin{fulllineitems}
\phantomsection\label{\detokenize{xanespy:xanespy.utilities.Pixel}}\pysiglinewithargsret{\sphinxstrong{class }\sphinxcode{xanespy.utilities.}\sphinxbfcode{Pixel}}{\emph{vertical}, \emph{horizontal}}{}
Bases: \sphinxcode{tuple}
\index{horizontal (xanespy.utilities.Pixel attribute)}

\begin{fulllineitems}
\phantomsection\label{\detokenize{xanespy:xanespy.utilities.Pixel.horizontal}}\pysigline{\sphinxbfcode{horizontal}}
Alias for field number 1

\end{fulllineitems}

\index{vertical (xanespy.utilities.Pixel attribute)}

\begin{fulllineitems}
\phantomsection\label{\detokenize{xanespy:xanespy.utilities.Pixel.vertical}}\pysigline{\sphinxbfcode{vertical}}
Alias for field number 0

\end{fulllineitems}


\end{fulllineitems}

\index{broadcast\_reverse() (in module xanespy.utilities)}

\begin{fulllineitems}
\phantomsection\label{\detokenize{xanespy:xanespy.utilities.broadcast_reverse}}\pysiglinewithargsret{\sphinxcode{xanespy.utilities.}\sphinxbfcode{broadcast\_reverse}}{\emph{array}, \emph{shape}, \emph{*args}, \emph{**kwargs}}{}
Take the array and extends it as much as possible to match
\sphinxtitleref{shape}. Similar to numpy’s broadcast\_to function, but starts with
the most significant axis. For example, if \sphinxtitleref{array} has shape (7,
29), it can be broadcast to (7, 29, 1024, 1024).

\end{fulllineitems}

\index{foreach() (in module xanespy.utilities)}

\begin{fulllineitems}
\phantomsection\label{\detokenize{xanespy:xanespy.utilities.foreach}}\pysiglinewithargsret{\sphinxcode{xanespy.utilities.}\sphinxbfcode{foreach}}{\emph{f}, \emph{l}, \emph{threads=4}, \emph{return\_=False}}{}
Apply f to each element of l, in parallel

\end{fulllineitems}

\index{get\_component() (in module xanespy.utilities)}

\begin{fulllineitems}
\phantomsection\label{\detokenize{xanespy:xanespy.utilities.get_component}}\pysiglinewithargsret{\sphinxcode{xanespy.utilities.}\sphinxbfcode{get\_component}}{\emph{data}, \emph{name}}{}
If complex, turn to given component, otherwise return original data.

\end{fulllineitems}

\index{is\_kernel() (in module xanespy.utilities)}

\begin{fulllineitems}
\phantomsection\label{\detokenize{xanespy:xanespy.utilities.is_kernel}}\pysiglinewithargsret{\sphinxcode{xanespy.utilities.}\sphinxbfcode{is\_kernel}}{}{}
Detect whether or not we’re running inside an IPython kernel. NB:
This does not distinguish between eg IPython notebook and IPython
QtConsole.

\end{fulllineitems}

\index{parallel\_map() (in module xanespy.utilities)}

\begin{fulllineitems}
\phantomsection\label{\detokenize{xanespy:xanespy.utilities.parallel_map}}\pysiglinewithargsret{\sphinxcode{xanespy.utilities.}\sphinxbfcode{parallel\_map}}{\emph{f}, \emph{l}, \emph{threads=4}}{}
\end{fulllineitems}

\index{pixel\_to\_xy() (in module xanespy.utilities)}

\begin{fulllineitems}
\phantomsection\label{\detokenize{xanespy:xanespy.utilities.pixel_to_xy}}\pysiglinewithargsret{\sphinxcode{xanespy.utilities.}\sphinxbfcode{pixel\_to\_xy}}{\emph{pixel}, \emph{extent}, \emph{shape}}{}
Take an xy location on an image and convert it to a pixel location
suitable for numpy indexing.

\end{fulllineitems}

\index{position (class in xanespy.utilities)}

\begin{fulllineitems}
\phantomsection\label{\detokenize{xanespy:xanespy.utilities.position}}\pysiglinewithargsret{\sphinxstrong{class }\sphinxcode{xanespy.utilities.}\sphinxbfcode{position}}{\emph{x}, \emph{y}, \emph{z}}{}
Bases: \sphinxcode{tuple}
\index{x (xanespy.utilities.position attribute)}

\begin{fulllineitems}
\phantomsection\label{\detokenize{xanespy:xanespy.utilities.position.x}}\pysigline{\sphinxbfcode{x}}
Alias for field number 0

\end{fulllineitems}

\index{y (xanespy.utilities.position attribute)}

\begin{fulllineitems}
\phantomsection\label{\detokenize{xanespy:xanespy.utilities.position.y}}\pysigline{\sphinxbfcode{y}}
Alias for field number 1

\end{fulllineitems}

\index{z (xanespy.utilities.position attribute)}

\begin{fulllineitems}
\phantomsection\label{\detokenize{xanespy:xanespy.utilities.position.z}}\pysigline{\sphinxbfcode{z}}
Alias for field number 2

\end{fulllineitems}


\end{fulllineitems}

\index{prog() (in module xanespy.utilities)}

\begin{fulllineitems}
\phantomsection\label{\detokenize{xanespy:xanespy.utilities.prog}}\pysiglinewithargsret{\sphinxcode{xanespy.utilities.}\sphinxbfcode{prog}}{\emph{iterable=None}, \emph{leave=None}, \emph{dynamic\_ncols=True}, \emph{*args}, \emph{**kwargs}}{}
A progress bar for displaying how many iterations have been
completed.

This is mostly just a wrapper around the tqdm library. args and
kwargs are passed directly to either tqdm.tqdm or
tqdm.tqdm\_notebook. This function also takes into account the
value of \sphinxcode{USE\_PROG} defined in this module. If tqdm is no
installed, then calls to prog will just return the iterable again.
\begin{quote}\begin{description}
\item[{Parameters}] \leavevmode\begin{itemize}
\item {} 
\sphinxstyleliteralstrong{iterable} \textendash{} Iterable to decorate with a progressbar. See \sphinxcode{tqdm.tqdm}
documentation for more details.

\item {} 
\sphinxstyleliteralstrong{leave} (\sphinxstyleliteralemphasis{bool}\sphinxstyleliteralemphasis{, }\sphinxstyleliteralemphasis{optional}) \textendash{} Whether to leave the progress bar in the stream after it’s
completed. If omitted, will depend on terminal used.

\item {} 
\sphinxstyleliteralstrong{dynamic\_ncols} (\sphinxstyleliteralemphasis{bool}\sphinxstyleliteralemphasis{, }\sphinxstyleliteralemphasis{optional}) \textendash{} Whether to adapt dynamically to the width of the environment.

\item {} 
\sphinxstyleliteralstrong{args} \textendash{} Positional arguments passed directly to \sphinxcode{tqdm} or \sphinxcode{tqdm\_notebook}.

\item {} 
\sphinxstyleliteralstrong{kwargs} \textendash{} Keyword arguments passed directly to \sphinxcode{tqdm} or \sphinxcode{tqdm\_notebook}.

\end{itemize}

\end{description}\end{quote}

\end{fulllineitems}

\index{shape (class in xanespy.utilities)}

\begin{fulllineitems}
\phantomsection\label{\detokenize{xanespy:xanespy.utilities.shape}}\pysiglinewithargsret{\sphinxstrong{class }\sphinxcode{xanespy.utilities.}\sphinxbfcode{shape}}{\emph{rows}, \emph{columns}}{}
Bases: \sphinxcode{tuple}
\index{columns (xanespy.utilities.shape attribute)}

\begin{fulllineitems}
\phantomsection\label{\detokenize{xanespy:xanespy.utilities.shape.columns}}\pysigline{\sphinxbfcode{columns}}
Alias for field number 1

\end{fulllineitems}

\index{rows (xanespy.utilities.shape attribute)}

\begin{fulllineitems}
\phantomsection\label{\detokenize{xanespy:xanespy.utilities.shape.rows}}\pysigline{\sphinxbfcode{rows}}
Alias for field number 0

\end{fulllineitems}


\end{fulllineitems}

\index{xy\_to\_pixel() (in module xanespy.utilities)}

\begin{fulllineitems}
\phantomsection\label{\detokenize{xanespy:xanespy.utilities.xy_to_pixel}}\pysiglinewithargsret{\sphinxcode{xanespy.utilities.}\sphinxbfcode{xy\_to\_pixel}}{\emph{xy}, \emph{extent}, \emph{shape}}{}
Take an xy location on an image and convert it to a pixel location
suitable for numpy indexing.

\end{fulllineitems}

\index{xycoord (class in xanespy.utilities)}

\begin{fulllineitems}
\phantomsection\label{\detokenize{xanespy:xanespy.utilities.xycoord}}\pysiglinewithargsret{\sphinxstrong{class }\sphinxcode{xanespy.utilities.}\sphinxbfcode{xycoord}}{\emph{x}, \emph{y}}{}
Bases: \sphinxcode{tuple}
\index{x (xanespy.utilities.xycoord attribute)}

\begin{fulllineitems}
\phantomsection\label{\detokenize{xanespy:xanespy.utilities.xycoord.x}}\pysigline{\sphinxbfcode{x}}
Alias for field number 0

\end{fulllineitems}

\index{y (xanespy.utilities.xycoord attribute)}

\begin{fulllineitems}
\phantomsection\label{\detokenize{xanespy:xanespy.utilities.xycoord.y}}\pysigline{\sphinxbfcode{y}}
Alias for field number 1

\end{fulllineitems}


\end{fulllineitems}



\subsection{xanespy.xanes\_frameset module}
\label{\detokenize{xanespy:xanespy-xanes-frameset-module}}\label{\detokenize{xanespy:module-xanespy.xanes_frameset}}\index{xanespy.xanes\_frameset (module)}
Class definitions for working with a whole stack of X-ray
microscopy frames. Each frame is a micrograph at a different energy. A
frameset then is a three-dimenional dataset with of dimenions (energy,
row, column).
\index{PtychoFrameset (class in xanespy.xanes\_frameset)}

\begin{fulllineitems}
\phantomsection\label{\detokenize{xanespy:xanespy.xanes_frameset.PtychoFrameset}}\pysiglinewithargsret{\sphinxstrong{class }\sphinxcode{xanespy.xanes\_frameset.}\sphinxbfcode{PtychoFrameset}}{\emph{filename}, \emph{edge}, \emph{groupname=None}}{}
Bases: {\hyperref[\detokenize{xanespy:xanespy.xanes_frameset.XanesFrameset}]{\sphinxcrossref{\sphinxcode{xanespy.xanes\_frameset.XanesFrameset}}}}

A set of images (“frames”) at different energies moving across an
absorption edge. The individual frames should be generated by
ptychographic reconstruction of scanning transmission X-ray
microscopy (STXM) to produce an array complex intensity
values. This class does \sphinxstyleemphasis{not} include any code responsible for the
collection and reconstruction of such data, only for the analysis
in the context of X-ray absorption near edge spectroscopy.
\index{representations() (xanespy.xanes\_frameset.PtychoFrameset method)}

\begin{fulllineitems}
\phantomsection\label{\detokenize{xanespy:xanespy.xanes_frameset.PtychoFrameset.representations}}\pysiglinewithargsret{\sphinxbfcode{representations}}{}{}
Retrieve a list of valid representations for these data, such as
modulus or phase data for ptychography.

\end{fulllineitems}


\end{fulllineitems}

\index{XanesFrameset (class in xanespy.xanes\_frameset)}

\begin{fulllineitems}
\phantomsection\label{\detokenize{xanespy:xanespy.xanes_frameset.XanesFrameset}}\pysiglinewithargsret{\sphinxstrong{class }\sphinxcode{xanespy.xanes\_frameset.}\sphinxbfcode{XanesFrameset}}{\emph{filename}, \emph{edge}, \emph{groupname=None}}{}
Bases: \sphinxcode{object}

A collection of TXM frames at different energies moving across an
absorption edge. Iterating over this object gives the individual
Frame() objects. The class assumes that the data have been
imported into an HDF file.
\index{active\_group (xanespy.xanes\_frameset.XanesFrameset attribute)}

\begin{fulllineitems}
\phantomsection\label{\detokenize{xanespy:xanespy.xanes_frameset.XanesFrameset.active_group}}\pysigline{\sphinxbfcode{active\_group}\sphinxstrong{ = ”}}
\end{fulllineitems}

\index{align\_frames() (xanespy.xanes\_frameset.XanesFrameset method)}

\begin{fulllineitems}
\phantomsection\label{\detokenize{xanespy:xanespy.xanes_frameset.XanesFrameset.align_frames}}\pysiglinewithargsret{\sphinxbfcode{align\_frames}}{\emph{reference\_frame=’mean’}, \emph{blur=None}, \emph{method: str = ‘cross\_correlation’}, \emph{template=None}, \emph{passes=1}, \emph{commit=True}, \emph{component=’modulus’}, \emph{plot\_results=True}}{}
Use cross correlation algorithm to line up the frames. All frames
will have their sample position set to (0, 0) since we don’t
know which one is the real position. This operation will
interpolate between pixels so introduces error. If multiple
passes are performed, the translations are saved and combined
at the end so this error is only introduced once. Using the
\sphinxtitleref{commit=False} argument allows for multiple different types of
registration to be performed in sequence, since uncommitted
translations will be applied before the next round of
registration.
\begin{quote}\begin{description}
\item[{Parameters}] \leavevmode\begin{itemize}
\item {} 
\sphinxstyleliteralstrong{(}\sphinxstyleliteralstrong{int}\sphinxstyleliteralstrong{, }\sphinxstyleliteralstrong{str}\sphinxstyleliteralstrong{ or }\sphinxstyleliteralstrong{None}\sphinxstyleliteralstrong{)} (\sphinxstyleliteralemphasis{reference\_frame}) \textendash{} which all other frames should be aligned. If None, the frame
of highest intensity will be used. If “mean” (default) or
“median”, the average or median of all frames will be
used. If “max”, the frame with highest absorbance is
used. This attribute has no effect if template matching is
used.

\item {} 
\sphinxstyleliteralstrong{blur} (\sphinxstyleliteralemphasis{A type of filter to apply to each frame of the data}) \textendash{} before attempting registration. Choices are “median” or None

\item {} 
\sphinxstyleliteralstrong{method} (\sphinxstyleliteralemphasis{Which technique to use to calculate the translation}) \textendash{} \begin{itemize}
\item {} 
“cross\_correlation” (default)

\item {} 
”template\_match”

\end{itemize}

(If “template\_match” is used, the \sphinxtitleref{template} argument should
also be provided.)


\item {} 
\sphinxstyleliteralstrong{passes} (\sphinxstyleliteralemphasis{How many times this alignment should be done. Default: 1.}) \textendash{} 

\item {} 
\sphinxstyleliteralstrong{template} (\sphinxstyleliteralemphasis{Image data that should be matched if the}) \textendash{} \sphinxtitleref{template\_match} method is used.

\item {} 
\sphinxstyleliteralstrong{commit} (\sphinxstyleliteralemphasis{If truthy}\sphinxstyleliteralemphasis{ (}\sphinxstyleliteralemphasis{default}\sphinxstyleliteralemphasis{)}\sphinxstyleliteralemphasis{, }\sphinxstyleliteralemphasis{the final translation will be}) \textendash{} applied to the data stored on disk by calling
\sphinxtitleref{self.apply\_translations(crop=True)} after all passes have
finished.

\item {} 
\sphinxstyleliteralstrong{component} (\sphinxstyleliteralemphasis{What component of the data to use: 'modulus'}\sphinxstyleliteralemphasis{,}) \textendash{} ‘phase’, ‘imag’ or ‘real’.

\item {} 
\sphinxstyleliteralstrong{plot\_results} (\sphinxstyleliteralemphasis{If truthy}\sphinxstyleliteralemphasis{ (}\sphinxstyleliteralemphasis{default}\sphinxstyleliteralemphasis{)}\sphinxstyleliteralemphasis{, }\sphinxstyleliteralemphasis{plot the root-mean-square of the}) \textendash{} translation distance for each pass.

\end{itemize}

\end{description}\end{quote}

\end{fulllineitems}

\index{apply\_internal\_reference() (xanespy.xanes\_frameset.XanesFrameset method)}

\begin{fulllineitems}
\phantomsection\label{\detokenize{xanespy:xanespy.xanes_frameset.XanesFrameset.apply_internal_reference}}\pysiglinewithargsret{\sphinxbfcode{apply\_internal\_reference}}{\emph{plot\_background=True}, \emph{ax=None}}{}
Use a portion of each frame for internal reference correction. The
result is the complex refraction for each pixel: the real
component describes the phase shift, and the imaginary
component is exponential decay, ie. absorbance.
\begin{quote}\begin{description}
\item[{Parameters}] \leavevmode\begin{itemize}
\item {} 
\sphinxstyleliteralstrong{plot\_background} (\sphinxstyleliteralemphasis{If truthy}\sphinxstyleliteralemphasis{, }\sphinxstyleliteralemphasis{the values of I\_0 are plotted as}) \textendash{} a function of energy.

\item {} 
\sphinxstyleliteralstrong{ax} (The axes to use for plotting if \sphinxtitleref{plot\_background} is) \textendash{} truthy.

\end{itemize}

\end{description}\end{quote}

\end{fulllineitems}

\index{apply\_transformations() (xanespy.xanes\_frameset.XanesFrameset method)}

\begin{fulllineitems}
\phantomsection\label{\detokenize{xanespy:xanespy.xanes_frameset.XanesFrameset.apply_transformations}}\pysiglinewithargsret{\sphinxbfcode{apply\_transformations}}{\emph{crop=True}, \emph{commit=True}}{}
Take any transformations staged with \sphinxtitleref{self.stage\_transformations()}
and apply them. If commit is truthy, the staged
transformations are reset.

Returns: Transformed array of the absorbances frames.
\begin{quote}\begin{description}
\item[{Parameters}] \leavevmode\begin{itemize}
\item {} 
\sphinxstyleliteralstrong{crop} (\sphinxstyleliteralemphasis{-}) \textendash{} 

\item {} 
\sphinxstyleliteralstrong{so there are not edges. If falsy}\sphinxstyleliteralstrong{, }\sphinxstyleliteralstrong{the images will} (\sphinxstyleliteralemphasis{translated}\sphinxstyleliteralemphasis{,}) \textendash{} 

\item {} 
\sphinxstyleliteralstrong{wrapped.} (\sphinxstyleliteralemphasis{be}) \textendash{} 

\item {} 
\sphinxstyleliteralstrong{commit} (\sphinxstyleliteralemphasis{-}) \textendash{} store for absorbances, intensities and references, and the
staged transformations will be cleared. Otherwise, only the
absorbance data will be transformed and returned.

\item {} 
\sphinxstyleliteralstrong{frames\_name} (\sphinxstyleliteralemphasis{-}) \textendash{} transformation too (eg. ‘absorbances’)

\end{itemize}

\end{description}\end{quote}

\end{fulllineitems}

\index{calculate\_maps() (xanespy.xanes\_frameset.XanesFrameset method)}

\begin{fulllineitems}
\phantomsection\label{\detokenize{xanespy:xanespy.xanes_frameset.XanesFrameset.calculate_maps}}\pysiglinewithargsret{\sphinxbfcode{calculate\_maps}}{\emph{fit\_spectra=False}}{}
Generate a set of maps based on pixel-wise Xanes spectra: whiteline
position, particle labels.
\begin{quote}\begin{description}
\item[{Parameters}] \leavevmode
\sphinxstyleliteralstrong{-fit\_spectra} (\sphinxstyleliteralemphasis{If truthy}\sphinxstyleliteralemphasis{, }\sphinxstyleliteralemphasis{the whiteline will be found by}) \textendash{} fitting curves, instead of the default of taking the direct
maximum. This is likely to be very slow.

\end{description}\end{quote}

\end{fulllineitems}

\index{calculate\_signals() (xanespy.xanes\_frameset.XanesFrameset method)}

\begin{fulllineitems}
\phantomsection\label{\detokenize{xanespy:xanespy.xanes_frameset.XanesFrameset.calculate_signals}}\pysiglinewithargsret{\sphinxbfcode{calculate\_signals}}{\emph{n\_components=2}, \emph{method=’nmf’}, \emph{frame\_source=’absorbances’}, \emph{edge\_mask=True}}{}
Extract signals and assign each pixel to a group, then save the
resulting RGB cluster map.
\begin{quote}\begin{description}
\item[{Parameters}] \leavevmode\begin{itemize}
\item {} 
\sphinxstyleliteralstrong{n\_components} (\sphinxstyleliteralemphasis{int}\sphinxstyleliteralemphasis{, }\sphinxstyleliteralemphasis{optional}) \textendash{} The number of signals and number of clusters into which the
data will be separated.

\item {} 
\sphinxstyleliteralstrong{method} (\sphinxstyleliteralemphasis{str}\sphinxstyleliteralemphasis{, }\sphinxstyleliteralemphasis{optional}) \textendash{} The technique to use for extracting signals. Currently only
“nmf” is supported.

\item {} 
\sphinxstyleliteralstrong{frame\_source} (\sphinxstyleliteralemphasis{str}\sphinxstyleliteralemphasis{, }\sphinxstyleliteralemphasis{optional}) \textendash{} Name of the frame-set to use as the input data.

\item {} 
\sphinxstyleliteralstrong{edge\_mask} (\sphinxstyleliteralemphasis{bool}\sphinxstyleliteralemphasis{, }\sphinxstyleliteralemphasis{optional}) \textendash{} If truthy (default), only those pixels passing the edge
filter will be considered.

\end{itemize}

\end{description}\end{quote}

\end{fulllineitems}

\index{calculate\_whitelines() (xanespy.xanes\_frameset.XanesFrameset method)}

\begin{fulllineitems}
\phantomsection\label{\detokenize{xanespy:xanespy.xanes_frameset.XanesFrameset.calculate_whitelines}}\pysiglinewithargsret{\sphinxbfcode{calculate\_whitelines}}{\emph{edge\_mask=False}}{}
Calculate and save a map of the whiteline position of each pixel by
calculating the energy of simple maximum absorbance.
\begin{quote}\begin{description}
\item[{Parameters}] \leavevmode
\sphinxstyleliteralstrong{edge\_mask} (\sphinxstyleliteralemphasis{-}) \textendash{} be fit and the remaning pixels will be set to a default
value. This can help reduce computing time.

\end{description}\end{quote}

\end{fulllineitems}

\index{clear\_caches() (xanespy.xanes\_frameset.XanesFrameset method)}

\begin{fulllineitems}
\phantomsection\label{\detokenize{xanespy:xanespy.xanes_frameset.XanesFrameset.clear_caches}}\pysiglinewithargsret{\sphinxbfcode{clear\_caches}}{}{}
Clear cached function values so they will be recomputed with fresh
data

\end{fulllineitems}

\index{cmap (xanespy.xanes\_frameset.XanesFrameset attribute)}

\begin{fulllineitems}
\phantomsection\label{\detokenize{xanespy:xanespy.xanes_frameset.XanesFrameset.cmap}}\pysigline{\sphinxbfcode{cmap}\sphinxstrong{ = ‘plasma’}}
\end{fulllineitems}

\index{components() (xanespy.xanes\_frameset.XanesFrameset method)}

\begin{fulllineitems}
\phantomsection\label{\detokenize{xanespy:xanespy.xanes_frameset.XanesFrameset.components}}\pysiglinewithargsret{\sphinxbfcode{components}}{}{}
Retrieve a list of valid representations for these data.

\end{fulllineitems}

\index{data\_name (xanespy.xanes\_frameset.XanesFrameset attribute)}

\begin{fulllineitems}
\phantomsection\label{\detokenize{xanespy:xanespy.xanes_frameset.XanesFrameset.data_name}}\pysigline{\sphinxbfcode{data\_name}}
\end{fulllineitems}

\index{data\_tree() (xanespy.xanes\_frameset.XanesFrameset method)}

\begin{fulllineitems}
\phantomsection\label{\detokenize{xanespy:xanespy.xanes_frameset.XanesFrameset.data_tree}}\pysiglinewithargsret{\sphinxbfcode{data\_tree}}{}{}
Wrapper around the TXMStore.data\_tree() method.

\end{fulllineitems}

\index{edge\_mask (xanespy.xanes\_frameset.XanesFrameset attribute)}

\begin{fulllineitems}
\phantomsection\label{\detokenize{xanespy:xanespy.xanes_frameset.XanesFrameset.edge_mask}}\pysigline{\sphinxbfcode{edge\_mask}}
Calculate a mask for what is likely active material at this
edge.
\begin{quote}\begin{description}
\item[{Parameters}] \leavevmode\begin{itemize}
\item {} 
\sphinxstyleliteralstrong{sensitivity} (\sphinxstyleliteralemphasis{float}) \textendash{} A multiplier for the otsu value to determine
the actual threshold.

\item {} 
\sphinxstyleliteralstrong{min\_size} (\sphinxstyleliteralemphasis{int}) \textendash{} Objects below this size (in pixels) will be removed. Passing
zero (default) will result in no effect.

\end{itemize}

\end{description}\end{quote}

\end{fulllineitems}

\index{endtime() (xanespy.xanes\_frameset.XanesFrameset method)}

\begin{fulllineitems}
\phantomsection\label{\detokenize{xanespy:xanespy.xanes_frameset.XanesFrameset.endtime}}\pysiglinewithargsret{\sphinxbfcode{endtime}}{\emph{timeidx=None}}{}
Determine the latest timestamp amongst all of the frames.
\begin{quote}\begin{description}
\item[{Parameters}] \leavevmode
\sphinxstyleliteralstrong{timeidx} (\sphinxstyleliteralemphasis{int}\sphinxstyleliteralemphasis{, }\sphinxstyleliteralemphasis{optional}) \textendash{} Which timestep to use for finding the end time. If
omitted or None (default), all timesteps will be checked.

\item[{Returns}] \leavevmode
\sphinxstylestrong{start\_time} \textendash{} Naive datetime representing the latest known frame for
this time index. Timezone will always be UTC no matter where
the data were collected.

\item[{Return type}] \leavevmode
np.datetime64

\end{description}\end{quote}

\end{fulllineitems}

\index{energies (xanespy.xanes\_frameset.XanesFrameset attribute)}

\begin{fulllineitems}
\phantomsection\label{\detokenize{xanespy:xanespy.xanes_frameset.XanesFrameset.energies}}\pysigline{\sphinxbfcode{energies}}
Return the array of beam energies for the given time index.
\begin{quote}\begin{description}
\item[{Returns}] \leavevmode
\sphinxstylestrong{energies} \textendash{} A 1-dimensional array with the energy for each frame.

\item[{Return type}] \leavevmode
np.ndarray

\end{description}\end{quote}

\end{fulllineitems}

\index{extent (xanespy.xanes\_frameset.XanesFrameset attribute)}

\begin{fulllineitems}
\phantomsection\label{\detokenize{xanespy:xanespy.xanes_frameset.XanesFrameset.extent}}\pysigline{\sphinxbfcode{extent}}
Determine physical dimensions for axes values.

If an index is given, it will first be applied to the frames
array. For any remaining dimensions besides the last two, the
median will be taken. For an array of extents for each frame,
use the \sphinxcode{extent\_array} method.
\begin{quote}\begin{description}
\item[{Parameters}] \leavevmode\begin{itemize}
\item {} 
\sphinxstyleliteralstrong{representation} (\sphinxstyleliteralemphasis{str}\sphinxstyleliteralemphasis{, }\sphinxstyleliteralemphasis{optional}) \textendash{} Name for which dataset to use.

\item {} 
\sphinxstyleliteralstrong{idx} (\sphinxstyleliteralemphasis{int}\sphinxstyleliteralemphasis{, }\sphinxstyleliteralemphasis{optional}) \textendash{} Index for choosing a frame. Any valid numpy index is
allowed, eg. \sphinxcode{...} (default) uses all frame.

\end{itemize}

\item[{Returns}] \leavevmode
\sphinxstylestrong{extent} \textendash{} The spatial extent for the frame with order specified by
\sphinxcode{utilities.Extent}

\item[{Return type}] \leavevmode
tuple

\end{description}\end{quote}

\end{fulllineitems}

\index{extent\_array() (xanespy.xanes\_frameset.XanesFrameset method)}

\begin{fulllineitems}
\phantomsection\label{\detokenize{xanespy:xanespy.xanes_frameset.XanesFrameset.extent_array}}\pysiglinewithargsret{\sphinxbfcode{extent\_array}}{\emph{representation=’intensities’}}{}
\end{fulllineitems}

\index{fit\_spectra() (xanespy.xanes\_frameset.XanesFrameset method)}

\begin{fulllineitems}
\phantomsection\label{\detokenize{xanespy:xanespy.xanes_frameset.XanesFrameset.fit_spectra}}\pysiglinewithargsret{\sphinxbfcode{fit\_spectra}}{\emph{edge\_mask=True}}{}
Fit a series of curves to the spectrum at each pixel.

For anything other than trivially small data-sets, this method
can take a very long time. To speed up processing, MPI calls
are used. Calling this function with mpiexec will allow for
more parallel processing.
\begin{quote}\begin{description}
\item[{Parameters}] \leavevmode
\sphinxstyleliteralstrong{edge\_mask} (\sphinxstyleliteralemphasis{bool}\sphinxstyleliteralemphasis{, }\sphinxstyleliteralemphasis{optional}) \textendash{} If true, only pixels passing the edge\_mask will be fit and
the remaning pixels will be set to a default value. This can
help reduce computing time.

\end{description}\end{quote}

\end{fulllineitems}

\index{fitted\_spectrum() (xanespy.xanes\_frameset.XanesFrameset method)}

\begin{fulllineitems}
\phantomsection\label{\detokenize{xanespy:xanespy.xanes_frameset.XanesFrameset.fitted_spectrum}}\pysiglinewithargsret{\sphinxbfcode{fitted\_spectrum}}{\emph{energies=None}, \emph{pixel=None}, \emph{index=0}, \emph{representation=’fit\_parameters’}}{}
Take the previously calculated fitting parameters from
\sphinxtitleref{fit\_spectra} and predict the XANES spectrum.
\begin{quote}\begin{description}
\item[{Parameters}] \leavevmode\begin{itemize}
\item {} 
\sphinxstyleliteralstrong{energies} (\sphinxstyleliteralemphasis{np.ndarray}\sphinxstyleliteralemphasis{, }\sphinxstyleliteralemphasis{optional}) \textendash{} The input values to give to the predictor function. If
omitted, a np.linspace will be created covering the range of
energies.

\item {} 
\sphinxstyleliteralstrong{pixel} (\sphinxstyleliteralemphasis{2-tuple}\sphinxstyleliteralemphasis{, }\sphinxstyleliteralemphasis{optional.}) \textendash{} The specific pixel to use for getting fit parameters. If
omitted, the average for each parameter calculated.

\item {} 
\sphinxstyleliteralstrong{index} (\sphinxstyleliteralemphasis{int}\sphinxstyleliteralemphasis{, }\sphinxstyleliteralemphasis{optional}) \textendash{} The timeindex to use for retrieving fit parameters.

\item {} 
\sphinxstyleliteralstrong{representation} (\sphinxstyleliteralemphasis{str}\sphinxstyleliteralemphasis{, }\sphinxstyleliteralemphasis{optional}) \textendash{} Which dataset name to use for accessing the fit
parameters. The default value (“fit\_parameters”) is the one
used by \sphinxcode{fit\_spectra} to save the results.

\end{itemize}

\item[{Returns}] \leavevmode
\sphinxstylestrong{fit} \textendash{} The predicted spectrum based on the previously calculated
fit parameters.

\item[{Return type}] \leavevmode
pd.Series

\end{description}\end{quote}

\end{fulllineitems}

\index{fork\_data\_group() (xanespy.xanes\_frameset.XanesFrameset method)}

\begin{fulllineitems}
\phantomsection\label{\detokenize{xanespy:xanespy.xanes_frameset.XanesFrameset.fork_data_group}}\pysiglinewithargsret{\sphinxbfcode{fork\_data\_group}}{\emph{dest}, \emph{src=None}}{}
Turn on different active data for this frameset’s store
object. Similar to \sphinxtitleref{switch\_data\_group} except that this method
deletes the existing group and copies symlinks from the current one.
\begin{quote}\begin{description}
\item[{Parameters}] \leavevmode\begin{itemize}
\item {} 
\sphinxstyleliteralstrong{dest} (\sphinxstyleliteralemphasis{str}) \textendash{} Name for the data group.

\item {} 
\sphinxstyleliteralstrong{src} (\sphinxstyleliteralemphasis{str}) \textendash{} String with the name of the data group to copy from. If None
(default), the current data group will be used.

\end{itemize}

\end{description}\end{quote}

\end{fulllineitems}

\index{frame\_shape() (xanespy.xanes\_frameset.XanesFrameset method)}

\begin{fulllineitems}
\phantomsection\label{\detokenize{xanespy:xanespy.xanes_frameset.XanesFrameset.frame_shape}}\pysiglinewithargsret{\sphinxbfcode{frame\_shape}}{\emph{representation=’intensities’}}{}
Return the shape of the individual energy frames.

\end{fulllineitems}

\index{frames (xanespy.xanes\_frameset.XanesFrameset attribute)}

\begin{fulllineitems}
\phantomsection\label{\detokenize{xanespy:xanespy.xanes_frameset.XanesFrameset.frames}}\pysigline{\sphinxbfcode{frames}}
Return the frames for the given time index.

If \sphinxtitleref{representation} is really mapping data, then the source
frames will be returned.
\begin{quote}\begin{description}
\item[{Parameters}] \leavevmode\begin{itemize}
\item {} 
\sphinxstyleliteralstrong{timeidx} (\sphinxstyleliteralemphasis{int}) \textendash{} Index for the first dimension of the combined data array.

\item {} 
\sphinxstyleliteralstrong{representation} (\sphinxstyleliteralemphasis{str}) \textendash{} The group name for these data. Eg “absorbances”,
“whiteline\_map”, “intensities”

\end{itemize}

\item[{Returns}] \leavevmode
\sphinxstylestrong{frames} \textendash{} A 3-dimensional array with the form (energy, row, col).

\item[{Return type}] \leavevmode
np.ndarray

\end{description}\end{quote}

\end{fulllineitems}

\index{has\_representation() (xanespy.xanes\_frameset.XanesFrameset method)}

\begin{fulllineitems}
\phantomsection\label{\detokenize{xanespy:xanespy.xanes_frameset.XanesFrameset.has_representation}}\pysiglinewithargsret{\sphinxbfcode{has\_representation}}{\emph{representation}}{}
\end{fulllineitems}

\index{hdf\_path() (xanespy.xanes\_frameset.XanesFrameset method)}

\begin{fulllineitems}
\phantomsection\label{\detokenize{xanespy:xanespy.xanes_frameset.XanesFrameset.hdf_path}}\pysiglinewithargsret{\sphinxbfcode{hdf\_path}}{\emph{representation=None}}{}
Return the hdf path for the active group.
\begin{quote}\begin{description}
\item[{Parameters}] \leavevmode
\sphinxstyleliteralstrong{representation} (\sphinxstyleliteralemphasis{str}\sphinxstyleliteralemphasis{, }\sphinxstyleliteralemphasis{optional}) \textendash{} Name of third-level group to use. If omitted, the path to
the parent group will be given.

\item[{Returns}] \leavevmode
\sphinxstylestrong{path} \textendash{} The path to the current group in the HDF5 file. Returns an
empty string if the representation does not exists.

\item[{Return type}] \leavevmode
str

\end{description}\end{quote}

\end{fulllineitems}

\index{label\_particles() (xanespy.xanes\_frameset.XanesFrameset method)}

\begin{fulllineitems}
\phantomsection\label{\detokenize{xanespy:xanespy.xanes_frameset.XanesFrameset.label_particles}}\pysiglinewithargsret{\sphinxbfcode{label\_particles}}{\emph{min\_distance=20}}{}
Use watershed segmentation to identify particles.
\begin{quote}\begin{description}
\item[{Parameters}] \leavevmode
\sphinxstyleliteralstrong{min\_distance} (\sphinxstyleliteralemphasis{-}) \textendash{} grouping areas into particles. Lower numbers means more
particles, but might split large particles into two.

\end{description}\end{quote}

\end{fulllineitems}

\index{map\_data (xanespy.xanes\_frameset.XanesFrameset attribute)}

\begin{fulllineitems}
\phantomsection\label{\detokenize{xanespy:xanespy.xanes_frameset.XanesFrameset.map_data}}\pysigline{\sphinxbfcode{map\_data}}
Return map data for the given time index and representation.

If \sphinxtitleref{representation} is really mapping data, then the result
will have more dimensions than expected.
\begin{quote}\begin{description}
\item[{Parameters}] \leavevmode\begin{itemize}
\item {} 
\sphinxstyleliteralstrong{timeidx} (\sphinxstyleliteralemphasis{int}) \textendash{} Index for the first dimension of the combined data array. If
the underlying map data has only 2 dimensions, this
parameter is ignored.

\item {} 
\sphinxstyleliteralstrong{representation} (\sphinxstyleliteralemphasis{str}) \textendash{} The group name for these data. Eg “absorbances”,
“whiteline\_map”, “intensities”

\end{itemize}

\item[{Returns}] \leavevmode
\sphinxstylestrong{map\_data} \textendash{} A 2-dimensional array with the form (row, col).

\item[{Return type}] \leavevmode
np.ndarray

\end{description}\end{quote}

\end{fulllineitems}

\index{mean\_frame() (xanespy.xanes\_frameset.XanesFrameset method)}

\begin{fulllineitems}
\phantomsection\label{\detokenize{xanespy:xanespy.xanes_frameset.XanesFrameset.mean_frame}}\pysiglinewithargsret{\sphinxbfcode{mean\_frame}}{\emph{representation=’absorbances’}}{}
Return the mean value with the same shape as an individual
frame.

\end{fulllineitems}

\index{normalize() (xanespy.xanes\_frameset.XanesFrameset method)}

\begin{fulllineitems}
\phantomsection\label{\detokenize{xanespy:xanespy.xanes_frameset.XanesFrameset.normalize}}\pysiglinewithargsret{\sphinxbfcode{normalize}}{\emph{plot\_fit=False}, \emph{new\_name=’normalized’}}{}
Correct for background material not absorbing at this edge. Uses
method described in DOI 10.1038/ncomms7883: fit line against
material that fails edge\_jump\_filter and use this line to
correct entire frame.
\begin{quote}\begin{description}
\item[{Parameters}] \leavevmode\begin{itemize}
\item {} 
\sphinxstyleliteralstrong{plot\_fit} (\sphinxstyleliteralemphasis{-}) \textendash{} 

\item {} 
\sphinxstyleliteralstrong{line.} (\sphinxstyleliteralemphasis{best-fit}) \textendash{} 

\end{itemize}

\end{description}\end{quote}

\end{fulllineitems}

\index{num\_energies (xanespy.xanes\_frameset.XanesFrameset attribute)}

\begin{fulllineitems}
\phantomsection\label{\detokenize{xanespy:xanespy.xanes_frameset.XanesFrameset.num_energies}}\pysigline{\sphinxbfcode{num\_energies}}
\end{fulllineitems}

\index{num\_timesteps (xanespy.xanes\_frameset.XanesFrameset attribute)}

\begin{fulllineitems}
\phantomsection\label{\detokenize{xanespy:xanespy.xanes_frameset.XanesFrameset.num_timesteps}}\pysigline{\sphinxbfcode{num\_timesteps}}
\end{fulllineitems}

\index{particle\_regions() (xanespy.xanes\_frameset.XanesFrameset method)}

\begin{fulllineitems}
\phantomsection\label{\detokenize{xanespy:xanespy.xanes_frameset.XanesFrameset.particle_regions}}\pysiglinewithargsret{\sphinxbfcode{particle\_regions}}{\emph{intensity\_image=None}, \emph{labels=None}}{}
Return a list of regions (1 for each particle) sorted by area.
(largest first). This requires that the \sphinxtitleref{label\_particles}
method be called first.
\begin{quote}\begin{description}
\item[{Parameters}] \leavevmode\begin{itemize}
\item {} 
\sphinxstyleliteralstrong{intensity\_image} (\sphinxstyleliteralemphasis{np.ndarray}\sphinxstyleliteralemphasis{, }\sphinxstyleliteralemphasis{optional}) \textendash{} 2D array passed on to the skimage \sphinxtitleref{regionprops} function to
determine what shows up in the image for each particle.

\item {} 
\sphinxstyleliteralstrong{labels} (\sphinxstyleliteralemphasis{np.ndarray}\sphinxstyleliteralemphasis{, }\sphinxstyleliteralemphasis{optional}) \textendash{} Array of the same shape as the map, with the particles
segmented. If None (default), the \sphinxtitleref{particle\_labels}
attribute of the TXM store will be used.

\end{itemize}

\end{description}\end{quote}

\end{fulllineitems}

\index{particle\_series() (xanespy.xanes\_frameset.XanesFrameset method)}

\begin{fulllineitems}
\phantomsection\label{\detokenize{xanespy:xanespy.xanes_frameset.XanesFrameset.particle_series}}\pysiglinewithargsret{\sphinxbfcode{particle\_series}}{\emph{map\_name=’whiteline\_max’}}{}
Generate an array of values from map\_name averaged across each
particle.

Returns: A 2D array where the first dimension is particles and
the second is the first dimension of the map dataset (usually time).

\end{fulllineitems}

\index{pixel\_unit() (xanespy.xanes\_frameset.XanesFrameset method)}

\begin{fulllineitems}
\phantomsection\label{\detokenize{xanespy:xanespy.xanes_frameset.XanesFrameset.pixel_unit}}\pysiglinewithargsret{\sphinxbfcode{pixel\_unit}}{}{}
Return the unit of measure for the size of a pixel.

\end{fulllineitems}

\index{plot\_frame() (xanespy.xanes\_frameset.XanesFrameset method)}

\begin{fulllineitems}
\phantomsection\label{\detokenize{xanespy:xanespy.xanes_frameset.XanesFrameset.plot_frame}}\pysiglinewithargsret{\sphinxbfcode{plot\_frame}}{\emph{idx}, \emph{ax=None}, \emph{cmap=’gray’}, \emph{*args}, \emph{**kwargs}}{}
Plot the frame with given index as an image.

\end{fulllineitems}

\index{plot\_histogram() (xanespy.xanes\_frameset.XanesFrameset method)}

\begin{fulllineitems}
\phantomsection\label{\detokenize{xanespy:xanespy.xanes_frameset.XanesFrameset.plot_histogram}}\pysiglinewithargsret{\sphinxbfcode{plot\_histogram}}{\emph{plotter=None}, \emph{timeidx=None}, \emph{ax=None}, \emph{vmin=None}, \emph{vmax=None}, \emph{goodness\_filter=False}, \emph{representation=’whiteline\_fit’}, \emph{component=’real’}, \emph{active\_pixel=None}, \emph{bins=’energies’}, \emph{*args}, \emph{**kwargs}}{}
Use a default frameset plotter to draw a map of the chemical
data.

\end{fulllineitems}

\index{plot\_map() (xanespy.xanes\_frameset.XanesFrameset method)}

\begin{fulllineitems}
\phantomsection\label{\detokenize{xanespy:xanespy.xanes_frameset.XanesFrameset.plot_map}}\pysiglinewithargsret{\sphinxbfcode{plot\_map}}{\emph{ax=None}, \emph{map\_name=’whiteline\_fit’}, \emph{timeidx=0}, \emph{vmin=None}, \emph{vmax=None}, \emph{median\_size=0}}{}
Prepare data and plot a map of whiteline positions.
\begin{quote}\begin{description}
\item[{Parameters}] \leavevmode
\sphinxstyleliteralstrong{median\_size} (\sphinxstyleliteralemphasis{int}) \textendash{} Kernel size for the median rank filter.

\end{description}\end{quote}

\end{fulllineitems}

\index{plot\_map\_pixel\_spectra() (xanespy.xanes\_frameset.XanesFrameset method)}

\begin{fulllineitems}
\phantomsection\label{\detokenize{xanespy:xanespy.xanes_frameset.XanesFrameset.plot_map_pixel_spectra}}\pysiglinewithargsret{\sphinxbfcode{plot\_map\_pixel\_spectra}}{\emph{pixels}, \emph{map\_ax=None}, \emph{spectra\_ax=None}, \emph{map\_name=’whiteline\_map’}, \emph{timeidx=0}, \emph{step\_size=0}, \emph{*args}, \emph{**kwargs}}{}
Plot the frameset’s map and highlight some pixels on it then plot
those pixel’s spectra on another set of axes.
\begin{quote}\begin{description}
\item[{Parameters}] \leavevmode\begin{itemize}
\item {} 
\sphinxstyleliteralstrong{pixels} (\sphinxstyleliteralemphasis{iterable}) \textendash{} An iterable of 2-tuples indicating which (row, column)
pixels to highlight.

\item {} 
\sphinxstyleliteralstrong{map\_ax} (\sphinxstyleliteralemphasis{optional}) \textendash{} A matplotlib axes object to put the map onto. If None, a new
2-wide subplot will be created for both map\_ax and
spectra\_ax.

\item {} 
\sphinxstyleliteralstrong{spectra\_ax} (\sphinxstyleliteralemphasis{optional}) \textendash{} A matplotlib axes to be used for plotting spectra. Will only
be used if \sphinxcode{map\_ax} is not None.

\item {} 
\sphinxstyleliteralstrong{map\_name} (\sphinxstyleliteralemphasis{str}\sphinxstyleliteralemphasis{, }\sphinxstyleliteralemphasis{optional}) \textendash{} Name of the map to use for plotting. It will be passed to
the TXM store object and retrieved from the hdf5 file. If
falsy, no map will be plotted.

\item {} 
\sphinxstyleliteralstrong{timeidx} (\sphinxstyleliteralemphasis{int}\sphinxstyleliteralemphasis{, }\sphinxstyleliteralemphasis{optional}) \textendash{} Index of which timestep to use (default: 0).

\item {} 
\sphinxstyleliteralstrong{kwargs} (\sphinxstyleliteralemphasis{args}\sphinxstyleliteralemphasis{,}) \textendash{} Passed to plots.plot\_pixel\_spectra()

\end{itemize}

\end{description}\end{quote}

\end{fulllineitems}

\index{plot\_mean\_image() (xanespy.xanes\_frameset.XanesFrameset method)}

\begin{fulllineitems}
\phantomsection\label{\detokenize{xanespy:xanespy.xanes_frameset.XanesFrameset.plot_mean_image}}\pysiglinewithargsret{\sphinxbfcode{plot\_mean\_image}}{\emph{ax=None}, \emph{component=’modulus’}, \emph{cmap=’gray’}, \emph{*args}, \emph{**kwargs}}{}
\end{fulllineitems}

\index{plot\_signal\_map() (xanespy.xanes\_frameset.XanesFrameset method)}

\begin{fulllineitems}
\phantomsection\label{\detokenize{xanespy:xanespy.xanes_frameset.XanesFrameset.plot_signal_map}}\pysiglinewithargsret{\sphinxbfcode{plot\_signal\_map}}{\emph{ax=None}, \emph{signals\_idx=None}, \emph{interpolation=None}}{}
Plot the map of signal strength for signals extracted from
self.calculate\_signals().
\begin{quote}\begin{description}
\item[{Parameters}] \leavevmode\begin{itemize}
\item {} 
\sphinxstyleliteralstrong{ax} (\sphinxstyleliteralemphasis{-}) \textendash{} axes object is created.

\item {} 
\sphinxstyleliteralstrong{signals\_idx} (\sphinxstyleliteralemphasis{-}) \textendash{} 

\item {} 
\sphinxstyleliteralstrong{as a numpy array index. Special value None}\sphinxstyleliteralstrong{ (}\sphinxstyleliteralstrong{default}\sphinxstyleliteralstrong{)} (\sphinxstyleliteralemphasis{passed}) \textendash{} 

\item {} 
\sphinxstyleliteralstrong{first three signals will be plotted.} (\sphinxstyleliteralemphasis{means}) \textendash{} 

\item {} 
\sphinxstyleliteralstrong{interpolation} (\sphinxstyleliteralemphasis{-}) \textendash{} How to smooth the image when plotting.

\end{itemize}

\end{description}\end{quote}

\end{fulllineitems}

\index{plot\_signals() (xanespy.xanes\_frameset.XanesFrameset method)}

\begin{fulllineitems}
\phantomsection\label{\detokenize{xanespy:xanespy.xanes_frameset.XanesFrameset.plot_signals}}\pysiglinewithargsret{\sphinxbfcode{plot\_signals}}{\emph{cmap=’viridis’}}{}
Plot the signals from the previously extracted data. Requires that
self.store().signals and self.store().signal\_weights be set.

\end{fulllineitems}

\index{plot\_xanes\_spectrum() (xanespy.xanes\_frameset.XanesFrameset method)}

\begin{fulllineitems}
\phantomsection\label{\detokenize{xanespy:xanespy.xanes_frameset.XanesFrameset.plot_xanes_spectrum}}\pysiglinewithargsret{\sphinxbfcode{plot\_xanes\_spectrum}}{\emph{ax=None}, \emph{pixel=None}, \emph{norm\_range=None}, \emph{normalize=False}, \emph{representation=’absorbances’}, \emph{show\_fit=False}, \emph{edge\_jump\_filter=False}, \emph{linestyle=’:’}, \emph{*args}, \emph{**kwargs}}{}
Calculate and plot the xanes spectrum for this field-of-view.
\begin{quote}\begin{description}
\item[{Parameters}] \leavevmode\begin{itemize}
\item {} 
\sphinxstyleliteralstrong{- matplotlib axes object on which to draw} (\sphinxstyleliteralemphasis{ax}) \textendash{} 

\item {} 
\sphinxstyleliteralstrong{- Coordinates of a specific pixel on the image to plot.} (\sphinxstyleliteralemphasis{pixel}) \textendash{} 

\item {} 
\sphinxstyleliteralstrong{- If truthy}\sphinxstyleliteralstrong{, }\sphinxstyleliteralstrong{will set the pre-edge at zero and the} ({\hyperref[\detokenize{xanespy:xanespy.xanes_frameset.XanesFrameset.normalize}]{\sphinxcrossref{\sphinxstyleliteralemphasis{normalize}}}}) \textendash{} post-edge at 1.

\item {} 
\sphinxstyleliteralstrong{- If truthy}\sphinxstyleliteralstrong{, }\sphinxstyleliteralstrong{will use the edge object to fit the data} (\sphinxstyleliteralemphasis{show\_fit}) \textendash{} and plot the resulting fit line.

\item {} 
\sphinxstyleliteralstrong{- If truthy}\sphinxstyleliteralstrong{, }\sphinxstyleliteralstrong{will only include those values} (\sphinxstyleliteralemphasis{edge\_jump\_filter}) \textendash{} that show a strong absorbtion jump across this edge.

\end{itemize}

\end{description}\end{quote}

\end{fulllineitems}

\index{qt\_viewer() (xanespy.xanes\_frameset.XanesFrameset method)}

\begin{fulllineitems}
\phantomsection\label{\detokenize{xanespy:xanespy.xanes_frameset.XanesFrameset.qt_viewer}}\pysiglinewithargsret{\sphinxbfcode{qt\_viewer}}{}{}
\end{fulllineitems}

\index{spectra() (xanespy.xanes\_frameset.XanesFrameset method)}

\begin{fulllineitems}
\phantomsection\label{\detokenize{xanespy:xanespy.xanes_frameset.XanesFrameset.spectra}}\pysiglinewithargsret{\sphinxbfcode{spectra}}{\emph{edge\_filter=False}}{}
Return a two-dimensional array of spectra for all the pixels in
shape of (pixel, energy).
\begin{quote}\begin{description}
\item[{Parameters}] \leavevmode
\sphinxstyleliteralstrong{(}\sphinxstyleliteralstrong{bool}\sphinxstyleliteralstrong{ or }\sphinxstyleliteralstrong{str}\sphinxstyleliteralstrong{)} (\sphinxstyleliteralemphasis{edge\_jump\_filter}) \textendash{} truthy, only pixels that pass the edge jump filter are
used to calculate the spectrum. If “inverse” is given,
then the edge jump filter is logically not-ted and
calculated with a more conservative threshold.

\end{description}\end{quote}

\end{fulllineitems}

\index{spectrum() (xanespy.xanes\_frameset.XanesFrameset method)}

\begin{fulllineitems}
\phantomsection\label{\detokenize{xanespy:xanespy.xanes_frameset.XanesFrameset.spectrum}}\pysiglinewithargsret{\sphinxbfcode{spectrum}}{\emph{pixel=None}, \emph{edge\_jump\_filter=False}, \emph{representation=’absorbances’}, \emph{index=0}}{}
Collapse the frameset down to an energy spectrum.

Any dimensions (besides the energy dimension) that remain
after applying the arguments below, will be averaged to give
the final intensity at each energy.
\begin{quote}\begin{description}
\item[{Returns}] \leavevmode
\sphinxstylestrong{spectrum} \textendash{} A pandas Series with the spectrum.

\item[{Return type}] \leavevmode
pd.Series

\item[{Parameters}] \leavevmode\begin{itemize}
\item {} 
\sphinxstyleliteralstrong{pixel} (\sphinxstyleliteralemphasis{tuple}\sphinxstyleliteralemphasis{, }\sphinxstyleliteralemphasis{optional}) \textendash{} A 2-tuple that causes the returned series to represent
the spectrum for only 1 pixel in the frameset. If None, a
larger part of the frame will be used, depending on the
other arguments.

\item {} 
\sphinxstyleliteralstrong{edge\_jump\_filter} (\sphinxstyleliteralemphasis{bool}\sphinxstyleliteralemphasis{ or }\sphinxstyleliteralemphasis{str}\sphinxstyleliteralemphasis{, }\sphinxstyleliteralemphasis{optional}) \textendash{} If truthy, only pixels that
pass the edge jump filter are used to calculate the
spectrum. If “inverse” is given, then the edge jump filter
is logically not-ted and calculated with a more
conservative threshold.

\item {} 
\sphinxstyleliteralstrong{representation} (\sphinxstyleliteralemphasis{str}\sphinxstyleliteralemphasis{, }\sphinxstyleliteralemphasis{optional}) \textendash{} What kind of data to use for creating the spectrum. This
will be passed to TXMStore.get\_map()

\item {} 
\sphinxstyleliteralstrong{index} (\sphinxstyleliteralemphasis{int}\sphinxstyleliteralemphasis{, }\sphinxstyleliteralemphasis{optional}) \textendash{} Which step in the frameset to use. When used to index
store().absorbances, this should return a 3D array like
(energy, rows, columns).

\end{itemize}

\end{description}\end{quote}

\end{fulllineitems}

\index{stage\_transformations() (xanespy.xanes\_frameset.XanesFrameset method)}

\begin{fulllineitems}
\phantomsection\label{\detokenize{xanespy:xanespy.xanes_frameset.XanesFrameset.stage_transformations}}\pysiglinewithargsret{\sphinxbfcode{stage\_transformations}}{\emph{translations=None}, \emph{rotations=None}, \emph{center=(0}, \emph{0)}, \emph{scales=None}}{}
Allows for deferred transformation of the frame data.

Since each transformation introduces interpolation error, the
best results occur when the translations are saved up and then
applied all in one shot. Takes a combination of arrays of
translations (x, y), rotations and/or scales and saves them
for later application. This method should be used in
conjunction apply\_transformations().

All three arguments should have shapes that are compatible
with the frame data, though this is not strictly enforced for
now. Rotation will necessarily have one less degree of freedom
than translation/scale values.

Example Shapes:


\begin{savenotes}\sphinxattablestart
\centering
\begin{tabulary}{\linewidth}[t]{|T|T|T|T|}
\hline
\sphinxstylethead{\sphinxstyletheadfamily 
Frames
\unskip}\relax &\sphinxstylethead{\sphinxstyletheadfamily 
Translations
\unskip}\relax &\sphinxstylethead{\sphinxstyletheadfamily 
Rotations
\unskip}\relax &\sphinxstylethead{\sphinxstyletheadfamily 
Scales
\unskip}\relax \\
\hline
(10, 48, 1024, 1024)
&
(10, 48, 2)
&
(10, 48, 1)
&
(10, 48, 2)
\\
\hline
(10, 48, 1024, 1024, 1024)
&
(10, 48, 3)
&
(10, 48, 2)
&
(10, 48, 3)
\\
\hline
\end{tabulary}
\par
\sphinxattableend\end{savenotes}
\begin{quote}\begin{description}
\item[{Parameters}] \leavevmode\begin{itemize}
\item {} 
\sphinxstyleliteralstrong{translations} (\sphinxstyleliteralemphasis{np.ndarray}) \textendash{} How much to move each axis (x, y{[}, z{]}).

\item {} 
\sphinxstyleliteralstrong{rotations} (\sphinxstyleliteralemphasis{np.ndarray}) \textendash{} How much to rotate around the origin (0, 0) pixel.

\item {} 
\sphinxstyleliteralstrong{center} (\sphinxstyleliteralemphasis{2-tuple}) \textendash{} Where to set the origin of rotation. Default is the first
pixel (0, 0).

\item {} 
\sphinxstyleliteralstrong{scales} (\sphinxstyleliteralemphasis{np.ndarray}) \textendash{} How much to scale the image by in each dimension (x, y{[},
z{]}).

\end{itemize}

\end{description}\end{quote}

\end{fulllineitems}

\index{starttime() (xanespy.xanes\_frameset.XanesFrameset method)}

\begin{fulllineitems}
\phantomsection\label{\detokenize{xanespy:xanespy.xanes_frameset.XanesFrameset.starttime}}\pysiglinewithargsret{\sphinxbfcode{starttime}}{\emph{timeidx=None}}{}
Determine the earliest timestamp amongst all of the frames.
\begin{quote}\begin{description}
\item[{Parameters}] \leavevmode
\sphinxstyleliteralstrong{timeidx} (\sphinxstyleliteralemphasis{int}\sphinxstyleliteralemphasis{, }\sphinxstyleliteralemphasis{optional}) \textendash{} Which timestep to use for finding the start time. If
omitted or None (default), all timesteps will be checked.

\item[{Returns}] \leavevmode
\sphinxstylestrong{start\_time} \textendash{} Naive datetime representing the earliest known frame for
this timeidx. Timezone will always be UTC no matter where
the data were collected.

\item[{Return type}] \leavevmode
np.datetime64

\end{description}\end{quote}

\end{fulllineitems}

\index{store() (xanespy.xanes\_frameset.XanesFrameset method)}

\begin{fulllineitems}
\phantomsection\label{\detokenize{xanespy:xanespy.xanes_frameset.XanesFrameset.store}}\pysiglinewithargsret{\sphinxbfcode{store}}{\emph{mode=’r’}}{}
Get a TXM Store object that saves and retrieves data from the HDF5
file. The mode argument is passed to h5py as is. This method
should be used as a context manager, especially if mode is
something writeable:

\begin{sphinxVerbatim}[commandchars=\\\{\}]
\PYG{c+c1}{\PYGZsh{} The \PYGZsq{}r+\PYGZsq{} creates the file or appends if one exists}
\PYG{k}{with} \PYG{n+nb+bp}{self}\PYG{o}{.}\PYG{n}{store}\PYG{p}{(}\PYG{n}{mode}\PYG{o}{=}\PYG{l+s+s1}{\PYGZsq{}}\PYG{l+s+s1}{r+}\PYG{l+s+s1}{\PYGZsq{}}\PYG{p}{)} \PYG{k}{as} \PYG{n}{store}\PYG{p}{:}
    \PYG{c+c1}{\PYGZsh{} Do stuff with the store...}
    \PYG{n}{img} \PYG{o}{=} \PYG{n}{store}\PYG{o}{.}\PYG{n}{absorbances}\PYG{p}{[}\PYG{l+m+mi}{0}\PYG{p}{,}\PYG{l+m+mi}{0}\PYG{p}{]}
\end{sphinxVerbatim}

\end{fulllineitems}

\index{subtract\_surroundings() (xanespy.xanes\_frameset.XanesFrameset method)}

\begin{fulllineitems}
\phantomsection\label{\detokenize{xanespy:xanespy.xanes_frameset.XanesFrameset.subtract_surroundings}}\pysiglinewithargsret{\sphinxbfcode{subtract\_surroundings}}{}{}
Use the edge mask to separate “surroundings” from “sample”, then
subtract the average surrounding absorbance from each
frame. This effectively removes effects where the entire frame
is brighter from one energy to the next.

\end{fulllineitems}


\end{fulllineitems}



\subsection{xanespy.xanes\_math module}
\label{\detokenize{xanespy:module-xanespy.xanes_math}}\label{\detokenize{xanespy:xanespy-xanes-math-module}}\index{xanespy.xanes\_math (module)}
Module containing all the computationally demanding functions. This
allows for easy optimization of parallelizable algorithms. Most
functions will operate on large arrays of data.
\index{KEdgeParams (class in xanespy.xanes\_math)}

\begin{fulllineitems}
\phantomsection\label{\detokenize{xanespy:xanespy.xanes_math.KEdgeParams}}\pysiglinewithargsret{\sphinxstrong{class }\sphinxcode{xanespy.xanes\_math.}\sphinxbfcode{KEdgeParams}}{\emph{scale}, \emph{voffset}, \emph{E0}, \emph{sigw}, \emph{bg\_slope}, \emph{ga}, \emph{gb}, \emph{gc}}{}
Bases: \sphinxcode{tuple}
\index{E0 (xanespy.xanes\_math.KEdgeParams attribute)}

\begin{fulllineitems}
\phantomsection\label{\detokenize{xanespy:xanespy.xanes_math.KEdgeParams.E0}}\pysigline{\sphinxbfcode{E0}}
Alias for field number 2

\end{fulllineitems}

\index{bg\_slope (xanespy.xanes\_math.KEdgeParams attribute)}

\begin{fulllineitems}
\phantomsection\label{\detokenize{xanespy:xanespy.xanes_math.KEdgeParams.bg_slope}}\pysigline{\sphinxbfcode{bg\_slope}}
Alias for field number 4

\end{fulllineitems}

\index{ga (xanespy.xanes\_math.KEdgeParams attribute)}

\begin{fulllineitems}
\phantomsection\label{\detokenize{xanespy:xanespy.xanes_math.KEdgeParams.ga}}\pysigline{\sphinxbfcode{ga}}
Alias for field number 5

\end{fulllineitems}

\index{gb (xanespy.xanes\_math.KEdgeParams attribute)}

\begin{fulllineitems}
\phantomsection\label{\detokenize{xanespy:xanespy.xanes_math.KEdgeParams.gb}}\pysigline{\sphinxbfcode{gb}}
Alias for field number 6

\end{fulllineitems}

\index{gc (xanespy.xanes\_math.KEdgeParams attribute)}

\begin{fulllineitems}
\phantomsection\label{\detokenize{xanespy:xanespy.xanes_math.KEdgeParams.gc}}\pysigline{\sphinxbfcode{gc}}
Alias for field number 7

\end{fulllineitems}

\index{scale (xanespy.xanes\_math.KEdgeParams attribute)}

\begin{fulllineitems}
\phantomsection\label{\detokenize{xanespy:xanespy.xanes_math.KEdgeParams.scale}}\pysigline{\sphinxbfcode{scale}}
Alias for field number 0

\end{fulllineitems}

\index{sigw (xanespy.xanes\_math.KEdgeParams attribute)}

\begin{fulllineitems}
\phantomsection\label{\detokenize{xanespy:xanespy.xanes_math.KEdgeParams.sigw}}\pysigline{\sphinxbfcode{sigw}}
Alias for field number 3

\end{fulllineitems}

\index{voffset (xanespy.xanes\_math.KEdgeParams attribute)}

\begin{fulllineitems}
\phantomsection\label{\detokenize{xanespy:xanespy.xanes_math.KEdgeParams.voffset}}\pysigline{\sphinxbfcode{voffset}}
Alias for field number 1

\end{fulllineitems}


\end{fulllineitems}

\index{apply\_internal\_reference() (in module xanespy.xanes\_math)}

\begin{fulllineitems}
\phantomsection\label{\detokenize{xanespy:xanespy.xanes_math.apply_internal_reference}}\pysiglinewithargsret{\sphinxcode{xanespy.xanes\_math.}\sphinxbfcode{apply\_internal\_reference}}{\emph{intensities}, \emph{out=None}}{}
Apply a reference correction to complex data to convert intensities
into refractive index. \(I_0\) is determined by separating the pixels
into background and foreground using Otsu’s method.

Arrays \sphinxtitleref{intensities} and \sphinxtitleref{out} must all have the same shape where
the last two dimensions are image rows and column.

\end{fulllineitems}

\index{apply\_mosaic\_reference() (in module xanespy.xanes\_math)}

\begin{fulllineitems}
\phantomsection\label{\detokenize{xanespy:xanespy.xanes_math.apply_mosaic_reference}}\pysiglinewithargsret{\sphinxcode{xanespy.xanes\_math.}\sphinxbfcode{apply\_mosaic\_reference}}{\emph{intensity}, \emph{reference}}{}
Use a single reference frame to calculate the reference for a
mosaic of intensity frames.
\begin{quote}\begin{description}
\item[{Returns}] \leavevmode
\sphinxstylestrong{- out} \textendash{} Same shape as \sphinxcode{intensity} (I) but with optical depth: ln(I/I0)
where I0 is the reference frame.

\item[{Return type}] \leavevmode
ndarray

\item[{Parameters}] \leavevmode\begin{itemize}
\item {} 
\sphinxstyleliteralstrong{intensity} (\sphinxstyleliteralemphasis{-}) \textendash{} Intensity data image of the sample. It’s shape must be a whole
multiple of the shape of \sphinxcode{reference}.

\item {} 
\sphinxstyleliteralstrong{reference} (\sphinxstyleliteralemphasis{-}) \textendash{} A reference image with no sample that will be applied.

\end{itemize}

\end{description}\end{quote}

\end{fulllineitems}

\index{apply\_references() (in module xanespy.xanes\_math)}

\begin{fulllineitems}
\phantomsection\label{\detokenize{xanespy:xanespy.xanes_math.apply_references}}\pysiglinewithargsret{\sphinxcode{xanespy.xanes\_math.}\sphinxbfcode{apply\_references}}{\emph{intensities}, \emph{references}, \emph{out=None}}{}
Apply a reference correction to convert intensity values to optical
depth.

The formula \(-ln\frac{intensities}{references}\) is used to
calculate the new values. Arrays \sphinxcode{intensities}, \sphinxcode{references}
and \sphinxcode{out} must all have the same shape where the last two
dimensions are image rows and columns.
\begin{quote}\begin{description}
\item[{Parameters}] \leavevmode\begin{itemize}
\item {} 
\sphinxstyleliteralstrong{intensities} (\sphinxstyleliteralemphasis{np.ndarray}) \textendash{} Sample input signal data.

\item {} 
\sphinxstyleliteralstrong{references} (\sphinxstyleliteralemphasis{np.ndarray}) \textendash{} Background input signal data. Must be the same shape as
intensities.

\item {} 
\sphinxstyleliteralstrong{out} (\sphinxstyleliteralemphasis{np.ndarray}\sphinxstyleliteralemphasis{, }\sphinxstyleliteralemphasis{optional}) \textendash{} Array to receive the results.

\end{itemize}

\end{description}\end{quote}

\end{fulllineitems}

\index{direct\_whitelines() (in module xanespy.xanes\_math)}

\begin{fulllineitems}
\phantomsection\label{\detokenize{xanespy:xanespy.xanes_math.direct_whitelines}}\pysiglinewithargsret{\sphinxcode{xanespy.xanes\_math.}\sphinxbfcode{direct\_whitelines}}{\emph{spectra}, \emph{energies}, \emph{edge}}{}
Takes an array of X-ray absorbance spectra and calculates the
positions of maximum intensities over the near-edge region.
\begin{quote}\begin{description}
\item[{Parameters}] \leavevmode\begin{itemize}
\item {} 
\sphinxstyleliteralstrong{spectra} (\sphinxstyleliteralemphasis{np.array}) \textendash{} 2D numpy array of absorbance spectra where the last dimension is
energy.

\item {} 
\sphinxstyleliteralstrong{energies} (\sphinxstyleliteralemphasis{np.array}) \textendash{} Array of X-ray energies in electron-volts. Must be broadcastable
to the shape of spectra.

\item {} 
\sphinxstyleliteralstrong{edge} \textendash{} An XAS Edge object that describes the absorbance edge in
question.

\end{itemize}

\item[{Returns}] \leavevmode
\sphinxstylestrong{out} \textendash{} Array with the whiteline position of each spectrum.

\item[{Return type}] \leavevmode
np.ndarray

\end{description}\end{quote}

\end{fulllineitems}

\index{extract\_signals\_nmf() (in module xanespy.xanes\_math)}

\begin{fulllineitems}
\phantomsection\label{\detokenize{xanespy:xanespy.xanes_math.extract_signals_nmf}}\pysiglinewithargsret{\sphinxcode{xanespy.xanes\_math.}\sphinxbfcode{extract\_signals\_nmf}}{\emph{spectra}, \emph{n\_components}, \emph{nmf\_kwargs=None}, \emph{mask=None}}{}
Extract the signal components present in the given spectra using
non-negative matrix factorization. Input data can be negative, but
it will be shifted up, processed, then shifted down again.
\begin{quote}\begin{description}
\item[{Parameters}] \leavevmode\begin{itemize}
\item {} 
\sphinxstyleliteralstrong{spectra} (\sphinxstyleliteralemphasis{-}) \textendash{} 

\item {} 
\sphinxstyleliteralstrong{n\_components} (\sphinxstyleliteralemphasis{-}) \textendash{} 

\item {} 
\sphinxstyleliteralstrong{nmf\_kwargs} (\sphinxstyleliteralemphasis{-}) \textendash{} the constructor of the estimator.

\end{itemize}

\item[{Returns}] \leavevmode


\item[{Return type}] \leavevmode
2-tuple of arrays ({\hyperref[\detokenize{xanespy:xanespy.xanes_frameset.XanesFrameset.components}]{\sphinxcrossref{components}}}, weights)

\end{description}\end{quote}

\end{fulllineitems}

\index{fit\_kedge() (in module xanespy.xanes\_math)}

\begin{fulllineitems}
\phantomsection\label{\detokenize{xanespy:xanespy.xanes_math.fit_kedge}}\pysiglinewithargsret{\sphinxcode{xanespy.xanes\_math.}\sphinxbfcode{fit\_kedge}}{\emph{spectra}, \emph{energies}, \emph{p0}}{}
Use least squares to fit a set of curves to the data. Currently
this is a line for the baseline absorbance decreasing at higher
energies, plus a sigmoid for the edge and a gaussian for the
whiteline.

Returns an array with a similar shape to spectra but the last axis
is replaced with fitting parameters, describe by the named tupled
\sphinxtitleref{KParams} defined in this module.
\begin{quote}\begin{description}
\item[{Parameters}] \leavevmode\begin{itemize}
\item {} 
\sphinxstyleliteralstrong{spectra} (\sphinxstyleliteralemphasis{-}) \textendash{} index is energy. This can be a multi-dimensional array, which
allows calculation of image frames, etc. The last axis should be
X-ray energy.

\item {} 
\sphinxstyleliteralstrong{energies} (\sphinxstyleliteralemphasis{-}) \textendash{} 

\item {} 
\sphinxstyleliteralstrong{p0} (\sphinxstyleliteralemphasis{-}) \textendash{} described by kedge\_params.

\item {} 
\sphinxstyleliteralstrong{out} (\sphinxstyleliteralemphasis{-}) \textendash{} 

\item {} 
\sphinxstyleliteralstrong{be created.} (\sphinxstyleliteralemphasis{will}) \textendash{} 

\end{itemize}

\end{description}\end{quote}

\end{fulllineitems}

\index{fit\_kedge\_mpi() (in module xanespy.xanes\_math)}

\begin{fulllineitems}
\phantomsection\label{\detokenize{xanespy:xanespy.xanes_math.fit_kedge_mpi}}\pysiglinewithargsret{\sphinxcode{xanespy.xanes\_math.}\sphinxbfcode{fit\_kedge\_mpi}}{\emph{spectra}, \emph{energies}, \emph{p0}}{}
Use least squares to fit a set of curves to the data. Very similar
to \sphinxcode{fit\_k\_edge()} except using message passing interface (MPI) for
parallel processing.

Returns an array with a similar shape to spectra but the last axis
is replaced with fitting parameters, described by the named tupled
\sphinxcode{KParams} defined in this module.
\begin{quote}\begin{description}
\item[{Parameters}] \leavevmode\begin{itemize}
\item {} 
\sphinxstyleliteralstrong{spectra} (\sphinxstyleliteralemphasis{np.ndarray}) \textendash{} An array containing absorbance data. Assumes that the index is
energy. This can be a multi-dimensional array, which allows
calculation of image frames, etc. The last axis should be X-ray
energy.

\item {} 
\sphinxstyleliteralstrong{energies} (\sphinxstyleliteralemphasis{np.ndarray}) \textendash{} Array of X-ray energies. Must have same shape as \sphinxtitleref{spectra}.

\item {} 
\sphinxstyleliteralstrong{p0} (\sphinxstyleliteralemphasis{tuple}) \textendash{} A tuple with the initial guess. The correct order is described
by kedge\_params.

\item {} 
\sphinxstyleliteralstrong{out} (\sphinxstyleliteralemphasis{np.ndarray}) \textendash{} To hold the results. If omitted, a new array will be created.

\end{itemize}

\end{description}\end{quote}

\end{fulllineitems}

\index{guess\_kedge() (in module xanespy.xanes\_math)}

\begin{fulllineitems}
\phantomsection\label{\detokenize{xanespy:xanespy.xanes_math.guess_kedge}}\pysiglinewithargsret{\sphinxcode{xanespy.xanes\_math.}\sphinxbfcode{guess\_kedge}}{\emph{spectrum}, \emph{energies}, \emph{edge}}{}
Guess initial starting parameters for a k-edge curve. This will
give a rough estimate, appropriate for giving to the fit\_kedge
function as the starting parameters, p0.
\begin{quote}\begin{description}
\item[{Parameters}] \leavevmode\begin{itemize}
\item {} 
\sphinxstyleliteralstrong{spectrum} (\sphinxstyleliteralemphasis{-}) \textendash{} K-edge spectrum. Only 1-dimensional data are currently accepted.

\item {} 
\sphinxstyleliteralstrong{energies} (\sphinxstyleliteralemphasis{-}) \textendash{} the points in \sphinxtitleref{spectrum}. Must have the same shape as \sphinxtitleref{spectrum}.

\item {} 
\sphinxstyleliteralstrong{edge} (\sphinxstyleliteralemphasis{-}) \textendash{} actual edge energy itself.

\item {} 
\sphinxstyleliteralstrong{Returns} (\sphinxstyleliteralemphasis{A named tuple with the estimated parameters}\sphinxstyleliteralemphasis{ (}\sphinxstyleliteralemphasis{see}) \textendash{} .KEdgeParams for definition)

\end{itemize}

\end{description}\end{quote}

\end{fulllineitems}

\index{iter\_indices() (in module xanespy.xanes\_math)}

\begin{fulllineitems}
\phantomsection\label{\detokenize{xanespy:xanespy.xanes_math.iter_indices}}\pysiglinewithargsret{\sphinxcode{xanespy.xanes\_math.}\sphinxbfcode{iter\_indices}}{\emph{data}, \emph{leftover\_dims=1}, \emph{desc=None}}{}
Accept an array of frames, indices, etc. and generate slices for
each frame. Assumes the last two dimensions of \sphinxtitleref{data} are rows and
columns. All other dimensions will be iterated over.
\begin{itemize}
\item {} 
leftover\_dims : Integer describing which dimensions should not
be iterated over. Eg. if data is 3D array and leftover\_dims == 1,
only first two dimenions will be iterated.

\item {} 
desc : String to put in the progress bar.

\end{itemize}

\end{fulllineitems}

\index{k\_edge\_jump() (in module xanespy.xanes\_math)}

\begin{fulllineitems}
\phantomsection\label{\detokenize{xanespy:xanespy.xanes_math.k_edge_jump}}\pysiglinewithargsret{\sphinxcode{xanespy.xanes\_math.}\sphinxbfcode{k\_edge\_jump}}{\emph{frames: numpy.ndarray}, \emph{energies: numpy.ndarray}, \emph{edge}}{}
Determine what the difference is between the post\_edge and the
pre\_edge.

\end{fulllineitems}

\index{k\_edge\_mask() (in module xanespy.xanes\_math)}

\begin{fulllineitems}
\phantomsection\label{\detokenize{xanespy:xanespy.xanes_math.k_edge_mask}}\pysiglinewithargsret{\sphinxcode{xanespy.xanes\_math.}\sphinxbfcode{k\_edge\_mask}}{\emph{frames: numpy.ndarray}, \emph{energies: numpy.ndarray}, \emph{edge}, \emph{sensitivity: float = 1}, \emph{min\_size=0}}{}
Calculate a mask for what is likely active material at this
edge. This is done by comparing the edge-jump to the standard
deviation. Foreground material will be identified when the
edge-jump accounts for most of the standard deviation.
\begin{quote}\begin{description}
\item[{Parameters}] \leavevmode\begin{itemize}
\item {} 
\sphinxstyleliteralstrong{frames} (\sphinxstyleliteralemphasis{numpy.ndarray}) \textendash{} Array with images at different energies.

\item {} 
\sphinxstyleliteralstrong{energies} (\sphinxstyleliteralemphasis{numpy.ndarray}) \textendash{} X-ray energies corresponding to images in \sphinxcode{frames}. Must have
the same shape along the first dimenion as \sphinxcode{frames}.

\item {} 
\sphinxstyleliteralstrong{edge} ({\hyperref[\detokenize{xanespy:xanespy.edges.KEdge}]{\sphinxcrossref{\sphinxstyleliteralemphasis{KEdge}}}}) \textendash{} A {\hyperref[\detokenize{xanespy:xanespy.edges.KEdge}]{\sphinxcrossref{\sphinxcode{xanespy.edges.KEdge}}}} object that contains a
description of the elemental edge being studied.

\item {} 
\sphinxstyleliteralstrong{sensitivity} (\sphinxstyleliteralemphasis{float}\sphinxstyleliteralemphasis{, }\sphinxstyleliteralemphasis{optional}) \textendash{} A multiplier for the otsu value to determine the actual
threshold.

\item {} 
\sphinxstyleliteralstrong{min\_size} (\sphinxstyleliteralemphasis{int}\sphinxstyleliteralemphasis{, }\sphinxstyleliteralemphasis{optional}) \textendash{} Objects below this size (in pixels) will be removed. Passing
zero (default) will result in no effect.

\end{itemize}

\item[{Returns}] \leavevmode
\sphinxstylestrong{mask} \textendash{} A boolean mask with the same shape as the last two dimensions of
\sphinxcode{frames} where True pixels are likely to be background
material.

\item[{Return type}] \leavevmode
numpy.ndarray

\end{description}\end{quote}

\end{fulllineitems}

\index{l\_edge\_mask() (in module xanespy.xanes\_math)}

\begin{fulllineitems}
\phantomsection\label{\detokenize{xanespy:xanespy.xanes_math.l_edge_mask}}\pysiglinewithargsret{\sphinxcode{xanespy.xanes\_math.}\sphinxbfcode{l\_edge\_mask}}{\emph{frames: numpy.ndarray}, \emph{energies: numpy.ndarray}, \emph{edge}, \emph{sensitivity: float = 1}, \emph{frame\_dims=2}, \emph{min\_size=0}}{}
Calculate a mask for what is likely active material at this
edge. This is done by comparing each spectrum to the overall
spectrum using the dot product. A normalization is first applied
to mitigate differences in total intensity.
\begin{quote}\begin{description}
\item[{Parameters}] \leavevmode\begin{itemize}
\item {} 
\sphinxstyleliteralstrong{frames} (\sphinxstyleliteralemphasis{-}) \textendash{} 

\item {} 
\sphinxstyleliteralstrong{energies} (\sphinxstyleliteralemphasis{-}) \textendash{} \sphinxtitleref{frames}. Must have the same shape along the first dimenion as
\sphinxtitleref{frames}.

\item {} 
\sphinxstyleliteralstrong{edge} (\sphinxstyleliteralemphasis{-}) \textendash{} elemental edge being studied.

\item {} 
\sphinxstyleliteralstrong{sensitivity} (\sphinxstyleliteralemphasis{-}) \textendash{} the actual threshold.

\item {} 
\sphinxstyleliteralstrong{frame\_dims} (\sphinxstyleliteralemphasis{-}) \textendash{} means each frame is a two-dimensional image.

\item {} 
\sphinxstyleliteralstrong{min\_size} (\sphinxstyleliteralemphasis{-}) \textendash{} removed. Passing zero (default) will result in no effect.

\end{itemize}

\item[{Returns}] \leavevmode
\begin{itemize}
\item {} 
\sphinxstyleemphasis{- A boolean mask with the same shape as the last two dimensions of}

\item {} 
\sphinxtitleref{frames} where True pixels are likely to be background material.

\end{itemize}


\end{description}\end{quote}

\end{fulllineitems}

\index{particle\_labels() (in module xanespy.xanes\_math)}

\begin{fulllineitems}
\phantomsection\label{\detokenize{xanespy:xanespy.xanes_math.particle_labels}}\pysiglinewithargsret{\sphinxcode{xanespy.xanes\_math.}\sphinxbfcode{particle\_labels}}{\emph{frames: numpy.ndarray}, \emph{energies: numpy.ndarray}, \emph{edge}, \emph{min\_distance=20}}{}
Prepare a map by segmenting the images into particles.
\begin{quote}\begin{description}
\item[{Parameters}] \leavevmode
\sphinxstyleliteralstrong{frames} (\sphinxstyleliteralemphasis{numpy.ndarray}) \textendash{} An array of images, each one at a different energy. These will
be merged and used for segmentation.

\end{description}\end{quote}

\end{fulllineitems}

\index{predict\_edge() (in module xanespy.xanes\_math)}

\begin{fulllineitems}
\phantomsection\label{\detokenize{xanespy:xanespy.xanes_math.predict_edge}}\pysiglinewithargsret{\sphinxcode{xanespy.xanes\_math.}\sphinxbfcode{predict\_edge}}{\emph{energies}, \emph{*params}}{}
Defines the curve function that gets fit to the data for an
absorbance K-edge.

The predicted curve is a combination of a straight line (for
background), an arctan function (for the edge), and a gaussian
peak (for the whiteline).
\begin{quote}\begin{description}
\item[{Parameters}] \leavevmode\begin{itemize}
\item {} 
\sphinxstyleliteralstrong{energies} (\sphinxstyleliteralemphasis{np.ndarray}) \textendash{} Array with energy values to be predicted.

\item {} 
\sphinxstyleliteralstrong{*params} (\sphinxstyleliteralemphasis{tuple}\sphinxstyleliteralemphasis{(}\sphinxstyleliteralemphasis{int}\sphinxstyleliteralemphasis{)}) \textendash{} The curve parameters that should be used for the
prediction. Their order is described by kedge\_params variable.

\end{itemize}

\item[{Returns}] \leavevmode
\sphinxstylestrong{curve} \textendash{} The predicted absorbance values based on the input
parameters. Shape will match \sphinxtitleref{energies}.

\item[{Return type}] \leavevmode
np.ndarray

\end{description}\end{quote}

\end{fulllineitems}

\index{register\_correlations() (in module xanespy.xanes\_math)}

\begin{fulllineitems}
\phantomsection\label{\detokenize{xanespy:xanespy.xanes_math.register_correlations}}\pysiglinewithargsret{\sphinxcode{xanespy.xanes\_math.}\sphinxbfcode{register\_correlations}}{\emph{frames}, \emph{reference}, \emph{upsample\_factor=10}, \emph{desc=’Registering’}}{}
Calculate the relative translation between the reference image and
a series of frames.

This uses phase correlation through scikit-image’s
\sphinxtitleref{register\_translation} function.
\begin{quote}\begin{description}
\item[{Parameters}] \leavevmode\begin{itemize}
\item {} 
\sphinxstyleliteralstrong{frames} (\sphinxstyleliteralemphasis{np.ndarray}) \textendash{} Array where the last two dimensions are (column, row) of images
to be registered.

\item {} 
\sphinxstyleliteralstrong{reference} (\sphinxstyleliteralemphasis{np.ndarray}) \textendash{} Image frame against which to align the entries in \sphinxtitleref{frames}.

\item {} 
\sphinxstyleliteralstrong{upsample\_factor} (\sphinxstyleliteralemphasis{int}\sphinxstyleliteralemphasis{, }\sphinxstyleliteralemphasis{optional}) \textendash{} Factor controls subpixel registration via scikit-image.

\item {} 
\sphinxstyleliteralstrong{desc} (\sphinxstyleliteralemphasis{str}\sphinxstyleliteralemphasis{, }\sphinxstyleliteralemphasis{optional}) \textendash{} Description for putting in the progress bar.

\end{itemize}

\item[{Returns}] \leavevmode
\sphinxstylestrong{translations} \textendash{} Array with same dimensions as 0-th axis of \sphinxtitleref{frames} containing
(x, y) translations for each frame.

\item[{Return type}] \leavevmode
np.ndarray

\end{description}\end{quote}

\end{fulllineitems}

\index{register\_template() (in module xanespy.xanes\_math)}

\begin{fulllineitems}
\phantomsection\label{\detokenize{xanespy:xanespy.xanes_math.register_template}}\pysiglinewithargsret{\sphinxcode{xanespy.xanes\_math.}\sphinxbfcode{register\_template}}{\emph{frames}, \emph{reference}, \emph{template}, \emph{desc=’Registering’}}{}
Calculate the relative translation between the reference image and
a series of frames.

This uses template cross correlation through scikit-image’s
\sphinxtitleref{match\_template} function.

The \sphinxtitleref{register\_correlations} algorithm is simpler to use in most
cases but sometimes results in unreasonable results; in those
cases, this method can be more reliable to achieve a first
approximation.
\begin{quote}\begin{description}
\item[{Parameters}] \leavevmode\begin{itemize}
\item {} 
\sphinxstyleliteralstrong{frames} (\sphinxstyleliteralemphasis{np.ndarray}) \textendash{} Array where the last two dimensions are (column, row) of images
to be registered.

\item {} 
\sphinxstyleliteralstrong{reference} (\sphinxstyleliteralemphasis{np.ndarray}) \textendash{} Image frame against which to align the entries in \sphinxtitleref{frames}.

\item {} 
\sphinxstyleliteralstrong{template} (\sphinxstyleliteralemphasis{np.ndarray}) \textendash{} A 2D array (smaller than frames and reference) that will be
identified in each frame and used for alignment.

\item {} 
\sphinxstyleliteralstrong{desc} (\sphinxstyleliteralemphasis{str}\sphinxstyleliteralemphasis{, }\sphinxstyleliteralemphasis{optional}) \textendash{} Description for putting in the progress bar.

\end{itemize}

\item[{Returns}] \leavevmode
\sphinxstylestrong{translations} \textendash{} Array with same dimensions as 0-th axis of \sphinxtitleref{frames} containing
(x, y) translations for each frame.

\item[{Return type}] \leavevmode
np.ndarray

\end{description}\end{quote}

\end{fulllineitems}

\index{transform\_images() (in module xanespy.xanes\_math)}

\begin{fulllineitems}
\phantomsection\label{\detokenize{xanespy:xanespy.xanes_math.transform_images}}\pysiglinewithargsret{\sphinxcode{xanespy.xanes\_math.}\sphinxbfcode{transform\_images}}{\emph{data}, \emph{transformations}, \emph{out=None}, \emph{mode=’median’}}{}
Takes image data and applies the given translation matrices.

It is assumed that the first dimension of \sphinxtitleref{data} is the same as
the length of \sphinxtitleref{transformations}. The transformation matrices can
be generated from translation, rotation and scale parameters via
the \sphinxcode{xanespy.xanes\_math.transformation\_matrics()}
function. Data will be written to \sphinxtitleref{out} if given, otherwise
returned as a new array.
\begin{quote}\begin{description}
\item[{Parameters}] \leavevmode\begin{itemize}
\item {} 
\sphinxstyleliteralstrong{data} (\sphinxstyleliteralemphasis{np.ndarray}) \textendash{} Numeric array with frames to transform. Last two dimensions are
assumed to be (row, columns).

\item {} 
\sphinxstyleliteralstrong{transformations} (\sphinxstyleliteralemphasis{np.ndarray}) \textendash{} A numeric array shaped compatibally with \sphinxtitleref{data}. The last two
dimensions are assumed to be (3, 3) and each (3, 3) encodes a
transformation matrix for the corresponding frame in \sphinxtitleref{data}.

\item {} 
\sphinxstyleliteralstrong{out} (\sphinxstyleliteralemphasis{np.ndarray}\sphinxstyleliteralemphasis{, }\sphinxstyleliteralemphasis{optional}) \textendash{} A numeric array with same shape as \sphinxtitleref{data} that will hold the
transformed data.

\item {} 
\sphinxstyleliteralstrong{mode} (\sphinxstyleliteralemphasis{str}\sphinxstyleliteralemphasis{, }\sphinxstyleliteralemphasis{optional}) \textendash{} Describes how to deal with edges. See scikit-image documentation
for options. Special value “median” (default), takes the median
pixel intensity of that frame and uses it as the constant value.

\end{itemize}

\item[{Returns}] \leavevmode
\sphinxstylestrong{out} \textendash{} A new array with similar dimensions to \sphinxtitleref{data} but with
transformations applied and converted to float datatype.

\item[{Return type}] \leavevmode
np.ndarray

\end{description}\end{quote}

\end{fulllineitems}

\index{transformation\_matrices() (in module xanespy.xanes\_math)}

\begin{fulllineitems}
\phantomsection\label{\detokenize{xanespy:xanespy.xanes_math.transformation_matrices}}\pysiglinewithargsret{\sphinxcode{xanespy.xanes\_math.}\sphinxbfcode{transformation\_matrices}}{\emph{translations=None}, \emph{rotations=None}, \emph{scales=None}, \emph{center=(0}, \emph{0)}}{}
Takes array of operations and calculates (3, 3) transformation
matrices.

This function operates by calculating an AffineTransform similar
to that described in the scikit-image package.

All three arguments (\sphinxtitleref{translations}, \sphinxtitleref{rotations}, and \sphinxtitleref{scales},
should have shapes that are compatible with the frame data, though
this is not strictly enforced for now. Rotation will necessarily
have one less degree of freedom than translation/scale values.

Example Shapes:


\begin{savenotes}\sphinxattablestart
\centering
\begin{tabulary}{\linewidth}[t]{|T|T|T|T|}
\hline
\sphinxstylethead{\sphinxstyletheadfamily 
Frames
\unskip}\relax &\sphinxstylethead{\sphinxstyletheadfamily 
Translations
\unskip}\relax &\sphinxstylethead{\sphinxstyletheadfamily 
Rotations
\unskip}\relax &\sphinxstylethead{\sphinxstyletheadfamily 
Scales
\unskip}\relax \\
\hline
(10, 48, 1024, 1024)
&
(10, 48, 2)
&
(10, 48, 1)
&
(10, 48, 2)
\\
\hline
(10, 48, 1024, 1024, 1024)
&
(10, 48, 3)
&
(10, 48, 2)
&
(10, 48, 3)
\\
\hline
\end{tabulary}
\par
\sphinxattableend\end{savenotes}
\begin{quote}\begin{description}
\item[{Parameters}] \leavevmode\begin{itemize}
\item {} 
\sphinxstyleliteralstrong{translations} (\sphinxstyleliteralemphasis{np.ndarray}\sphinxstyleliteralemphasis{, }\sphinxstyleliteralemphasis{optional}) \textendash{} How much to move each axis (x, y{[}, z{]}).

\item {} 
\sphinxstyleliteralstrong{rotations} (\sphinxstyleliteralemphasis{np.ndarray}\sphinxstyleliteralemphasis{, }\sphinxstyleliteralemphasis{optional}) \textendash{} How much to rotate around the origin (0, 0) pixel.

\item {} 
\sphinxstyleliteralstrong{center} (\sphinxstyleliteralemphasis{np.ndarray}\sphinxstyleliteralemphasis{, }\sphinxstyleliteralemphasis{optional}) \textendash{} Where to set the origin of rotation. Default is the
first pixel (0, 0).

\item {} 
\sphinxstyleliteralstrong{scales} (\sphinxstyleliteralemphasis{np.ndarray}\sphinxstyleliteralemphasis{, }\sphinxstyleliteralemphasis{optional}) \textendash{} How much to scale the image by in each dimension
(x, y{[}, z{]}).

\end{itemize}

\item[{Returns}] \leavevmode
\sphinxstylestrong{new\_transforms} \textendash{} Resulting transformation matrices. Will have the same shape as the
input arrays but with the last dimension replaced by (3, 3).

\item[{Return type}] \leavevmode
np.ndarray

\end{description}\end{quote}

\end{fulllineitems}



\subsection{xanespy.xradia module}
\label{\detokenize{xanespy:xanespy-xradia-module}}\label{\detokenize{xanespy:module-xanespy.xradia}}\index{xanespy.xradia (module)}
Tools for importing X-ray microscopy frames in formats produced by
Xradia instruments.
\index{TXRMFile (class in xanespy.xradia)}

\begin{fulllineitems}
\phantomsection\label{\detokenize{xanespy:xanespy.xradia.TXRMFile}}\pysiglinewithargsret{\sphinxstrong{class }\sphinxcode{xanespy.xradia.}\sphinxbfcode{TXRMFile}}{\emph{filename}, \emph{flavor: str}}{}
Bases: {\hyperref[\detokenize{xanespy:xanespy.xradia.XRMFile}]{\sphinxcrossref{\sphinxcode{xanespy.xradia.XRMFile}}}}

Similar to {\hyperref[\detokenize{xanespy:xanespy.xradia.XRMFile}]{\sphinxcrossref{\sphinxcode{XRMFile}}}} but contains multiple images in a data-set.
\index{energies() (xanespy.xradia.TXRMFile method)}

\begin{fulllineitems}
\phantomsection\label{\detokenize{xanespy:xanespy.xradia.TXRMFile.energies}}\pysiglinewithargsret{\sphinxbfcode{energies}}{}{}
\end{fulllineitems}

\index{image\_stack() (xanespy.xradia.TXRMFile method)}

\begin{fulllineitems}
\phantomsection\label{\detokenize{xanespy:xanespy.xradia.TXRMFile.image_stack}}\pysiglinewithargsret{\sphinxbfcode{image\_stack}}{}{}
\end{fulllineitems}


\end{fulllineitems}

\index{XRMFile (class in xanespy.xradia)}

\begin{fulllineitems}
\phantomsection\label{\detokenize{xanespy:xanespy.xradia.XRMFile}}\pysiglinewithargsret{\sphinxstrong{class }\sphinxcode{xanespy.xradia.}\sphinxbfcode{XRMFile}}{\emph{filename}, \emph{flavor: str}}{}
Bases: \sphinxcode{object}

Single X-ray micrscopy frame created using XRadia XRM format.

Formats from different beamlines have subtly different storage
patterns. The \sphinxcode{flavor} argument controls this parameter. The
following metadata are affected by this coice:
\begin{itemize}
\item {} 
Energy: SSRL 6-2c does not store the X-ray beam energy in the
file so it must be extracted from the filename.

\item {} 
starttime, endtime : The XRMFile does not store timezone info,
so we have to guess based on beamline.

\item {} 
pixel\_size : The APS microscope automatically corrects
magnification when changing energy, the SSRL microscope does
not.

\end{itemize}
\begin{quote}\begin{description}
\item[{Parameters}] \leavevmode\begin{itemize}
\item {} 
\sphinxstyleliteralstrong{filename} (\sphinxstyleliteralemphasis{str}) \textendash{} The path to the .xrm file

\item {} 
\sphinxstyleliteralstrong{flavor} (\sphinxstyleliteralemphasis{str}) \textendash{} The variety of data represented in the xrm file. Valid
choices are {[}‘ssrl’, ‘aps’, ‘aps-old1’{]}. These choices should
line up with whatever is generated using the scripts in
beamlines moudles.

\end{itemize}

\end{description}\end{quote}
\index{aps\_old1\_regex (xanespy.xradia.XRMFile attribute)}

\begin{fulllineitems}
\phantomsection\label{\detokenize{xanespy:xanespy.xradia.XRMFile.aps_old1_regex}}\pysigline{\sphinxbfcode{aps\_old1\_regex}\sphinxstrong{ = re.compile(‘(\textbackslash{}d\{8\})\_({[}a-zA-Z0-9\_{]}+)\_({[}a-zA-Z0-9{]}+)\_(\textbackslash{}d\{4\}).xrm’)}}
\end{fulllineitems}

\index{binning() (xanespy.xradia.XRMFile method)}

\begin{fulllineitems}
\phantomsection\label{\detokenize{xanespy:xanespy.xradia.XRMFile.binning}}\pysiglinewithargsret{\sphinxbfcode{binning}}{}{}
Binning mode of the image.

Binning reduces pixel density but improves signal/noise ratio
by combining adjacent pixels. For example, a 2048 x 2048 CCD
with binning 4 would produce a 512 x 512 image.

\end{fulllineitems}

\index{close() (xanespy.xradia.XRMFile method)}

\begin{fulllineitems}
\phantomsection\label{\detokenize{xanespy:xanespy.xradia.XRMFile.close}}\pysiglinewithargsret{\sphinxbfcode{close}}{}{}
Close original XRM (ole) file on disk.

\end{fulllineitems}

\index{endtime() (xanespy.xradia.XRMFile method)}

\begin{fulllineitems}
\phantomsection\label{\detokenize{xanespy:xanespy.xradia.XRMFile.endtime}}\pysiglinewithargsret{\sphinxbfcode{endtime}}{}{}
Retrieve a datetime object representing when this frame was
finished collecting. Duration is decided by exposure time of the frame
and the start time.

\end{fulllineitems}

\index{energy() (xanespy.xradia.XRMFile method)}

\begin{fulllineitems}
\phantomsection\label{\detokenize{xanespy:xanespy.xradia.XRMFile.energy}}\pysiglinewithargsret{\sphinxbfcode{energy}}{}{}
Beam energy in electronvoltes.

\end{fulllineitems}

\index{image\_data() (xanespy.xradia.XRMFile method)}

\begin{fulllineitems}
\phantomsection\label{\detokenize{xanespy:xanespy.xradia.XRMFile.image_data}}\pysiglinewithargsret{\sphinxbfcode{image\_data}}{\emph{idx=0}}{}
TXM Image frame.

\end{fulllineitems}

\index{image\_dtype() (xanespy.xradia.XRMFile method)}

\begin{fulllineitems}
\phantomsection\label{\detokenize{xanespy:xanespy.xradia.XRMFile.image_dtype}}\pysiglinewithargsret{\sphinxbfcode{image\_dtype}}{}{}
\end{fulllineitems}

\index{image\_shape() (xanespy.xradia.XRMFile method)}

\begin{fulllineitems}
\phantomsection\label{\detokenize{xanespy:xanespy.xradia.XRMFile.image_shape}}\pysiglinewithargsret{\sphinxbfcode{image\_shape}}{}{}
\end{fulllineitems}

\index{is\_valid() (xanespy.xradia.XRMFile method)}

\begin{fulllineitems}
\phantomsection\label{\detokenize{xanespy:xanespy.xradia.XRMFile.is_valid}}\pysiglinewithargsret{\sphinxbfcode{is\_valid}}{}{}
Check that the XRM file has valid data.

\end{fulllineitems}

\index{num\_images() (xanespy.xradia.XRMFile method)}

\begin{fulllineitems}
\phantomsection\label{\detokenize{xanespy:xanespy.xradia.XRMFile.num_images}}\pysiglinewithargsret{\sphinxbfcode{num\_images}}{}{}
\end{fulllineitems}

\index{ole\_value() (xanespy.xradia.XRMFile method)}

\begin{fulllineitems}
\phantomsection\label{\detokenize{xanespy:xanespy.xradia.XRMFile.ole_value}}\pysiglinewithargsret{\sphinxbfcode{ole\_value}}{\emph{stream}, \emph{fmt=None}, \emph{as\_array=False}}{}
Get arbitrary data from the ole file and convert from bytes.

\end{fulllineitems}

\index{print\_ole() (xanespy.xradia.XRMFile method)}

\begin{fulllineitems}
\phantomsection\label{\detokenize{xanespy:xanespy.xradia.XRMFile.print_ole}}\pysiglinewithargsret{\sphinxbfcode{print\_ole}}{}{}
\end{fulllineitems}

\index{sample\_position() (xanespy.xradia.XRMFile method)}

\begin{fulllineitems}
\phantomsection\label{\detokenize{xanespy:xanespy.xradia.XRMFile.sample_position}}\pysiglinewithargsret{\sphinxbfcode{sample\_position}}{}{}
\end{fulllineitems}

\index{starttime() (xanespy.xradia.XRMFile method)}

\begin{fulllineitems}
\phantomsection\label{\detokenize{xanespy:xanespy.xradia.XRMFile.starttime}}\pysiglinewithargsret{\sphinxbfcode{starttime}}{}{}
Return the earliest timestamp for the collected frames.

\end{fulllineitems}

\index{starttimes() (xanespy.xradia.XRMFile method)}

\begin{fulllineitems}
\phantomsection\label{\detokenize{xanespy:xanespy.xradia.XRMFile.starttimes}}\pysiglinewithargsret{\sphinxbfcode{starttimes}}{}{}
Retrieve all datetime objects representing when these frames were
collected. Timezone is inferred from flavor (eg. ssrl -\textgreater{}
california time).

\end{fulllineitems}

\index{um\_per\_pixel() (xanespy.xradia.XRMFile method)}

\begin{fulllineitems}
\phantomsection\label{\detokenize{xanespy:xanespy.xradia.XRMFile.um_per_pixel}}\pysiglinewithargsret{\sphinxbfcode{um\_per\_pixel}}{}{}
Describe the size of a pixel in microns. If this is an SSRL frame,
the pixel size is dependent on energy. For APS frames, the pixel size
is uniform and assumes a 40µm field-of-view.

\end{fulllineitems}


\end{fulllineitems}



\subsection{Module contents}
\label{\detokenize{xanespy:module-contents}}\label{\detokenize{xanespy:module-xanespy}}\index{xanespy (module)}

\chapter{Indices and tables}
\label{\detokenize{index:indices-and-tables}}\begin{itemize}
\item {} 
\DUrole{xref,std,std-ref}{genindex}

\item {} 
\DUrole{xref,std,std-ref}{modindex}

\item {} 
\DUrole{xref,std,std-ref}{search}

\end{itemize}


\renewcommand{\indexname}{Python Module Index}
\begin{sphinxtheindex}
\def\bigletter#1{{\Large\sffamily#1}\nopagebreak\vspace{1mm}}
\bigletter{x}
\item {\sphinxstyleindexentry{xanespy}}\sphinxstyleindexpageref{xanespy:\detokenize{module-xanespy}}
\item {\sphinxstyleindexentry{xanespy.beamlines}}\sphinxstyleindexpageref{xanespy:\detokenize{module-xanespy.beamlines}}
\item {\sphinxstyleindexentry{xanespy.edges}}\sphinxstyleindexpageref{xanespy:\detokenize{module-xanespy.edges}}
\item {\sphinxstyleindexentry{xanespy.exceptions}}\sphinxstyleindexpageref{xanespy:\detokenize{module-xanespy.exceptions}}
\item {\sphinxstyleindexentry{xanespy.importers}}\sphinxstyleindexpageref{xanespy:\detokenize{module-xanespy.importers}}
\item {\sphinxstyleindexentry{xanespy.plots}}\sphinxstyleindexpageref{xanespy:\detokenize{module-xanespy.plots}}
\item {\sphinxstyleindexentry{xanespy.qt\_frame\_view}}\sphinxstyleindexpageref{xanespy:\detokenize{module-xanespy.qt_frame_view}}
\item {\sphinxstyleindexentry{xanespy.qt\_frameset\_presenter}}\sphinxstyleindexpageref{xanespy:\detokenize{module-xanespy.qt_frameset_presenter}}
\item {\sphinxstyleindexentry{xanespy.qt\_map\_view}}\sphinxstyleindexpageref{xanespy:\detokenize{module-xanespy.qt_map_view}}
\item {\sphinxstyleindexentry{xanespy.txmstore}}\sphinxstyleindexpageref{xanespy:\detokenize{module-xanespy.txmstore}}
\item {\sphinxstyleindexentry{xanespy.utilities}}\sphinxstyleindexpageref{xanespy:\detokenize{module-xanespy.utilities}}
\item {\sphinxstyleindexentry{xanespy.xanes\_frameset}}\sphinxstyleindexpageref{xanespy:\detokenize{module-xanespy.xanes_frameset}}
\item {\sphinxstyleindexentry{xanespy.xanes\_math}}\sphinxstyleindexpageref{xanespy:\detokenize{module-xanespy.xanes_math}}
\item {\sphinxstyleindexentry{xanespy.xradia}}\sphinxstyleindexpageref{xanespy:\detokenize{module-xanespy.xradia}}
\end{sphinxtheindex}

\renewcommand{\indexname}{Index}
\printindex
\end{document}