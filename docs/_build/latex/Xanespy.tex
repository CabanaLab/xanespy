%% Generated by Sphinx.
\def\sphinxdocclass{report}
\documentclass[letterpaper,10pt,english]{sphinxmanual}
\ifdefined\pdfpxdimen
   \let\sphinxpxdimen\pdfpxdimen\else\newdimen\sphinxpxdimen
\fi \sphinxpxdimen=49336sp\relax

\usepackage[margin=1in,marginparwidth=0.5in]{geometry}
\usepackage[utf8]{inputenc}
\ifdefined\DeclareUnicodeCharacter
  \DeclareUnicodeCharacter{00A0}{\nobreakspace}
\fi
\usepackage{cmap}
\usepackage[T1]{fontenc}
\usepackage{amsmath,amssymb,amstext}
\usepackage{babel}
\usepackage{times}
\usepackage[Bjarne]{fncychap}
\usepackage{longtable}
\usepackage{sphinx}

\usepackage{multirow}
\usepackage{eqparbox}

% Include hyperref last.
\usepackage{hyperref}
% Fix anchor placement for figures with captions.
\usepackage{hypcap}% it must be loaded after hyperref.
% Set up styles of URL: it should be placed after hyperref.
\urlstyle{same}
\addto\captionsenglish{\renewcommand{\contentsname}{Contents:}}

\addto\captionsenglish{\renewcommand{\figurename}{Fig.\@ }}
\addto\captionsenglish{\renewcommand{\tablename}{Table }}
\addto\captionsenglish{\renewcommand{\literalblockname}{Listing }}

\addto\extrasenglish{\def\pageautorefname{page}}

\setcounter{tocdepth}{1}



\title{Xanespy Documentation}
\date{Jan 08, 2017}
\release{0.1}
\author{Mark Wolf}
\newcommand{\sphinxlogo}{}
\renewcommand{\releasename}{Release}
\makeindex

\begin{document}

\maketitle
\sphinxtableofcontents
\phantomsection\label{\detokenize{index::doc}}


Python library for analyzing \sphinxstylestrong{X-Ray absorbance spectroscopy} data.

\begin{sphinxadmonition}{warning}{Warning:}
This documentation is still under development. The code
it represents could change at any time.
\end{sphinxadmonition}


\chapter{Introduction}
\label{\detokenize{intro::doc}}\label{\detokenize{intro:introduction}}\label{\detokenize{intro:xanespy-documentation}}
Xanespy is a toolkit for interacting with X-ray microscopy data, most
likely collected at a synchrotron beamline. By collecting a set of
frames at multiple X-ray energies, spectral maps are reconstructed to
provide chemical insight. Multiple framesets can be collected
sequentially as part of an \sphinxstyleemphasis{operando} experiment and analyzed
simultaneously in python. Slow operations take advantage of multiple
cores when available.


\section{X-Ray Absorbance Basics}
\label{\detokenize{intro:x-ray-absorbance-basics}}

\section{Example Workflow}
\label{\detokenize{intro:example-workflow}}
A typical prcocedure for interacting with microscope frame-sets involves the following parts:
\begin{itemize}
\item {} 
Import the raw data

\item {} 
Apply corrections and align the images

\item {} 
Calculate some metric and create maps of it

\item {} 
Visualize the maps, static or interactively.

\end{itemize}

Example for a single frameset across an X-ray absorbance edge:

\begin{sphinxVerbatim}[commandchars=\\\{\}]
\PYG{k+kn}{import} \PYG{n+nn}{xanespy}

\PYG{c+c1}{\PYGZsh{} Example for importing from SSRL beamline 6\PYGZhy{}2c}
\PYG{n}{xanespy}\PYG{o}{.}\PYG{n}{import\PYGZus{}ssrl\PYGZus{}frameset}\PYG{p}{(}\PYG{l+s+s1}{\PYGZsq{}}\PYG{l+s+s1}{\PYGZlt{}data\PYGZus{}dir\PYGZgt{}}\PYG{l+s+s1}{\PYGZsq{}}\PYG{p}{,} \PYG{n}{hdf\PYGZus{}filename}\PYG{o}{=}\PYG{l+s+s1}{\PYGZsq{}}\PYG{l+s+s1}{imported\PYGZus{}data.h5}\PYG{l+s+s1}{\PYGZsq{}}\PYG{p}{)}

\PYG{c+c1}{\PYGZsh{} Load a pre\PYGZhy{}defined XAS edge or create your own subclass xanespy.Edge}
\PYG{n}{edge} \PYG{o}{=} \PYG{n}{xanespy}\PYG{o}{.}\PYG{n}{k\PYGZus{}edges}\PYG{p}{[}\PYG{l+s+s1}{\PYGZsq{}}\PYG{l+s+s1}{Ni\PYGZus{}NCA}\PYG{l+s+s1}{\PYGZsq{}}\PYG{p}{]}
\PYG{c+c1}{\PYGZsh{} Now load the newly created HDF5 file and the X\PYGZhy{}ray absorbance edge}
\PYG{n}{fs} \PYG{o}{=} \PYG{n}{xanespy}\PYG{o}{.}\PYG{n}{XanesFrameset}\PYG{p}{(}\PYG{n}{filename}\PYG{o}{=}\PYG{l+s+s1}{\PYGZsq{}}\PYG{l+s+s1}{imported\PYGZus{}data.h5}\PYG{l+s+s1}{\PYGZsq{}}\PYG{p}{,} \PYG{n}{edge}\PYG{o}{=}\PYG{n}{edge}\PYG{p}{)}

\PYG{c+c1}{\PYGZsh{} Perform automatic frame alignment}
\PYG{n}{fs}\PYG{o}{.}\PYG{n}{align\PYGZus{}frames}\PYG{p}{(}\PYG{n}{passes}\PYG{o}{=}\PYG{l+m+mi}{5}\PYG{p}{)}
\PYG{c+c1}{\PYGZsh{} Fit the absorbance spectra and extract the edge position (SLOW!)}
\PYG{n}{fs}\PYG{o}{.}\PYG{n}{fit\PYGZus{}spectra}\PYG{p}{(}\PYG{p}{)}

\PYG{c+c1}{\PYGZsh{} Inspect the result with the built\PYGZhy{}in Qt5 GUI}
\PYG{n}{fs}\PYG{o}{.}\PYG{n}{qt\PYGZus{}viewer}\PYG{p}{(}\PYG{p}{)}
\end{sphinxVerbatim}


\chapter{Importing Data into Xanespy}
\label{\detokenize{importing::doc}}\label{\detokenize{importing:importing-data-into-xanespy}}

\section{TXM from 8-BM (APS) and 6-2c (SSRL)}
\label{\detokenize{importing:txm-from-8-bm-aps-and-6-2c-ssrl}}

\section{Ptychography from 5.3.2.1 (ALS)}
\label{\detokenize{importing:ptychography-from-5-3-2-1-als}}

\chapter{Analyzing the Data}
\label{\detokenize{analysis::doc}}\label{\detokenize{analysis:analyzing-the-data}}

\section{Frame Alignment}
\label{\detokenize{analysis:frame-alignment}}

\section{Subtracting Surroundings}
\label{\detokenize{analysis:subtracting-surroundings}}

\section{Spectrum Fitting - K-Edge}
\label{\detokenize{analysis:spectrum-fitting-k-edge}}

\section{Spectrum Fitting - L-Edge}
\label{\detokenize{analysis:spectrum-fitting-l-edge}}

\chapter{Visualization of Results}
\label{\detokenize{visualization::doc}}\label{\detokenize{visualization:visualization-of-results}}

\section{Plotting}
\label{\detokenize{visualization:plotting}}

\section{Interactive Viewer (Qt)}
\label{\detokenize{visualization:interactive-viewer-qt}}

\chapter{Accessing the Data Directly}
\label{\detokenize{data_stores::doc}}\label{\detokenize{data_stores:accessing-the-data-directly}}
While the \sphinxcode{XanesFrameset} class has methods for common tasks,
sometimes it is necessary to access the data directly, as either numpy
arrays or h5py datasets. The xanes\_frameset has a \sphinxcode{store()} method
that returns an interface (\sphinxcode{TXMStore}) to the underlying HDF5 file.

\begin{sphinxadmonition}{warning}{Warning:}
The \sphinxcode{TXMStore} created by \sphinxcode{xanes\_frameset.store()} is
attached to an open HDF5 file. It is strongly recommended to use
the \sphinxcode{with} statement described below. Otherwise make sure to call
the store's \sphinxcode{close()} method in a \sphinxcode{try...except}
block. File corruption is possible if not opened in this manner.
\end{sphinxadmonition}

Call the following to get access to the associated datasets. Properties of the interface will
return an HDF5 dataset in most cases.:

\begin{sphinxVerbatim}[commandchars=\\\{\}]
\PYG{k+kn}{import} \PYG{n+nn}{xanespy} \PYG{k}{as} \PYG{n+nn}{xp}
\PYG{n}{frameset} \PYG{o}{=} \PYG{n}{xp}\PYG{o}{.}\PYG{n}{XanesFrameset}\PYG{p}{(}\PYG{o}{.}\PYG{o}{.}\PYG{o}{.}\PYG{p}{)}

\PYG{c+c1}{\PYGZsh{} Open the TXMStore interface}
\PYG{k}{with} \PYG{n}{frameset}\PYG{o}{.}\PYG{n}{store}\PYG{p}{(}\PYG{p}{)} \PYG{k}{as} \PYG{n}{store}\PYG{p}{:}
    \PYG{c+c1}{\PYGZsh{} For example, the images are in (timestep, energy, row, column) order}
    \PYG{k}{assert} \PYG{n}{store}\PYG{o}{.}\PYG{n}{absorbances}\PYG{o}{.}\PYG{n}{shape} \PYG{o}{==} \PYG{p}{(}\PYG{l+m+mi}{10}\PYG{p}{,} \PYG{l+m+mi}{62}\PYG{p}{,} \PYG{l+m+mi}{1024}\PYG{p}{,} \PYG{l+m+mi}{1024}\PYG{p}{)}
    \PYG{c+c1}{\PYGZsh{} Energies are in (timestep, energy) order}
    \PYG{k}{assert} \PYG{n}{store}\PYG{o}{.}\PYG{n}{energies}\PYG{o}{.}\PYG{n}{shape} \PYG{o}{==} \PYG{p}{(}\PYG{l+m+mi}{10}\PYG{p}{,} \PYG{l+m+mi}{62}\PYG{p}{)}
\end{sphinxVerbatim}


\chapter{Indices and tables}
\label{\detokenize{index:indices-and-tables}}\begin{itemize}
\item {} 
\DUrole{xref,std,std-ref}{genindex}

\item {} 
\DUrole{xref,std,std-ref}{modindex}

\item {} 
\DUrole{xref,std,std-ref}{search}

\end{itemize}



\renewcommand{\indexname}{Index}
\printindex
\end{document}